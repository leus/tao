\documentclass[hidelinks]{memoir}

\usepackage{ccicons}

\usepackage[T1]{fontenc}
\usepackage{mathpazo}
\usepackage{url}
\usepackage{comment}
\usepackage[svgnames]{xcolor}
\ifpdf
\usepackage{pdfcolmk}
\fi
\usepackage{graphicx}
\usepackage{hyperref}
%\usepackage[outline]{contour}
\usepackage{tikz}
\usepackage{pifont}


%\usepackage{pst-text}

%%%% Additional font macros
\makeatletter
%%%% light series
%% e.g., s:12
\DeclareRobustCommand\ltseries
{\not@math@alphabet\ltseries\relax
	\fontseries\ltdefault\selectfont}
%% e.g., t:32
\newcommand{\ltdefault}{l}
%% e.g., v:19
\DeclareTextFontCommand{\textlt}{\ltseries}

% heavy(bold) series
\DeclareRobustCommand\hbseries
{\not@math@alphabet\hbseries\relax
	\fontseries\hbdefault\selectfont}
%% e.g., t:32
\newcommand{\hbdefault}{hb}
%% e.g., v:19
\DeclareTextFontCommand{\texthb}{\hbseries}
\makeatother

\newcommand*{\isbn}{{\small\textsc{ISBN}}}

%%% for the Web-O-Mints fonts
\newcommand*{\wb}[2]{\fontsize{#1}{#2}\usefont{U}{webo}{xl}{n}}
%\renewcommand*{\wb}[2]{}%    probably kills Web-O-Mints (and some layouts?)
%%% for the Fontsite 500 fonts
\newcommand*{\FSfont}[1]{%
	\fontencoding{T1}\fontfamily{#1}\selectfont}
%\renewcommand*{\FSfont}[1]{}%    kills special font selections

\newcommand*{\labelit}[1]{\phantomsection\label{#1}}
\newcommand*{\refit}[1]{(graphic on page~\pageref{#1})}

\chapterstyle{dash}\renewcommand*{\chaptitlefont}{\normalfont\itshape\LARGE}
\setlength{\beforechapskip}{2\onelineskip}
\setsecheadstyle{\normalfont\Large\raggedright}
\makeindex
\renewcommand*{\indexname}{Index of Designers}
\makeatletter
\newcommand*{\boxminipage}{%
	\@ifnextchar [%]
	\@ibxminipage
	{\@iiibxminipage c\relax[s]}}
\def\@ibxminipage[#1]{%
	\@ifnextchar [%]
	{\@iibxminipage{#1}}%
	{\@iiibxminipage{#1}\relax[s]}}
\def\@iibxminipage#1[#2]{%
	\@ifnextchar [%]
	{\@iiibxminipage{#1}{#2}}%
	{\@iiibxminipage{#1}{#2}[#1]}}
\let\@bxminto\@empty
\def\@iiibxminipage#1#2[#3]#4{%
	\ifx\relax#2\else
	\setlength\@tempdimb{#2}%
	\def\@bxminto{to\@tempdimb}%
	\fi
	\leavevmode
	\@pboxswfalse
	\if #1b\vbox
	\else
	\if #1t\vtop
	\else
	\ifmmode \vcenter
	\else \@pboxswtrue $\vcenter
	\fi
	\fi
	\fi
	%  \@bxminto
	\bgroup%          outermost vbox
	\hsize #4
	\hrule\@height\fboxrule
	\hbox\bgroup%   inner hbox
	\vrule\@width\fboxrule \hskip\fboxsep 
	\vbox \@bxminto
	\bgroup% innermost vbox
	\vskip\fboxsep
	\advance\hsize -2\fboxrule \advance\hsize -2\fboxsep
	\textwidth\hsize \columnwidth\hsize
	\@parboxrestore
	\def\@mpfn{mpfootnote}\def\thempfn{\thempfootnote}\c@mpfootnote\z@
	\let\@footnotetext\@mpfootnotetext
	\let\@listdepth\@mplistdepth \@mplistdepth\z@
	\@minipagerestore\@minipagetrue
	\everypar{\global\@minipagefalse\everypar{}}}

\def\endboxminipage{%
	\par\vskip-\lastskip
	\ifvoid\@mpfootins\else
	\vskip\skip\@mpfootins\footnoterule\unvbox\@mpfootins\fi
	\vskip\fboxsep
	\egroup%    end innermost vbox
	\hskip\fboxsep \vrule\@width\fboxrule
	\egroup%    end hbox
	\hrule\@height\fboxrule
	\egroup%    end outermost vbox
	\if@pboxsw $\fi}
\makeatother

\DeclareRobustCommand{\cs}[1]{\texttt{\char`\\#1}}
\newlength{\tpheight}\setlength{\tpheight}{0.9\textheight}
\newlength{\txtheight}\setlength{\txtheight}{0.9\tpheight}
\newlength{\tpwidth}\setlength{\tpwidth}{0.9\textwidth}
\newlength{\txtwidth}\setlength{\txtwidth}{0.9\tpwidth}
\newlength{\drop}

\newenvironment{showtitle}{%
	\begin{boxminipage}[c][\tpheight]{\tpwidth}
		\centering\begin{vplace}\begin{minipage}[c][\txtheight]{\txtwidth}}%
			{\end{minipage}\end{vplace}\end{boxminipage}}

\definecolor{Dark}{gray}{.2}
\definecolor{MedDark}{gray}{.4}
\definecolor{Medium}{gray}{.6}
\definecolor{Light}{gray}{.8}



\begin{document}
	\title{Tao Te King\thanks{Lao Tse}}
	\author{
		John Thomas Wilke	\and
		Juan Tomás Bradanovic Pozo\\
		The Nectar Project}

\begin{titlingpage}

\begingroup% Robert Frost, T&H p 149
%\FSfont{5bo}%  Bembo/Bergamo
\drop = 0.2\txtheight
\centering
\vfill
{\Huge El Tao Te King de}\\[\baselineskip]
{\Huge Lao Tse}\\[\baselineskip]
{\large Traducido al Español desde la Traducción al Inglés de Gia-fu Feng y Jane English}\\[0.5\drop]

{\Large The Nectar Project}\par
{\large\scshape 2019}\par
\vfill\null
\endgroup

\end{titlingpage}
	
\frontmatter


	
	\chapter*{Nota Aclaratoria}
	
	Un autor declara que su obra es original.
	
	Sin embargo, admite que esta es una reproducción de otras
	reproducciones, inspiradas en la energía de otra dimensión.
	
	Si quieres discutir algo, establece un diálogo con esa otra dimensión,
	que es el verdadero Autor.
	

	\cleardoublepage
	~\vfill

		% \nohyphenation
		
		\textit{\textbf{Thanks}}

		\textit{A todos amigos y enemigos; producto de las inspiraciones.}
		
		\textit{Si molestamos a alguien por concretar estas inspiraciones, lo sentimos.}
	
	\newpage
		~\vfill
		\thispagestyle{empty}
		\setlength{\parindent}{0pt}
		\setlength{\parskip}{\baselineskip}
		
			\ccLogo \space \textsc{Creative Commons}
		
		\par\textsc{Published by The Nectar Project}
		
		\par John Thomas Wilke	\\
		Juan Tomás Bradanovic Pozo	\\
		nectartom@mail.com
		
		\par The Nectar Project, LLC\\
		P.O. Box 1455\\
		North Bend, OR 97459
		
		
		\par \textsc{www.nectarproject.org}
		
			
\mainmatter				
	
	\chapter*{}
	
	En la China Antigua, como en los Andes y en cualquier parte donde
	moraron los seres humanos, la gente vivía según una cosmovisión, no
	según las leyes del dinero. Ellos no permitieron que la consciencia de
	la cosmovisión fuera olvidada; esa era su fuente, era su constante y su
	Gran Certeza. La cosmovisión no es una cosmología, pues eso es un
	concepto del cerebro. Ni es la cosmovisión como un sueño o imagen a
	partir de los órganos sensoriales. La cosmovisión no tiene los
	contenidos. Es una Calidad. Es la calidad suprema. No hay una palabra
	que pueda comunicar su significado. El ``Tao'' es la palabra de los
	sabios de la china antigua para este todo-lo-que-tiene-sentido.
	
	Nosotros usamos a menudo la palabra ``Tao'' pero no somos ``taoistas''. El
	``tao'' como un concepto que puede ser usado como filosofía, creencia,
	religión, ritual o práctica, un juego de ética o moral, una base para
	las leyes, o los fundamentos de una sociedad, no es el Tao Viviente.
	
	Aunque el Tao Viviente pudo ser la experiencia de muchos seres humanos
	en China durante la era dorada (quinto al segundo siglo A.C.), esa
	experiencia auténtica casi desapareció con el taoísmo que le siguió.
	Tao-ísmo, como el Zen moderno, es una mera amalgama popularizada y
	degenerada, de cantos, oraciones, creencias, supersticiones, alquimia,
	magia, rituales y cultura de la salud física -- una especie de brujería.
	Todo esto puede estar un universo más cerca de lo Real, la Verdad, lo
	natural es que la tecnología y las religiones de las sociedades
	occidentales modernas, llenen nuestras mentes y formen nuestra
	perspectiva en el siglo 21.
	
	Nuestra mentalidad y nuestras sociedades son inimaginablemente
	diferentes de lo que fue China en 400 A.C. Nuestras mentes son incapaces
	de la tranquilidad y el vacío, del cual se cultiva una conciencia de lo
	más sutil. Nuestra facultad de conocimiento ha sido apartada de su
	naturaleza original como una conciencia real; por eso necesitamos una
	palabra ``Dios''. ¿Por qué es necesario una palabra? Para hablar, para
	escribir, para leer, para enseñar, para aprender --- para tratar
	encontrar y relacionar a otra gente que también necesita este concepto.
	Y eso es necesario solo para quien está fuera de su propio ser.
	
	Por contraste, la gente antigua no necesitó un concepto ``Dios'' ni
	tampoco de una palabra. Si un ser humano siente constantemente una
	Presencia sagrada, entonces supone que toda gente lo siente eso también.
	Y no solamente supone, pues uno que lleva la Presencia sagrada lo sabe y
	lo siente. ¿Para qué sirve una palabra? ¿Un pez necesita la palabra
	``Agua'' para nadar y respirar? No es necesario, ninguna mera creencia,
	ninguna afirmación de la santa fe. La guía clara interior de todas
	nuestras acciones y palabras, desinteresada, espontánea, de recta
	sabiduría, es en una perfección continua de toda la gente. La palabra,
	el pensamiento, solo diluye todo eso.
	
	Lao Tse (siglo sexto A.C.) y Chuang Tse (siglo cuarto A.C.) son los dos
	místicos chinos que permanecen como los más fieles seguidores del Tao
	verdadero. Generalmente hablando, todos los demás escritores --- incluso
	Confucio --- están contaminados por cierto tipo de interés o prejuicio.
	Los dos escritos que ofrecemos aquí, Tao Te King de Lao Tse y la
	colección titulada El Camino de Chuang Tse, también están probablemente
	contaminados en cierto grado, sin embargo, ellos presentarán muy bien la
	forma en que la mente y espíritu de estos dos seres puros expresaba su
	cosmovisión. Lao Tse, se dice, fue un bibliotecario humilde y un recluso
	silencioso. Aparentemente, él nunca escribió nada de su sabiduría. Luego
	cuando estaba listo para morir, no quiso morir entre la gente; iba
	saliendo por las puertas de la ciudad y fue reconocido por el portero,
	quien le pidió que le dictara lo más importante de sus enseñanzas. El
	resultado fue el que nosotros conocemos hoy como el Tao Te King, el cual
	ha sido traducido más veces que cualquier otro libro en la historia con
	excepción de la Biblia, y ciertamente no es enteramente de Lao Tse.
	Chuang Tse también era un ermitaño, viviendo el Tao como está reflejado
	en Lao Tse. Chuang Tse vivió cuando mucha gente en la China estaba
	alerta a la más profunda realidad, fue bien conocido y hasta
	reverenciado.
	
	Tampoco estos seres estaban interesados en la aceptación ajena o influir
	en los demás. Sus enseñanzas estaban dirigidas a personas sinceras que
	ya se encontraban viviendo el Tao, y necesitaban una corrección ligera,
	una afirmación y puesta a punto. En tiempos modernos, el mensaje del Tao
	viviente cae en oídos sordos y en almas muertas, por eso muchas veces ha
	sido deformado --- reducido a lo que es útil para los humanos que se
	preocupan por vivir en la sociedad. Tampoco Lao Tse ni Chuang Tse
	estaban interesados en el mundo hecho por humanos; nos remiten a ``un
	Lugar sin-lugar'' antes, más allá y dentro el mundo hecho por humanos.
	Este Lugar es tan profundo dentro de nosotros que para experimentarlo
	uno debe soltar sus ataduras con todo lo demás. Solamente aquellos que
	han soltado sus vínculos en todo lo demás están en una posición de saber
	siguiera que ese lugar existe. Y ``vivir el Tao'' significa liberación
	consistentemente --- de un momento a otro.
	
	¿Quién haría esto? Solamente el que ve que el mundo no ofrece nada, que
	son meros espejismos, podría estar dispuesto a perderlo. Pero, solamente
	el que pierde su vida puede encontrar su Vida. Esto es el mensaje
	verdadero de Tao, el cual Jesús más tarde llevó al occidente. Y también,
	esto era el mensaje original de Zen; nada más que esto. ¡En el siglo 20,
	fue el mensaje verdadero de Osho, quien le dijo a sus seguidores que los
	estaba acercando solo para poder golpearlos con su almádena! Pero la
	mayoría de sus seguidores escucharon sólo lo que sus egos quieren
	escuchar. También sus críticos: escucharon sólo lo que sus egos quieren
	escuchar. Y este mismo mensaje, la cosmovisión del Tao, es que que
	Ramana Maharshi y ``Sri Nadie'' (Albert Sorensen quien era conocido por
	Ramana como ``Sunyata''). Ambos murieron en los 1980s. En las palabras
	de su amigo A. U. Vasavada: ``No hay nada que hacer sino estar
	constantemente alerta y consciente, continuar en lo desconocido y en
	ninguna parte''.
	
	Lao Tse, Chuang Tse, y estos otros seres puros siguen viviendo en cada
	ser humano. Pero nosotros estamos apartados de nuestra esencia pura;
	existimos en la insignificancia de nuestra periferia, somos demasiado
	perezosos, intoxicados, distraídos, egoístas para curarnos, para dejar
	todo lo que hacemos y sólo SER. Para todos ustedes que están preparados
	para soltar las amarras, para todos ustedes que no tienen miedo a la
	oscuridad, al fracaso, a la soledad, silencio y humildad, les ofrecemos
	los siguientes escritos.
	
	\chapter*{¿Qué no es Tao?}


\textit{(Atribuido al Monje Shao, discípulo de Kumarajiva)}

\vspace{+1\baselineskip}
	
	Todas las cosas están acompañadas, pero el Tao permanece solitario.\\
	Fuera del Tao no hay nada; en él no hay separación.\\
	Sin exterior o interior, incluye el Único Primordial\\
	Y abarca los ocho reinos y las diez mil cosas.
	
	No es uno, ni muchos; no es oscuro, no es luminoso.\\
	Sin procedencia, sin final; no es vacío, no es sustancia.\\
	No va hacia arriba, no va hacia abajo; no construye, no destruye.\\
	Sin moverse, sin descansar; sin ir, sin venir.
	
	No es profundo, no es superficial; no es sabio, no es ignorante.\\
	Sin contradecir, sin armonía; no es nuevo, no es viejo.\\
	No es bueno, no es malo; no está solo, sin par.
	
	¿Pero porque esto es así?\\
	Porque si ves que tiene interior, abarca todo el universo;\\
	Si ves que tiene exterior, tiene capacidad y establece todas las
	cosas.\\
	Si ves que es pequeño, cubre ancho y lejos;\\
	Si ves que es grande, penetra en el reino de los átomos.
	
	Llámenlo el uno, y después comprende todas las cualidades;\\
	Llámenlo los muchos, y después su cuerpo es vacío.\\
	Llámenlo luminoso, y después es oscuro y turbio.\\
	Llámenlo oscuro, y después es intensamente brillante.\\
	Digan que se presenta, y no tiene cuerpo ni forma;\\
	Digan que se extingue, y brilla por toda la eternidad.
	
	Llámenlo vacío, tiene miles de funciones;\\
	Digan que existe, es silencioso sin forma.\\
	Llámenlo alto, es llano sin extensión;\\
	Llámenlo bajo, nada es igual.
	
	Proclaman que construye, y se dispersa por todas las estrellas;\\
	Proclaman que destruye, y las cosas duran por largo tiempo.
	
	Digan que se mueve, y permanece en silencio;\\
	Digan que se queda quieto, y funciona con todas las cosas.\\
	Pretendan que regrese, se va sin decir adiós;\\
	Pretendan que se marche, y será el momento de su regreso.\\
	Llama a lo profundo, se mezcla con todos los seres;\\
	Llama a lo superficial, sus raíces no pueden ser vistas.
	
	Llámalo pobre, y tiene miles ventajas;\\
	Llámalo rico, y nada existe en ello.
	
	Digan que está solo, y empieza a asociarse con todas las cosas;\\
	Digan que se conecta, y se cae vacío y solo.
	
	\chapter*{Prólogo}
	
	Todas las religiones son religiones porque están en sí mismas limitadas.
	``Una religión'' es definida por una doctrina y los rituales
	específicos. Esta doctrina y ritual son lo que da a la religión un
	sentido de la solidez, de certeza. Y este sentido de certeza es el sine
	qua non de la fe. Entonces, para obtener la fe del pueblo, las elites
	deben auto limitar su sistema. Pero esto va contra la esencia de lo que
	una religión pretende hacer: acercarse a lo Ilimitado.
	
	Lo que Lao Tse ofrece no es una religión. Lo que ofrece es fluidez.
	Enseña el camino de fluir con la vida -- la senda directa al Tao. Se
	llama ``La senda sin una senda''. No ordena mandamientos ni
	prohibiciones; no amenaza con el castigo divino ni seduce con la
	recompensa divina. No hay ``bueno'' ni ``malo''. Por ello, no hay una
	moral que limite la senda del que busca, por el contrario, Lao Tse anima
	a cada persona, individualmente, a seguir adelante en este momento,
	probando todo lo que nos atrae en el mundo exterior; el buscar sólo nos
	lleva en una dirección, a un callejón sin salida. Así puede descubrir
	por sí mismo que el Tao es la única vida digna de vivir. Los hombres y
	el mundo exterior son vistos sólo, como fantasmas. No importa que
	hagamos porque no tenemos el poder de nada. Con el tiempo, todos
	retornaremos al Tao, el Ser verdadero. ¿Cuándo? Esa es tu elección.
	
	Lao Tse tampoco ofrece una práctica de meditación, sólo el silencio de
	la mente. Gia-fu afirma: ``La meditación en el taoísmo es cuando tocas
	lo más bajo. Se llama la gran Quietud, o a veces lo traduzco como `la
	gran Certeza'. Todo el cuerpo se calma en la quietud. Alcanzas el fondo
	como una roca en el mar\ldots{} En verdad, eres una roca, porque no
	debes usar la mente. No hay interferencias de la mente''. (desde la
	entrevista `Vagando por el Camino', originalmente publicado en la
	revista \textit{The Dragon's Mouth} y disponible en el sitio \textit{Abode
		of the Eternal Tao (https://abodetao.com)}.
	
	Lao Tse tampoco ofrece la imagen de un santo a emular. Un devoto budista
	puede vivir como él cree que vivió el Buda, un cristiano puede tratar de
	ser ``como Cristo'' según alguna interpretación de su vida. Pero Lao Tse
	y ``el Sabio'' son tan resbaladizos como el mismo Tao. Son los
	auténticos originales: tú también debes ser un auténtico original.
	
	Así como han existido muchos traducciones e interpretaciones del Tao Te
	King, han existido muchas versiones de `tao-ismos'. Las enseñanzas de
	Lao Tse no pueden ser comprendidas en una sola. En verdad, ningún
	``ismo'' puede ser compatible con el Tao Te Ching. No hay rituales,
	sacerdotes, templos, congregaciones, ni dioses mencionados en ninguna
	parte. ¡Tampoco creencias! Todo eso son fantasmas y equipaje en exceso.
	Lao Tse aconseja que dejemos todo y que retornemos al estado Primigenio,
	tal como antes de nacer. Este estado es antes del yin y yang, que son
	las raíces primitivas del mundo; antes del miedo, del deseo, y de los
	pensamientos del cerebro. Es la Nada: la oscuridad, la vacuidad, el
	caos, la ausencia de una mente consciente de los objetos. ¿Qué es, lo
	que queda entonces? En las palabras de Gia-fu: ``Un estado de mente
	no-juicioso, un estado de `Es Así'. Experimentas que nada está
	determinado, todo es un flujo, y puedes mantener tu conciencia sin
	rechazar o abrazar... Si estás encolerizado, estar encolerizado. Si te
	sientes como mierda, ser como mierda. Si estás deprimido, estar
	deprimido. En verdad, ves la Luz pura que va más allá de abarcar o
	rechazar. El mantra taoísta es: ``nada falta, nada en exceso''. Es sólo
	porque abarcamos y rechazamos que desconocemos la `Tal Es' de las
	cosas''. (Ibid.)
	
	Hay varias traducciones del título de este texto. La palabra ``King''
	significa ``el clásico'', o ``el libro grande''. La palabra ``Te''
	significa ``la virtud'', ``la integridad'', ``la fortaleza interior'', o
	incluso ``la fuerza divina'', ``la fuerza sanadora''. La palabra ``Tao''
	es imposible de traducir. Por favor ver nuestra explicación en la
	sección Notas al fin del texto.)
	
	La información general sobre este texto, que tiene cerca de 2500 años, y
	sobre su autor Lao Tse, está disponible en muchas otras referencias, por
	ello no estará repetido en este texto. La historia cuenta que Confucio,
	después su sesión con Lao Tse, dijo a sus discípulos: ``Al animal que
	corre por la tierra se le coge con una trampa, al pez que nada en las
	aguas se le pesca con una red, al pájaro que vuela por los cielos se le
	caza con una flecha, pero al Dragón que se remonta por encima de las
	nubes, yo no sé cómo atraparlo. Yo he visto a Lao Tse; él es como el
	Dragón''. (De Colodron, A. Tao Te King de la versión de John Wu, Arca de
	Sabiduria, Editorial EDAT, 1993). ¿Porque un Dragón? Lao Tse no es
	interesado en tu salud, tu seguridad, tus relaciones con las demás, tus
	prendas; él a ti señala el caos, la confusión, la contradicción de tu
	vida, la esterilidad de todos tus esfuerzos, y que de un momento a otro
	algo puede cambiar en su contrario. Este Dragón no está interesado en
	presentarte ``una visión nueva del mundo''; él hace astillas tu
	realidad. No está interesado en ofrecerte una cosa en que pensar; él te
	agarra de los órganos sexuales y cierra de golpe tu cerebro mecánico. No
	está interesado en si te gustan sus palabras o no; él abre tu corazón
	sentimental y lo alimenta de crudeza.
	
	Están aquellos que miran sus propias vidas y después, creen que un ser
	humano como Lao Tse no pudo existir, sus egos necesitan creer eso. Par
	ellos, él será siempre ``sólo una figura legendaria''.
	
	De todas las varias traducciones del chino del Tao Te King de Lao Tse,
	la de Gia-fu Feng y Jane English (Vintage Press, 1972, ISBN
	0-394-71833-X) es la más popular, habiendo vendido más de un millón de
	ejemplares. Gia-fu fue un erudito, y en algún momento reconoció la falta
	de armonía en su vida. Entonces cambió su vida a un estado más salvaje,
	erótico, emotivo, e ilógico --- y a usar el lenguaje simple, franco, y
	directo. Su Centro de Retiro, Stillpoint (El Punto Quieto), cerca
	Wetmore Colorado, no siempre era un lugar tan quieto. Había una bañera
	en el gran salón de reuniones donde se le se podía ver bañando a las 4
	de la madrugada; el inodoro quedaba justo a la entrada del edificio.
	Gia-fu podía ver toda la comunidad desde su ventana en la noche, y a
	veces expuso a algún seguidor o seguidora, por propagar enfermedades de
	transmisión sexual. Los miembros de la comunidad eran animados a
	expresar las emociones abiertamente en lugar de reprimirlas, y yo mismo
	una vez observé un intercambio de bofetadas entre un hombre y una mujer,
	antes la sesión de la mañana enfrente de toda la asamblea, la que no fue
	interrumpida. Gia-fu en sí mismo nunca pretendió estar ``más allá el
	sexo'', se casó con dos de sus seguidoras. A veces bebía en abundancia,
	y su hábito de fumador sin duda contribuyó al cáncer pulmonar del que
	murió a la edad de sesenta y seis años. No representaba exactamente la
	vida de ``el Sabio''; sin embargo, a su caótica manera, ¡calzaba
	perfectamente con los principios del Tao!
	
	El estilo primitivo no es lo Primigenio a que se refiere el Tao Te King,
	pero eso está en el camino. Mucha gente que se declara ``espiritual'',
	``sin egoísmo'' o ``más allá del sexo'' confunden sus ideas sobre lo que
	es el estado verdaderamente espiritual, sin egoísmo, y más allá el sexo.
	Quizás niegan quienes son, y están confundidos en lo que creen o
	intentar ser. Gia-fu fue un maestro para detectar la hipocresía, y
	rutinariamente señalaba a cualquier pretencioso de su entorno. El que
	vive falsamente puede muchas veces quedar reducido a sus instintos. La
	charla beata es inútil. Asimismo, su Tao Te King es real, animado y nada
	pretencioso. Algunos traductores de Lao Tse, ``profesores
	espirituales'', endulzan el medicamento cuando es amargo. ¡Gia-fu no fue
	así! La gente sincera que busca la verdad, y que no teme la verdad,
	armonizará con él. Si buscas llegar, y no sólo seguir esta o esa senda
	toda tu vida, éste puede ser el único libro que necesitas.
	
	Cuando leemos algo que nos conmueve, usualmente sentimos un fuerte
	impulso de continuar leyendo más rápido. Hacer esto es un error. Las
	palabras se amontonan sobre palabras en la conciencia, pues intentamos
	mantener una emoción. Leemos más y más con la esperanza que el autor nos
	tomará de la mano y averiguará la salida de las capas de nuestra propia
	basura, y nos guiará a través de las catacumbas oscuras a la Luz. Ningún
	autor puede hacer eso. Nuestro intento de tomar el camino fácil nos deja
	con las manos vacías. Es una estrategia que diluye la intensidad de los
	símbolos que terminan mezclados en la sopa de nuestra memoria. El
	problema es bastante serio en lenguaje chino, pero es mucho peor en las
	lenguas modernas occidentales, que son tan racionales y lógicas que la
	Vida Real dentro hace difícil para el lector sentir los significados.
	
	Con el Tao Te King , es mejor leer unas pocas palabras, cerrar el libro,
	cerrar los ojos, y pasivamente hundirse en el espacio de No-Pensar.
	Entre las palabras y las líneas, te relajas en No-Actuar, No-Hablar,
	No-Juzgar. ¡Eso exige su tiempo! También, exige el silencio de un medio
	ambiente propicio. Algo en tu ser más profundo que la realidad de
	consenso, ha estada cerrado desde hace mucho tiempo. Para abrirlo no es
	exactamente necesario un esfuerzo, sólo necesitas el coraje de mirar más
	alto de lo que has mirado habitualmente. Mira tu silencio, tu vacuidad,
	tu Naturaleza Primigenia, la paradoja de quien eres, una manera de ser
	nueva, aunque muy antigua --- es decir, lo que somos verdaderamente. Así
	quedas, desnudo y solitario. Para sólo caer en esta enseñanza profunda.
	Desaparecer en él. Morir en él. Selecciona cualquier verso para tu
	meditación, y te conducirá al cambio que es la esencia de Lao Tse. Esta
	esencia no es una doctrina, es un sabor.
	
	Por eso, vas lentamente. No es una novela. Te das el tiempo para abrirte
	a todos los niveles de sentido, en un paisaje que armonice con tu ser.
	Encontrarás tú mismo la unidad con ``el Tao que trabaja silenciosamente,
	sin palabras -- que en verdad controla el universo entero''. (Gia-fu
	Feng, ibid.) La explosión hacia la iluminación sucederá.
	
	Esta traducción al español, y la anotación, son de los editores John
	Thomas Wilke y Tomás Bradanovic Pozo. Mientras hicimos la obra,
	consultábamos la excelente traducción de John Wu, y a veces utilizamos
	su palabra en lugar de nuestra. Siempre, la intención fue recordar lo
	que yo he aprendido personalmente de Gia-fu y sus amigos en Stillpoint,
	y he tratado de captarlo en palabras.
	
	El primer capítulo contiene la esencia del Tao Te King entero. Es
	profundo al final. Si puedes sentir los significados dentro de estas
	palabras podrás ver que el resto de los capítulos son un largo
	comentario del primero. Se recomienda al lector volver con frecuencia a
	este primer capítulo a medida que avanza en su lectura y verá como
	aparecen sentidos nuevos y más profundos.
	
	Mira. Porque si estás listo, tu Maestro ha aparecido.
	
	\chapter*{Uno}
	
	El tao\footnote{``El Tao'': ¡El Todo y la Nada! No hay una palabra equivalente en
		español o en inglés. En Aymara, quizá ``tata'', en Quechua ``aventaza''
		son las más cercanas.} que se puede expresar en palabras\footnote{La traducción original por Gia-fu y Jane fue:
		
		``El tao que se puede señalar no es el Tao eterno.\\
		El nombre que se puede nombrar no es el nombre eterno.\\
		Aquel sin nombre es el principio de cielo y tierra.\\
		Aquel que es nombrado es la madre de diez mil objetos.\\
		Siempre sin deseo, uno puede ver el misterio.\\
		Siempre deseando, uno puede ver las manifestaciones.\\
		Esos dos brotan de la misma fuente, pero son de distinto nombre;\\
		Por eso aparece la oscuridad.\\
		Oscuridad dentro de oscuridad;\\
		La entrada a todo misterio''.}\\
	No es el Tao eterno.\\
	La presencia que se puede nombrar\\
	No es la Presencia perdurable.
	
	Lo Indistinguible es el principio,\\
	Y genera lo de arriba-y-abajo.\footnote{A lo largo de este texto, arriba-y-abajo se refiere a la interacción dinámica entre las dos polaridades principales del yang y yin, que son inseparables. John Wu usa `cielo-y-tierra' con los guiones. Gia-fu no usa los guiones.}\\
	Anhelar distinguir es dar nacimiento\\
	A los diez mil objetos separados\footnote{``Los diez mil objetos'': Esta expresión ilustra la experiencia por la
		percepción ordinaria de un mundo de objetos separados. Cuando cambiamos
		su conciencia del verdadero Ser a lo objetivo, eso es solo el mundo de
		lo que está disponible. Comienza ``el juego-dual'' que son meros
		fantasmas de las sensaciones del cuerpo.}.
	
	Siempre sin deseo\\
	Uno siempre mora en el Fulgor.\\
	Siempre deseando\\
	Uno puede solamente mirar sus manifestaciones.
	
	Esos dos brotan de la misma fuente\\
	Pero parecen se ser dos realidades separadas;\\
	Entonces se surge el vagando en la oscuridad.
	
	Entra en el Vasto Vacío, en el centro de la oscuridad:\\
	La única salida de la oscuridad.
	
	\chapter*{Dos}
	
	Bajo el cielo se puede ver la belleza como tal sólo porque existe la
	fealdad.\\
	Se puede ver el bien como tal sólo porque existe la maldad.
	
	Por lo tanto, lo que se comprende y lo que no se comprende aparecen
	juntos.\\
	Lo difícil y lo fácil se complementan el uno al otro.\\
	Largo y corto se contrastan entre sí.\\
	Alto y bajo descansan el uno en el otro.\\
	La voz y el sonido armonizan entre sí.\\
	El comienzo y el fin se suceden, uno tras otro.
	
	Por lo tanto, el Sabio vive una vida de ``No-Actuar''\footnote{A lo largo de este texto, el valor del ``No-Actuar,'' o ``No-Hacer'' se
		enseña de varios modos. No se refiere a un esfuerzo por refrenarse de
		las actividades y estar quieto. Es la manera de vivir la vida es
		automática para el Sabio. Es simplemente su inclinación. Los movimientos
		ocurren, lo que se habla ocurre, pero sin tener el estado de mente de
		``Yo actúo'' o ``Yo hablo''.}, enseñando a
	``No-Hablar''.\\
	Entretanto los diez mil objetos suben y bajan sin cesar.\\
	Creando, aunque no poseyendo.\\
	Trabajando, aunque sin reclamar el mérito.\\
	El trabajo se finaliza y luego se olvida.\\
	Y por ello dura para siempre.
	
	\chapter*{Tres}
	
	No exaltar al talento evita las discusiones;\\
	No acumular tesoros evita el hurto.\\
	No ver las cosas como deseables evita la confusión del corazón.\\
	El Sabio, por lo tanto, logra el control liberando los corazones del
	deseo\\
	y alimentando en abundancia,\\
	Debilitando las ambiciones y fortaleciendo la energía\\
	desde la médula de los huesos.
	
	Si los hombres carecen de los conocimientos y deseos,\\
	entonces los astutos no tratarán de interferir.\\
	Si nada se hace, todo irá bien.
	
	\chapter*{Cuatro}
	
	El Tao es una vasija vacía; es usada, pero jamás se llena.\\
	¡Oh, insondable origen de diez mil objetos!\\
	Gastado el filo.\\
	Deshecho el nudo.\\
	Suavizado el fulgor.\\
	Volviendo al polvo.
	
	¡Oh, tan escondido, pero siempre-presente!\\
	Yo no sé de dónde viene.\\
	Es el antepasado de los dioses.
	
	\chapter*{Cinco}
	
	Cielo-y-tierra no negocia;	\footnote{Son dos traducciones distintas por Gia-fu y Jane de las cuatro primeras
		líneas. Su traducción original fue:
		
		``Cielo y tierra son crueles;\\
		Ellos ven a los diez mil objetos como muñecas.\\
		Los sabios también son crueles\\
		Ellos ven a los hombres como muñecos''.
		
		Y su traducción más tarde fue:
		
		``Cielo y tierra son imparciales;\\
		Ellos ven a los diez mil objetos como perros de paja.\\
		Los sabios también son imparciales;\\
		Ellos ven a los humanos como perros de paja''.
	
	El mensaje de estos versos es muy fuerte. La palabra ``muñeco'' lleva a
	la intuición de ``artificial,'' ``fingido,'' ``sin mente,'' y
	``muerto''. Para un místico tal como era Lao Tse no importa si la verdad
	insulta los egos o no. ¡Él nos dice exactamente lo que nosotros somos!
	Los que lo pueden escuchar aprenden. Lao Tse no se preocupa por los que
	no lo pueden escuchar; son solamente ``muñecos''.
	
	``Perros de paja'' se refiere a los perros falsos rellenos con paja,
	usados para los desfiles en las fiestas en China antigua, para
	representar la falta de sentido de la existencia física. Por eso, estos
	versos nos dicen que `los dioses' no favorecen a los que tratan de
	hacerse pasar como algo aparentemente real.	
}\\
	Él ve los diez mil objetos como falsos y sin sentido.\\
	Los Sabios tampoco negocian;\\
	Ellos ven a los hombres como falsos y sin sentido.
	
	El espacio entre el cielo y la tierra es como el interior de un
	fuelle:\\
	Cambia su apariencia, pero no su condición.\\
	Mientras más se mueve, más se debe ceder.
	
	Más palabras importan menos.\\
	Mantenerte firme en el Centro.
	
	\chapter*{Seis}
	
	El espíritu del valle nunca muere;\\
	Es la hembra misteriosa, yin, la Madre Primigenia\footnote{Lo masculino primal (el yang) es ``el lado soleado de la montaña'', y lo
		femenino primal (la yin) es ``el lado sombrado de la montaña''. Esos
		opuestos complementarios son el fundamento de todos los fenómenos, no
		sólo en la cosmovisión china antigua, sino también en la gente antigua,
		andina, de Sudamérica. Un estudio y reflexión de toda la vida sobre esta
		dualidad es un uso muy meritorio del tiempo. Sin embargo, evitaremos
		hacerlo aquí.}.\\
	Ella tiene la llave a vida debajo cielo.\\
	Es como un velo que apenas se lleva a cabo\footnote{Una segunda interpretación del símbolo de un velo es que una vez se tira, es difícil reconocer que obstruía la visión.}.\\
	Úsalo, nunca fallará.
	
	\chapter*{Siete}
	
	Cielo-y-Tierra permanece por siempre.\\
	¿Por qué Cielo-y-Tierra permanece por siempre?\\
	¡Porque nunca ha nacido!\\
	Y por ello vive eternamente.
	
	Por eso el Sabio permaneciendo detrás, llega primero.\\
	Él es desprendido, entonces está en unidad con todos.\\
	A través de la acción, sin conciencia de ``Yo actúo'', él realiza su
	Ser.
	
	\chapter*{Ocho}
	
	El bien más alto es como el agua.\\
	El agua da vida a diez mil objetos sin esfuerzo.\\
	Fluye por lugares que el hombre rechaza, de la misma manera que el Tao.
	
	Al vivir, mantente cercano a la tierra.\\
	Al meditar, profundiza dentro de tu corazón.\\
	Al tratar a los demás, sé gentil y bondadoso.\\
	Al hablar, sé veraz.\\
	Al mandar, sé justo.\\
	En la vida diaria, sé concienzudo.\\
	En la acción, está en el momento presente.
	
	Sin pelear, estarás libre de reproche.
	
	\chapter*{Nueve}
	
	Mejor detenerse un momento, que llegar hasta el borde.\\
	Afila demasiado la espada y el filo pronto se arruinará.\\
	Almacena una bodega de oro y jade, nadie será capaz de protegerlo.\\
	Exige riqueza y honores, luego vendrá el desastre.
	
	Retírate cuando el trabajo esté completo.\\
	Ese es el camino del cielo.
	
	\chapter*{Diez}
	
	Llevando cuerpo y alma, pero abrazando el Uno.\\
	¿Puedes evitar la separación?\\
	Completamente concentrado, pero flexible\\
	¿Puedes ser como un recién nacido?\\
	Purificando y aclarando la Visión Primigenia\footnote{``Primigenia'' se refiere a la condición antes la sobreposición de un
		ego en un niño, y la condición de algún humano en todos momentos en que
		es libre de su ego. Esta condición no se ha perdida, pero se obstruye
		cuando la mente enfoca un mundo. Es una conciencia más allá, o sin, la
		experiencia de un cuerpo, y antes la impureza del instinto de
		supervivencia y ``el principio del placer'' de Freud. Esa es la
		condición del sabio.},\\
	¿Puedes mantenerte impecable?\\
	Amando a todos los hombres y gobernando el país,\\
	¿Puedes hacerlo sin habilidad?\\
	Abriendo y cerrando las puertas del cielo,\\
	¿Puedes hacer el papel femenino?\\
	Con entendimiento y receptivo a todas las cosas,\\
	¿Eres capaz de No-Hacer?
	
	Dando la vida y alimentando,\\
	Cargando, pero no poseyendo,\\
	Trabajando sin atribuirse el mérito,\\
	Liderando, aunque sin dominar:\\
	Esta es la Virtud Primigenia.
	
	\chapter*{Once}
	
	Treinta rayos forman el eje de la rueda;\\
	Sin valor si no tiene el agujero que está al centro.\\
	Forma un vaso con arcilla;\\
	Sin valor si no tiene el espacio interior.\\
	Corta puertas y ventanas en una habitación;\\
	Son los espacios abiertos quienes la hacen accesible.\\
	Por lo tanto, los beneficios vienen de lo que está.\\
	La utilidad viene de lo que no está.
	
	\chapter*{Doce}
	
	Los cinco colores ciegan el Ojo.\\
	Los cinco tonos ensordecen el Oído.\\
	Los cinco sabores embotan el Gusto.\\
	Competencias y cacerías enloquecen la mente.\\
	Las riquezas conducen al extravío.
	
	Por lo tanto, el Sabio es guiado por lo que siente y no por lo que
	imagina.\\
	El deja ir el exterior, y escoge el interior.
	
	\chapter*{Trece}
	
	Acepta la humillación de buena gana.\\
	Acepta la mala fortuna como parte de la condición humana.
	
	¿Qué significa ``Acepta la humillación de buena gana''?\\
	Aceptar ser intrascendente.\\
	No preocuparse por perder o ganar.\\
	Esto es llamado ``Aceptar la humillación de buena gana''.
	
	¿Qué significa ``Acepta la mala fortuna como parte de la condición
	humana''?\\
	La mala fortuna es obligatoria si quieres tener un cuerpo.\\
	Si no preocupaste por un cuerpo, ¿cómo podría existir la mala fortuna?
	
	Ríndete humildemente;\\
	Entonces serás confiable para cuidar de todo.\\
	Ama al mundo como a tu propio ser;\\
	Entonces podrás realmente cuidar de todas las cosas.
	
	\chapter*{Catorce}
	
	Mira, no puede ser visto -- está más allá de la forma.\\
	Escucha, no puede ser oído -- está más allá del sonido.\\
	Tómalo, no puede ser sostenido -- es intangible.
	
	Esos tres son indefinibles;\\
	De esta manera, se juntan en Uno.
	
	Desde arriba no se ve iluminado;\\
	Desde abajo no se ve oscuro;\\
	Un hilo ininterrumpido más allá de toda descripción.
	
	Retorna a la experiencia de la Nada:\\
	La forma de lo amorfo,\\
	La imagen de lo invisible.\\
	Es llamado indefinible y más allá de la imaginación.
	
	Párate delante de Él, no hay comienzo,\\
	Después de Él, no hay fin.\\
	Permanece con el antiguo Tao,\\
	Fluye con el Tao que está siempre-presente.
	
	Disfrutar el Principio Primigenio.\\
	Eso es el néctar del Tao.
	
	\chapter*{Quince}
	
	Los antiguos maestros eran sutiles, misteriosos, profundos,
	responsivos.\\
	La profundidad de su conocimiento es insondable.\\
	Por ser insondable,\\
	Sólo podemos describir su apariencia:
	
	Cuidadosos, como los hombres que cruzan un arroyo invernal.\\
	Alertas, como los hombres conscientes del peligro.\\
	Corteses, como cuando hay invitados.\\
	Transitorios, como el hielo a punto de derretir.\\
	Simples, como bloques de madera sin tallar.\\
	Vacíos, como las cavernas.\footnote{Gia-fu comenta sobre este pasaje: ``Nuevamente, es el sentido de algún
		tipo de accidente divino. Estás muy receptivo y quieto hasta el momento
		en que eres movido a actuar. Luego, vuelves a la vacuidad, receptividad
		total, para escuchar completamente a tu propio centro, sin ideas
		preconcebidas. Así, creo que la vacuidad es la esencia del taoísmo.
		Muchos capítulos comienzan por ``Estar vacío''. (Ibid.)}\\
	Opacos, como los charcos de fango.
	
	¿Quién puede esperar quieto mientras el fango se asienta?\\
	¿Quién puede esperar inmóvil hasta el momento de la acción?\\
	Los observantes del Tao no buscan el tener éxito;\\
	Al no buscar el tener éxito no son sacudidos por el deseo del cambio.
	
	\chapter*{Dieciséis}
	
	Vacíate de todo.\\
	Deja que la mente descanse en silencio.\\
	Los diez mil objetos suben y caen mientras el Ser contempla el
	Retorno.\\
	Ellos crecen y prosperan, luego retornan al Principio.\\
	Retornar al Principio es la Quietud perfecta,\\
	lo que es el camino de la naturaleza.\\
	El camino de la naturaleza es inmutable.
	
	Conocer lo que dura es el conocimiento más profundo;\\
	No conocer lo que dura lleva al desastre.\\
	Conociendo esta Gran Certeza, la mente está abierta.\\
	Una mente abierta abrirá el corazón.\\
	Con el corazón abierto, actuarás noblemente.\\
	Siendo noble, alcanzarás la Divinidad.\\
	Siendo divino, serás Uno con el Tao.\\
	Ser Uno con el Tao es sempiterno.
	
	Y aunque el cuerpo muera, hay algo que quedase.\\
	Su nombre es Tao.
	
	\chapter*{Diecisiete}
	
	Lo Altísimo rara vez es conocido por los hombres.\\
	Luego están los que la gente conoce y ama.\\
	Luego los que son temidos.\\
	Luego los que son despreciados.
	
	El que no confía en lo Altísimo no será digno de confianza con él.
	
	Cuando las acciones del Sabio son ejecutadas\\
	sin palabras innecesarias,\footnote{La traducción por Colodrón del Tao Te King de John Wu incluye esta
		línea: ``El Sabio pasa desapercibido y ahorra las palabras''. (Arca de
		Sabiduría, Editorial EDAT, 1993).}\\
	La gente dice pretenciosamente ``¡lo hicimos!''
	
	\chapter*{Dieciocho}
	
	Cuando el gran Tao es olvidado,\\
	Aparecen la bondad y la moralidad;\\
	Cuando surgen la inteligencia y la razón,\\
	Empieza el gran fingimiento.
	
	Cuando no hay paz dentro de la familia,\\
	Crecen ``el apego filial y la devoción''.\\
	Cuando el país está confuso y en caos,\\
	Aparecen ``los ministros serviles''.
	
	\chapter*{Diecinueve}
	
	Abandona la santidad, renuncia a la filosofía,\\
	Será cien veces mejor para todos.
	
	Abandona la bondad, renuncia a la moralidad,\\
	Y los hombres redescubrirán el amor y el respeto por sus mayores.
	
	Abandona el ingenio, renuncia a las ganancias,\\
	Y los bandidos y ladrones desaparecerán.
	
	Esas tres son sólo formas superficiales; no son suficientes en sí
	mismas.\\
	Es más importante ver la simplicidad,\\
	Para comprender nuestra verdadera naturaleza,\\
	Para abandonar el egoísmo,\\
	Y atemperar el deseo.
	
	\chapter*{Veinte}
	
	Abandona el pensar, \footnote{Gia-fu y Jane dicen: ``Abandona el aprender''. Después de comparar eso
		con las traducciones por Wu y Mitchell, nosotros preferimos ``Abandona
		el pensar'' de Mitchell.} y pon un fin a tus problemas.\\
	¿Existe alguna diferencia entre un sí y un no?\\
	¿Existe alguna diferencia entre el bien y el mal?\\
	¿Debo yo temer lo que otros temen? ¡Que disparate!
	
	Otra gente está contenta, gozando del festín donde sacrifican al buey.\\
	En primavera algunos van al parque, y salen a la terraza.\\
	Pero yo solitario vivo sin rumbo, sin saber dónde estoy.\\
	Como un recién nacido antes de aprender a sonreír,\\
	Estoy solo, sin un lugar donde ir.\\
	Otros tienen más de lo que necesitan, pero solitario yo no tengo nada.\\
	Yo soy un tonto. ¡Oh, sí, estoy confundido!\\
	Otros están claros y brillan\\
	Pero yo sólo soy oscuro y débil.\\
	Otros son ingeniosos e inteligentes\\
	Pero yo sólo soy lento y estúpido.\\
	Oh, voy a la deriva como las olas del mar ---\\
	Sin dirección, como el viento veleidoso.\\
	Todos los demás están ocupados,\\
	Pero yo sólo estoy deprimido y sin objetivo.
	
	Soy diferente.\\
	Soy nutrido por la Gran Madre.
	
	
	
	\chapter*{Veintiuno}
	
	No existe mayor virtud que seguir el Tao y sólo el Tao.\\
	El Tao es elusivo e intangible.\\
	Oh, es intangible y elusivo, aun así lleva dentro de sí todas las
	imágenes.\\
	Oh, es intangible y elusivo, aun así lleva dentro de sí todas las
	formas.\\
	Oh, es sombrío y oscuro, aun así es Esencia en sí mismo.\\
	Esta Esencia es muy real, y por eso confiaríamos en ella.\\
	Desde el comienzo de todo hasta hoy su néctar nunca ha sido olvidado.\\
	Entonces, estoy consciente de la creación.
	
	¿Cómo conozco los caminos de la creación?\\
	Porque están obvios\footnote{Para clarificar este punto, considera que: ``El camino el más perfecto
		es buscar, hallar, y experimentar, primero y antes que todo, el Tao en
		cada momento de su vida''.}.
	
	\chapter*{Veintidós}
	
	Ceder y vencer.\\
	Doblar y enderezarse.\\
	Vaciar y llenarse.\\
	Usar y renovarse.\\
	Tener poco y ganar.\\
	Tener mucho y estar confundido.
	
	Por lo tanto, los Sabios abarcan al Uno,\\
	Y son un ejemplo para todos.\\
	Sin necesidad de mostrarlo\\
	Su brillo se presenta.\\
	Sin justificarse a sí mismos\\
	Se distinguen.\\
	Sin jactarse\\
	Reciben reconocimiento.\\
	Sin alardear\\
	Nunca dudan.\\
	Ellos no se oponen a nadie\\
	Así es que nadie se opone a ellos.\\
	Por eso los ancianos dicen: ``Ceder y vencer''.
	
	¿Es ésta una frase vacía?\\
	Sé realmente unido\\
	Y todas las cosas vendrán a ti.
	
	\chapter*{Veintitrés}
	
	Hablar poco es natural.\\
	Una tormenta no dura toda la mañana.\\
	Un chaparrón no dura todo el día.\\
	¿Por qué ocurre así? ¡Cielo y tierra! ¡Yang y Yin!\\
	Si el cielo y la tierra no pueden hacer cosas eternas,\\
	¿Cómo podría hacerlo el hombre?
	
	El que sigue al Tao\\
	Es uno con el Tao.\\
	El que es virtuoso\\
	Experimenta la Virtud\footnote{La palabra ``Virtud'' equivale a ``Integridad''. Esto se refiere a la
		persona que es completa, en Uno con el Todo.}.\\
	El que extravía el camino\\
	Está perdido.
	
	Cuando eres uno con el Tao\\
	El Tao te acoge.\\
	Cuando eres uno con la Virtud\\
	La Virtud está siempre allí.\\
	Cuando eres uno con la pérdida\\
	Experimentas la pérdida con alegría.
	
	El que no confía en el Tao\\
	No será digno de confianza con él.
	
	\chapter*{Veinticuatro}
	
	Quien se para en punta de pies no mantiene su equilibrio.\\
	Quien anda a zancadas no puede mantener el paso.\\
	Quien hace un espectáculo no es iluminado.\\
	Quien se cree mejor que los demás no es respetado.\\
	Quien es presumido no consigue nada.\\
	Quien es jactancioso no permanece.
	
	De acuerdo a los seguidores del Tao: ``Hay alimento extra y equipaje
	innecesario''.\\
	Esos no traen felicidad.\\
	Por ello los seguidores del Tao los evitan.
	
	\chapter*{Veinticinco}
	
	Algo misterioso\ldots{}\\
	Que parece haber nacido antes del cielo y de la tierra.\\
	Silencioso en El Vacío,\\
	Permanece por si mismo y es inmutable.\\
	Siempre-presente pero siempre-cambiando.\\
	Tal vez es la madre de los diez mil objetos;\\
	Yo no sé su nombre.\\
	Supongo que puedo llamarlo el Tao;\\
	A falta de una mejor palabra, yo lo llamo Supremo.\\
	Siendo Supremo, fluye.\\
	Fluye muy lejos ---\\
	Y al llegar lejos retorna.
	
	Por lo tanto, El Tao es supremo,\\
	El cielo es grande,\\
	La tierra es grande,\\
	El maestro también es grande.\\
	Existen cuatro grandes poderes del universo\\
	Y el maestro es uno de ellos.
	
	El maestro sigue a la tierra.\\
	La tierra sigue al cielo.\\
	El cielo sigue al Tao.\\
	El Tao sigue lo que es natural.
	
	\chapter*{Veintiséis}
	
	Lo pesado es sustento de lo leve,\\
	La quietud domina a la inquietud.
	
	Por lo tanto, el Sabio, andando todo el día,\\
	No pierde de vista a su equipaje.\\
	Aun cuando haya cosas maravillosas para contemplar\\
	Permanece desapegado y en calma.
	
	¿Por qué el uno que ha excedido el mundo hacer el ridículo en público?\\
	Para llegar a la levedad hay que perder la raíz profunda.\\
	Agitarse es quedar atrapado en el mundo.
	
	\chapter*{Veintisiete}
	
	El mejor cazador no deja huellas.\\
	El mejor orador no comete deslices.\\
	El mejor cajero no necesita anotar.\\
	La mejor puerta no necesita una cerradura.\\
	A pesar de ello nadie puede abrirla.\\
	La mejor atadura no requiere nudos.\\
	A pesar de ello nadie puede desatarla.
	
	Por lo tanto, el Sabio cuida a todo el mundo\\
	sin abandonar ni uno solo.\\
	Él cuida de todas las cosas\\
	sin abandonar nada.\\
	Eso se llama ``seguir la luz''.
	
	¿Qué es un buen hombre?\\
	Es el maestro de un mal hombre.\\
	¿Qué es un mal hombre?\\
	La carga de un buen hombre.\\
	Si el maestro no es respetado,\\
	O el estudiante no es atendido,\\
	Surge la confusión, a pesar de que uno de ellos sea astuto.
	
	Esto es un cruce de carreteras, un misterio profundo.
	
	\chapter*{Veintiocho}
	
	Conoce la fortaleza del hombre\\
	¡Pero conserva la sensibilidad de la mujer!\footnote{Las primeras líneas: No se trata de una distinción entre los sexos, sino
		entre el yang y la yin. Muchos seudo eruditos de esto y de los otros
		escritos chinos identifican lo masculino, yang, con los hombres, y lo
		femenina, yin, con las mujeres. En la realidad, ambos sexos poseen
		iguales ``cantidades'' de cada uno. Solamente los usamos en las maneras
		distintas.}\\
	¡Sé la corriente del universo!\\
	Siendo la corriente del universo,\\
	Siempre veraz y confiable,\\
	Te convertirás nuevamente en un niño.
	
	Conoce lo blanco,\\
	¡Pero mantén lo negro!\\
	¡Sé un ejemplo para el mundo!\\
	Siendo un ejemplo para el mundo,\\
	Siempre-veraz y confiable,\\
	Retornarás al Infinito.
	
	Conoce el honor\\
	Pero conserva la humildad.\\
	Ser el valle del universo.\\
	Siendo el valle del universo,\\
	Siempre-veraz y abundante,\\
	Retornarás al estado de una madera sin tallar.
	
	Cuando un trozo de madera es tallado se convierte en algo útil.\\
	En manos del Sabio su uso se acelerado.\\
	Por ello ``un gran sastre corta poco''.
	
	\chapter*{Veintinueve}
	
	¿Piensas que puedes controlar el universo y mejorarlo?\\
	Yo no creo que eso pueda lograrse.\\
	El universo es sagrado. Perfecto.\\
	Tú no puedes mejorarlo.\\
	Si tratas de cambiarlo, lo arruinarás.\\
	Si tratas de mantenerlo, lo perderás.
	
	Así, a veces las cosas están adelantadas y a veces están atrasadas.\\
	A veces es difícil respirar, y otras veces es fácil.\\
	A veces uno está robusto, y otras veces débil.\\
	A veces uno está arriba y otras veces abajo.
	
	Por lo tanto, el Sabio evita\\
	los extremos, los excesos y la satisfacción de sí mismo.
	
	\chapter*{Treinta}
	
	Cada vez que señales a un gobernante el camino del Tao,\\
	Aconseja que no use la fuerza para conquistar el universo.\\
	Pues eso sólo causará resistencia.\\
	Por donde han pasado los ejércitos crece la mala hierba.\\
	Después de una gran guerra vienen años difíciles.
	
	Solo haz lo que sea necesario.\\
	Nunca tomes ventaja del poder.\\
	Obtén resultados,\\
	Pero no te glorifiques en ellos.\\
	Obtén resultados,\\
	Pero nunca te jactes.\\
	Obtén resultados,\\
	Pero nunca te enorgullezcas.\\
	Obtén resultados,\\
	Porque ése es el camino de la naturaleza.\\
	Obtén resultados,\\
	Pero sin usar la violencia.
	
	La fuerza es seguida por una pérdida de poder.\\
	Ese no es el camino del Tao.\\
	Aquel que va contra el Tao, muere joven.
	
	\chapter*{Treinta y uno}
	
	Las buenas armas son instrumentos de miedo; todas las criaturas las
	odian.\\
	Por lo tanto, los seguidores del Tao no las usan jamás.\\
	El hombre sabio prefiere el lado izquierdo;\\
	El hombre de guerra prefiere el lado derecho\footnote{`La izquierda'' es un símbolo para yin, `la derecha' un símbolo para
		yang.}.
	
	Las armas son instrumentos del miedo;\\
	No son instrumentos para el hombre sabio.\\
	El las usa sólo cuando no existe otra alternativa.\\
	La paz y el silencio son queridos por su corazón,\\
	Y la victoria no es causa de regocijo.\\
	Si te regocijas en la victoria, entonces te deleitas matando.\\
	Si te deleitas matando, entonces no encontrarás tu propia realización.
	
	En ocasiones felices, la preferencia es hacia la izquierda;\\
	En ocasiones tristes es hacia la derecha.\\
	En un ejército, el general se coloca a la izquierda;\\
	El comandante en jefe se coloca a la derecha.\\
	Esto significa que la guerra es conducida como un funeral.\\
	Donde mucha gente terminará muerta,\\
	Ellos deben ser llorados con pesar y profunda angustia por los
	vencedores.\\
	Por eso una victoria debe ser considerada como un funeral.
	
	\chapter*{Treinta y dos}
	
	El Tao permanece indefinido por siempre.\\
	Aunque es pequeño no puede ser contenido.\\
	Si reyes y poderosos pudiesen controlarlo,\\
	Todas las cosas separadas del mundo se fundirían en Uno,\\
	Y caería una suave lluvia.\\
	Los hombres no necesitarían más instrucción,\\
	Y todas las cosas tomarían su curso natural.
	
	Una vez que lo Indiviso se divide, las cosas necesitan ser nombradas\footnote{No podemos experimentar directamente a lo que hemos asignado un
		nombre. No podemos saber su esencia, porque la palabra mueve nuestra
		mente desde el mundo de la experiencia inmediata a un mundo que consiste
		solamente en los símbolos. Esta transición se puede comparar a la
		transición desde vivo a muerto. Cuando nombramos, matamos nuestra
		experiencia, y nunca la recobraremos hasta que el pensamiento termine.
		El niño tiene una conexión pura con la vida, hasta que le enseñamos a
		aprender a memoria las palabras distintas para `las cosas distintas'.
		Eso es matar su gozo. Quizás, eso es peor que matar su cuerpo. Después
		que hemos pasados un cierto tiempo en el mundo de los símbolos, las
		experiencias de la Realidad, lo interior, son solamente otras `cosas'
		con nombres. No son ya las experiencias puras y directas. Por eso, aquí
		Lao Tse dice con firmeza que debemos desprendernos del mundo falso de
		los hechos, los números, los datos y la información en que vivimos
		inmersos.}.\\
	¡Ya existen suficientes nombres!\\
	¡Uno debe saber cuándo detenerse!\\
	Saber cuándo detenerse previene los problemas.
	
	El Tao en el mundo es como un gran río que fluye hacia su hogar en el
	mar.
	
	\chapter*{Treinta y tres}
	
	Conocer a los demás es inteligencia;\\
	Conocerse al Ser es iluminación.\\
	Controlar a otros requiere de fuerza;\\
	Controlarse a sí mismo requiere alma.
	
	El que sabe que tiene lo suficiente es rico.\\
	La perseverancia es un signo del poder del alma.\\
	El que acepta donde mismo está se fortalece\footnote{El que puede establecerse en paz dondequiera que está ha encontrado su
		verdadero hogar. Esta línea es un ejemplo excelente de los muchos
		niveles en que pueden ser interpretadas las palabras de Lao Tse. El
		lector puede comenzar tomando su sentido literal, luego, a través de la
		meditación puede descubrir muchos nuevos significados.}.\\
	Quien logra aniquilar su ego antes de morir estará presente por siempre.
	
	\chapter*{Treinta y cuatro}
	
	El gran Tao fluye a través de todas las cosas,\\
	tanto por la izquierda como por la derecha.\\
	Los diez mil objetos dependen de él; pero él nada retiene.\\
	Él entrega su totalidad silenciosamente y sin exigir nada.\\
	Y hace que los diez mil objetos maduren;\\
	Aunque sin dominarlos.\\
	El Tao no tiene una dirección --- es el más suave.
	
	Cuando los diez mil objetos mueren, retornan a él;\\
	Aunque sin dominarlos.\\
	Es muy grande.
	
	No muestra su grandeza,\\
	Y por ello es verdaderamente grande.
	
	\chapter*{Treinta y cinco}
	
	Con el tiempo, todos los hombres acuden a quien sigue al Uno.\\
	Porque allí residen el descanso, el bienestar y la paz.\\
	Los paseantes pueden distraerse con música y buena comida,\\
	Pero una descripción del Tao\\
	Les parece insustancial y sin sabor.\\
	No puede ser visto, no puede ser escuchado,\\
	Sin embargo jamás se llega a agotar.
	
	\chapter*{Treinta y seis}
	
	Si algo se contrae, primero se debe expandir.\\
	Si algo falla, primero debió estar bueno.\\
	Si algo es derribado, primero debió estar en pie.\\
	Antes de recibir, hubo que dar.\\
	Esto se llama ``la percepción de la naturaleza de las cosas''.\\
	Lo suave y lo débil vence a lo que es duro y resistente.
	
	Los peces no pueden dejar las profundidades,\\
	Y las armas de un país no deben estar a la vista.
	
	\chapter*{Treinta y siete}
	
	El Tao se fundamenta en el ``No-Actuar''.\\
	Sin embargo nada se deja de lado.\\
	Si los reyes y poderosos observaren esto,\\
	Entonces los diez mil objetos se desarrollarían naturalmente.\\
	Si ellos aun así decidiesen actuar,\\
	Sus acciones retornarían a la simplicidad de la sustancia informe.\\
	Sin forma no existe deseo;\\
	Sin deseo hay tranquilidad.\\
	Y de esta manera todas las cosas estarán tranquilas.
	
	\chapter*{Treinta y ocho}
	
	Un hombre verdaderamente bueno no está consciente de su bondad,\\
	Y por ello es bueno.\\
	Un hombre torpe trata de ser bueno,\\
	Y por ello no es bueno.
	
	Un hombre verdaderamente bueno no hace nada,\\
	Pero tampoco deja nada sin hacer.\\
	Un hombre torpe está siempre ocupado,\\
	Sin embargo deja mucho por hacer.
	
	Cuando un hombre verdaderamente bondadoso hace algo,\\
	no deja nada sin hacer.\\
	Cuando un hombre de preceptos hace algo, deja mucho por hacer\footnote{La referencia al futuro se puede interpretar como una advertencia de no
		quedar atrapado en un vuelo del ego sobre la clarividencia, y como una
		advertencia de no preocuparse o proyectar el futuro en general. Ambas
		cosas son meras diversiones.}.\\
	Cuando un hombre autoritario hace algo y nadie responde,\\
	Se sube las mangas en un intento de imponer el orden.
	
	Por lo tanto cuando se pierde el Tao, aparece la bondad.\\
	Cuando se pierde la bondad, aparece la amabilidad.\\
	Cuando se pierde la amabilidad, aparece los preceptos.\\
	Cuando se pierde los preceptos, aparece el ritual.\\
	Ahora bien, el ritual es la cáscara se llama ``fe y lealtad;'' comienza
	la confusión.\\
	Conocer lo que `debe ser' es sólo un adorno florido del Tao.\\
	Es el comienzo del sin sentido.
	
	Por lo tanto, el hombre realmente de grandeza hace hincapié en lo que es
	real,\\
	y no en lo que está en la superficie;\\
	En la fruta y no en la flor.\\
	Por lo tanto, para obtener lo real es necesario rechazar lo superficial.
	
	\chapter*{Treinta y nueve}
	
	Estas cosas desde las épocas antiguas emergen del Uno:\\
	El cielo es saludable y transparente;\\
	La tierra es saludable y firme;\\
	El espíritu es saludable y fuerte;\\
	El valle es saludable y prolífico;\\
	Los diez mil objetos son saludables y animados;\\
	Reyes y poderosos son saludables y el país es justo;\\
	Todos ellos existen a causa de lo Indiviso en ellos.
	
	La transparencia del cielo evita su desplome;\\
	La firmeza de la tierra evita que se divida;\\
	La fortaleza del espíritu evita su agotamiento;\\
	Lo prolífico del valle evita que se marchite;\\
	El crecimiento de los diez mil objetos evita que pierdan su sentido;\\
	El liderazgo de reyes y poderosos evita la caída del país.
	
	Por lo tanto, en lo humilde está la raíz de la nobleza.\\
	Lo bajo es fundamento para lo alto.\\
	Príncipes y poderosos se consideran a sí mismos como ``huérfanos'',
	``viudos'' y ``despreciables''.\\
	¿Acaso no obtienen su poder por ser humildes?
	
	Demasiado éxito no constituye una ventaja.\\
	No brillan como el jade\\
	Ni repican como colgantes de piedras.
	
	\chapter*{Cuarenta}
	
	Retornar es la dirección del Tao,\\
	Ceder es el método del Tao.\\
	Los diez mil objetos nacieron del ser,\\
	Y el ser nació del no ser.
	
	\chapter*{Cuarenta y uno}
	
	El estudiante sincero escucha el Tao y lo practica con perseverancia.\\
	El estudiante típico escucha el Tao y piensa en él de vez en cuando.\\
	El estudiante torpe escucha el Tao y se ríe a carcajadas.\\
	Si no provocara esa risa, el Tao no sería lo que es.
	
	Por lo tanto, se puede decir:\\
	La senda brillante parece oscura;\\
	Al avanzar parece que retrocediera;\\
	El camino más fácil parece difícil;\\
	La más alta Virtud parece inútil;\\
	Lo más puro parece sucio.
	
	La plenitud de Virtud parece inadecuada;\\
	La fuerza de la Virtud parece frágil;\\
	La Virtud real parece irreal;\\
	Cuando el cuadrado está perfecto no se ve las esquinas!
	
	Los mejores talentos maduran tarde;\\
	El gran Sonido es el Silencio;\\
	La más bella forma es amorfa.
	
	El Tao está oculto e incógnito.\\
	Solo el Tao nutre y lleva a todo a su realización.
	
	\chapter*{Cuarenta y dos}
	
	El Tao engendra a uno.\\
	El uno engendra a dos.\\
	Los dos engendran a tres.\\
	Y los tres engendran los diez mil objetos.\\
	Los diez mil objetos llevan la yin y abrazan el yang;\\
	Ellos logran la armonía combinando esas fuerzas.
	
	Los hombres odian sentirse ``huérfanos'', ``viudos'' o
	``despreciables'',\\
	Pero así es como se describen a sí mismos los reyes y poderosos.
	
	Porque uno gana perdiendo\\
	Y pierde ganando.
	
	Yo enseño lo que otros enseñan, es esto:\\
	``¡Un hombre que vive por la violencia morirá por la violencia!''\footnote{``Cosechamos lo que sembramos'', por cierto. Un hombre que vive para la
		violencia no es precisamente un hombre que asalta las personas y las
		cosas físicamente, sino uno que viola u abusa a otros de alguna manera.
		Debemos aprender a aceptar el abuso cuando es nuestro destino, y esa
		negativa exigir ojo por un ojo, es una absolución poderosa. Para el
		Sabio, si es uno muy sensible, incluso discutir o defender cuando puede
		quedarse silencioso es una forma de violar. El Sabio es refinado y
		templado, y tiene consciencia del camino sin salido en que ponemos otra
		persona cuando llevamos su mente hacia lo físico. Si en cambio dejamos
		la persona tranquila, siempre es posible que pueda posarse en lo Real,
		en la conciencia sin palabras, sin acciones.}\\
	Eso será lo esencia de mi enseñanza.
	
	\chapter*{Cuarenta y tres}
	
	La cosa más suave en el universo\\
	Vence a la cosa más rígida.\\
	Eso que no tiene sustancia puede entrar adonde no hay lugar.\\
	Entonces yo aprendo el valor de la No-Actuar.
	
	Enseñar sin palabras, y trabajar sin hacer:\\
	Es un mundo donde moran muy pocos\footnote{En la entrevista ``Vagando en el camino'' Gia-fu dice que quiso hacer
		una nueva traducción de la última línea, que tradujo originalmente como
		``son entendidos por muy pocos''. Declaró que esto no comunica la fuerza
		del original en chino. Por eso propuso en cambio las palabras ``La gente
		entra allí raramente''.}.
	
	\chapter*{Cuarenta y cuatro}
	
	Ser famoso o simplemente lo Ser: ¿Qué interesa más?\\
	Ser rico o lo Ser: ¿Qué es más precioso?\\
	Ganar o perder ¿Qué es más doloroso?
	
	El que está atado a las cosas sufrirá mucho.\\
	El que ahorra sufrirá grandes pérdidas.\\
	Un hombre contento jamás sufre desilusiones.\\
	El que sabe cuándo parar, nunca se encuentra en problemas.\\
	El estará seguro por siempre.
	
	\chapter*{Cuarenta y cinco}
	
	La realización perfecta parece insuficiente,\\
	Aunque su valor es sempiterno.\\
	La gran plenitud parece vacía,\\
	Aunque no puede ser agotada.
	
	La gran rectitud parece torcida.\\
	La gran inteligencia parece estúpida.\\
	La gran elocuencia parece balbuceante.
	
	El movimiento permite vencer al frío\\
	Y la inmovilidad permite vencer al calor.
	
	La quietud y tranquilidad reestablecen el orden en el universo.
	
	\chapter*{Cuarenta y seis}
	
	Cuando el Tao está presente en el universo\\
	Hay caballos que acarrean el estiércol.\\
	Cuando el Tao está ausente del universo\\
	Se crían caballos de guerra fuera de la ciudad.
	
	No existe peor pecado que el deseo;\\
	No hay peor maldición que el descontento;\\
	No hay peor desgracia que desear algo para sí mismo.
	
	Por lo tanto, el que está contento con lo suficiente, tendrá siempre lo
	suficiente.
	
	\chapter*{Cuarenta y siete}
	
	Sin salir de tu casa, puedes conocer el mundo entero.\\
	Sin mirar por la ventana, puedes ver el método del cielo.\\
	Mientras más lejos vayas, menos conocerás.
	
	Así el Sabio conoce sin tener que viajar;\\
	Él ve sin mirar;\\
	Él trabaja sin actuar.
	
	\chapter*{Cuarenta y ocho}
	
	En la búsqueda del conocimiento, cada día adquirimos algo;\\
	En la búsqueda del Tao, cada día abandonamos algo.
	
	Se hace menos y menos cada vez hasta lograr el No-Actuar.\\
	Cuando no se hace nada, nada queda por hacer.
	
	Quien maneja el mundo deja que las cosas tomen su curso natural;\\
	No se puede manejar interfiriendo.
	
	\chapter*{Cuarenta y nueve}
	
	El Sabio no tiene voluntad propia;\\
	Él está consciente de las necesidades de otros.
	
	Yo soy bueno con la gente que es buena,\\
	Y también soy bueno con la gente que no es buena,\\
	Porque la Virtud es la bondad.\\
	Yo tengo fe en la gente confiable,\\
	Y también tengo fe en aquellos que no son confiables,\\
	Porque la Virtud es la fidelidad.
	
	El Sabio es tímido y humilde -- a los ojos del mundo es
	contradictorio.\\
	Otros lo ven y lo escuchan.\\
	Él se comporta como un niño pequeño.
	
	\chapter*{Cincuenta}
	
	Entre el nacimiento y la muerte,\\
	Tres de cada diez ponen su fe en la vida,\\
	Tres de cada diez ponen su fe en la muerte,\\
	Y los hombres que sólo pasan desde su nacimiento hasta la muerte sin
	fe.\\
	También son tres de cada diez.\\
	¿Por qué ocurre esto?\\
	Porque ellos viven su vida en un nivel craso
	
	El que sabe cómo vivir puede llegar a cualquier parte\\
	sin miedo del rinoceronte o el tigre.\\
	Él no será herido en la batalla.\\
	Porque en él, el rinoceronte no encuentra donde hincar su cuerno.\\
	Y el tigre no encuentra un lugar donde usar sus garras.\\
	Y las armas no encuentran lugar donde enterrarse.\\
	¿Por qué ocurre esto?\\
	Porque este Sabio ya ha muerto;\\
	su vacuidad no deja lugar por donde la muerte pueda entrar\footnote{Tal como el ratón que, permaneciendo inmóvil y relajado no excita el
		gato, la tranquilidad en el Sabio no atrae el ataque. Cuando una parte
		del cuerpo se tiene en tensión, para preparar el ataque o correr, esa es
		la parte que atrae la atención de otra persona. La energía vital (ch'i)
		no fluye libremente a través esta parte, sino que se concentra allí.
		Sensorialmente, y también quizás de modo clarividente, uno puede
		percibir eso, y un movimiento defensivo se moviliza de su propio ch'i en
		reacción. Es como un imán. Pero, hay un nivel más profundo, en que la
		pureza del sabio no atrae ninguno ataque, porque no tiene ese karma.}.
	
	\chapter*{Cincuenta y uno}
	
	Todas las cosas surgen del Tao.\\
	Son nutridas por la Virtud;\\
	Son formadas por la materia;\\
	Son modeladas por el entorno.
	
	Por lo tanto, los diez mil objetos reflejan el Tao y llevan a la
	Virtud.\\
	El respeto del Tao y el honor de la Virtud no necesitan ser invocados,\\
	Porque ellos están en la naturaleza de las cosas.
	
	Por lo tanto, todas las cosas surgen del Tao.\\
	Por la virtud son nutridas,\\
	Desarrolladas, cuidadas,\\
	Cobijadas, confortadas,\\
	Cultivadas y protegidas.
	
	Crear sin invocación,\\
	Hacer sin atribuirse ningún mérito,\\
	Guiar sin interferir:\\
	Esta es la Virtud Primigenia.
	
	\chapter*{Cincuenta y dos}
	
	El principio del universo\\
	Es la madre de todas las cosas.\\
	Conociendo a la madre, uno también conoce a su sucesión.\\
	Conociendo a su sucesión, pero permaneciendo en contacto con la madre,\\
	Nos libera del miedo y de la muerte, aunque su cuerpo sea aniquilado\footnote{La línea: ``aunque\ldots'' se toma de la versión de John Wu; se omite en
		Feng y English. Sospechamos que tales referencias a lo místico se omiten
		a propósito por Gia-fu.}.
	
	Mantén la boca cerrada,\\
	Descansa los sentidos,\\
	Y la vida será siempre plena.\\
	Abre tu boca,\\
	Mantente siempre ocupado,\\
	Y la vida no tendrá esperanza.
	
	Ver lo inadvertido es intuición.\\
	Ceder ante la fuerza es fortaleza.\\
	Usa tu brillo externo, pero vuelve a la luz interior.\\
	Y de esa manera te salvarás de todo daño.\\
	Eso es el camino a la inmortalidad.
	
	\chapter*{Cincuenta y tres}
	
	Si yo tuviera solo un poco de cordura,\\
	Iría por el camino principal,\\
	y mi único temor sería extraviarme de él.\\
	Mantenerse en el camino principal es fácil ---\\
	¡Pero las personas prefieren divertirse!\footnote{A la gente le encanta divertirse, y no decimos esto como un juicio
		moralista contra la diversión, el goce. Sin embargo, al desviarse se
		complica la vida y pierde el camino directo al Tao, que está disponible
		en cada momento. El jugar en las ilusiones, no obstante, es jugar en el
		Tao, con una de las potencialidades del Tao, y siempre ofrece una
		lección dura en la perfección de la ley de karma en la vida.}
	
	Cuando el palacio está adornado con esplendor,\\
	Los sembrados se llenan de mala hierba\\
	Y los graneros están vacíos.\\
	Algunos visten ropas elegantes,\\
	Llevan espadas afiladas,\\
	Y se entregan a la comida y bebida.\\
	Ellos tienen más posesiones de las que pueden usar.\\
	Ellos son barones bandidos.
	
	Ese ciertamente no es el camino del Tao.
	
	\chapter*{Cincuenta y cuatro}
	
	Lo que está establecido con firmeza no puede ser desarraigado.\\
	Lo que está sujeto con certeza no puede soltarse.\\
	Será honrado a través de las generaciones.
	
	Cultiva la Virtud en ti mismo\\
	Y esa Virtud será real.\\
	Cultívala en la familia\\
	Y la Virtud multiplicará.\\
	Cultívala en el pueblo\\
	Y la Virtud crecerá.\\
	Cultívala en el país\\
	Y la Virtud será abundante.\\
	Cultívala en el universo\\
	Y la Virtud estará en todas partes.
	
	Por lo tanto, trata al cuerpo como exactamente lo que sea;\\
	Trata a la familia como exactamente lo que sea;\\
	Trata a tu pueblo como exactamente lo que sea;\\
	Trata a tu país como exactamente lo que sea;\\
	Trata al universo como exactamente lo que sea.
	
	¿Cómo puedo saber los que sean esos?\\
	¡Por observación!
	
	\chapter*{Cincuenta y cinco}
	
	El que está pleno de Virtud es como un recién nacido.\\
	No lo pican las avispas, no lo muerden las serpientes;\\
	Las bestias salvajes no saltarán sobre él;\\
	No será atacado por las aves de presa.\\
	Sus huesos son suaves, sus músculos débiles,\\
	Pero su confianza es firme.\\
	Él no ha experimentado la unión de hombre y mujer, pero es completo.\\
	Su ser es lleno con energía vital; él es indivisible.\\
	Llora todo el día sin quedar ronco.\\
	Esta es armonía perfecta.
	
	Conocer la armonía es la gran Certeza\\
	Y conocer la gran Certeza es iluminación.
	
	No es sabio apresurarse.\\
	Controlar el aliento nos pone tensos.\\
	Si se usa demasiada energía, luego se extingue la vida.\\
	Este no es el camino del Tao.\\
	Todo lo que es contrario al Tao no dura mucho.
	
	\chapter*{Cincuenta y seis}
	
	Aquel que sabe no habla;\\
	Aquel que habla no sabe.
	
	Mantén tu boca cerrada,\\
	Modera tus sentidos,\\
	Templa tu filo,\\
	Y simplificarás tus problemas.\\
	Enmascara tu brillo,\\
	Hazte uno con el polvo de la tierra:\\
	Esta es la Unión Primigenia.
	
	El que ha logrado ese estado\\
	No se preocupa de amigos ni enemigos,\\
	Con bien y mal, con honor y desgracia.\\
	Este es, por lo tanto, el más alto estado del hombre.
	
	\chapter*{Cincuenta y siete}
	
	Gobierna una nación imparcialmente.\\
	Haz la guerra\footnote{``Haz la guerra'', como muchas referencias al gobierno y la política, es
		un símbolo para ``vive tu vida''. No debe ser confundido con rutinas
		automáticas, las que se imponen desde las cosas y la gente, las reglas y
		las palabras.} con acciones imprevistas.\\
	Te convierte en maestro del universo sin esfuerzo.\\
	¿Cómo sé que esto es así?\\
	¡Porque es así!:
	
	Mientras más leyes e impuestos existan\\
	Más se empobrece la gente.\\
	Mientras más afiladas son las armas\\
	Más problemas sobre la tierra.\\
	Mientras más ingeniosos y astutos es la gente\\
	Ocurren las cosas más extrañas.\\
	Mientras más tabúes y prohibiciones\\
	Habrá más ladrones y criminales.
	
	Por lo tanto, el Sabio dice:\\
	``Yo no actúo y la gente se transforma.\\
	Yo disfruto la paz y la gente se convierte en honrada.\\
	Yo no hago nada y la gente prospera.\\
	Yo no tengo deseos y la gente retornará a la vida buena y simple''.
	
	\chapter*{Cincuenta y ocho}
	
	Cuando el país es gobernado con mano suave,\\
	La gente es simple.\\
	Cuando el país es gobernado con severidad,\\
	La gente se torna astuta\footnote{Este capítulo es una advertencia clara y sucinta contra la presión a
		conformarse a la norma. Cuando tratamos proscribir lo que una persona
		``debe ser,'' le hacemos perder el control de su naturaleza verdadera;
		induciéndolo a seguir un ideal. Todo se vuelve agrio. Por eso, sentimos
		que la palabra ``honradez'' de Gia-fu en verdad tiene el sentido de
		``normal''. El esfuerzo a ser ``normal'', ``adaptarse'', ``encajar'' ha
		embrujado, confundido, obsesionado, engañado, y desviado al hombre por
		un largo tiempo. No le permite ver quién es en realidad.}.
	
	La felicidad tiene sus raíces en la miseria,\\
	La miseria atisba detrás de la felicidad\\
	¿Quién sabe lo que deberá el futuro?\\
	No hay `lo normal'.\\
	`Lo normal' se transforma en `lo no normal'.\\
	La bondad se transforma en un embrujo.\\
	Y después que un hombre es engañado por la ilusión, ésta dura un largo
	tiempo.
	
	Por lo tanto, el Sabio es afilado, pero no corta;\\
	Puntiagudo, pero no perfora;\\
	Franco, pero no malo;\\
	Brillante, pero no deslumbra.
	
	\chapter*{Cincuenta y nueve}
	
	Guiar a los demás y servir al cielo\\
	No hay nada comparado con usar la templanza.\\
	La templanza\footnote{``La templanza'' --- El texto original dice: ``la restricción''.} comienza cuando abandonamos nuestras propias ideas.\\
	Esto depende de la Virtud acumulada en el pasado.\\
	Si hay una buena cantidad de Virtud acumulada, nada es imposible.\\
	Si nada es imposible, no existen límites\\
	Si un hombre conoce lo Ilimitado, entonces está listo para gobernar.
	
	El principio madre para gobernar ha existido durante largo tiempo;\\
	Y consiste en tener raíces profundas en la gran Certeza\footnote{``La gran Certeza'' es una expresión propuesta por Gia-fu en la
		entrevista ``Vagando en el camino''.}:\\
	El Tao de larga vida y visión eterna.
	
	\chapter*{Sesenta}
	
	Gobernar un pueblo es como cocinar un pescado pequeño.
	
	Vive en el universo con el Tao,\\
	Y después el mal no será poderoso.\\
	Su poder no se despertará para dañar a otros.\\
	No sólo para no dañar a otros,\\
	También el propio Sabio será protegido.\\
	Ellos no se oponen a uno al otro,\\
	Y la Virtud resultante se reparte entre ambos.
	
	\chapter*{Sesenta y uno}
	
	Un gran país es como una tierra baja;\\
	Es el punto de encuentro del universo:\\
	La madre del universo.
	
	El yin femenino vence al yang masculino usando la quietud.\\
	Escondida tras la quietud.
	
	Por lo tanto, si un gran país cede el camino a un país pequeño,\\
	De esa manera lo conquistará.\\
	Y si un país pequeño se somete a un país grande,\\
	De esa manera podrá conquistarlo.\\
	Por lo tanto, aquellos que quieran conquistar deben ceder,\\
	Y los que conquistan, lo logran porque han cedido.
	
	Un país grande necesita más servidores;\\
	Un país pequeño necesita un lugar en donde servir.\\
	Cada uno obtiene lo que necesita.\\
	Es apropiado para un gran país cooperar.
	
	\chapter*{Sesenta y dos}
	
	Tao es la fuente de los diez mil objetos.\\
	Es el tesoro del hombre bueno y el refugio del malo.\\
	Con palabras dulces se puede comprar el honor;\\
	Las buenas acciones hacen ganar respeto.\\
	Si un hombre es malo, no lo abandones.
	
	Por lo tanto, en el día de la coronación del emperador,\\
	O cuando los ministros sean instalados,\\
	No envíes un regalo de jade o un grupo de cuatro corceles;\\
	Permanece tranquilo y ofrece el Tao.
	
	¿Por qué la gente aprecia el Tao en fin por sobre las demás cosas?\\
	¿No será porque encuentras lo que buscabas y tus errores son
	absorbidos?\\
	Por lo tanto, éste es el mayor tesoro del universo.
	
	\chapter*{Sesenta y tres}
	
	Practica el No-Actuar.\\
	Trabaja sin hacer.\\
	Paladea lo que no tiene gusto.\\
	Honra a los humildes, reconoce lo insólito.\\
	Contesta a la amargura con cariño.\\
	Busca la simplicidad en lo complicado.\\
	Logra la grandeza en las pequeñas cosas.
	
	En el universo, las cosas difíciles se hacen como si fueran fáciles\footnote{Nosotros sentimos que este capítulo parece se ser un resumen de las
		precondiciones esenciales del milagro. Un milagro es una interrupción
		del curso ordinario de las cosas. Cuando paramos de soportar el curso
		ordinario de las cosas y simplemente estamos quietos y tranquilos,
		permitimos el acontecimiento raro. Sin embargo, si tenemos miedo de
		algún acontecimiento raro en particular, nuestra tensión se roba la
		energía, y el milagro se hace muy improbable. Si tenemos algún tipo de
		interés cualquiera, nuestra impureza lo impide. Por eso, debemos
		estabilizar en un estado lo que es libre de ego. Quizás más milagros
		sucedieron en la era de Lao Tse porque había más gente con menos ego.
		Eso es una especulación. Pero es cierto que los humanos tuvieron la
		potencialidad vivir mucho menos egocéntricamente que nosotros, que
		vivimos en sociedades de alta tecnología.

Es una pregunta empírica: Sigue el tipo de vida de lo que Lao Tse da un
resumen para nosotros, para ver si los milagros suceden.	
}.\\
	En el universo, las grandes acciones se logran con pequeños pasos.\\
	El Sabio no intenta nada que sea demasiado grande,\\
	Y así logra la grandeza.
	
	Las promesas fáciles no son confiables.\\
	No se toman en serio y provocan grandes dificultades.\\
	Como el Sabio no evita las dificultades,\\
	Nunca llega a experimentarlas.
	
	\chapter*{Sesenta y cuatro}
	
	La paz se mantiene con facilidad;\\
	Los problemas se pueden vencer fácilmente antes de que comiencen.\\
	Lo inflexible se despedaza con facilidad\\
	Lo pequeño se esparce con facilidad.\\
	Hay que resolverlo antes de que ocurra;\\
	Ordenar las cosas antes de eso lleva a la confusión.
	
	Un árbol tan grande como el abrazo de un hombre nace de un brote
	diminuto;\\
	Una terraza de nueve pisos de altura comienza con una pila de tierra;\\
	Un viaje de mil millas comienza con un primer paso.
	
	Él que actúa, derrota sus propios propósitos.\\
	Él que se apropia pierde.\\
	El Sabio no actúa y por eso no es derrotado.\\
	Él no se apropia y por eso no pierde.
	
	La gente usualmente falla cuando está al borde del éxito.\\
	Así es que hay que poner el mismo cuidado tanto al final como al
	comienzo\\
	Así no habrá fracaso.
	
	Por lo tanto, el Sabio aspira deshacerse del deseo.\\
	El no acumula objetos preciosos.\\
	El aprende no para adquirir ideas.\\
	El lleva de vuelta a los hombres hacia lo que han perdido.\\
	El ayuda a los diez mil objetos a encontrar su propia naturaleza,\\
	Pero se refrena en la acción.
	
	\chapter*{Sesenta y cinco}
	
	En el principio, aquellos que conocían el Tao no trataban de iluminar a
	otros,\\
	Sino que lo mantenían oculto.
	
	¿Por qué es tan difícil de enseñar?\footnote{Nos hemos tomado una libertad aquí y hemos cambiado el sentido de la
		parte última de este capítulo para que siga más claramente las primeras
		líneas. El original dice: ``¿Por qué está tan difícil gobernar?'' y
		continúa hablando sobre los soberanos y como se gobierna un país.}\\
	Porque la gente es muy astuta.\\
	Pero, si el maestro trata de usar la astucia,\\
	Engaña a la gente.\\
	Aquellos que enseñan sin astucia\\
	Son una bendición para los demás.\\
	Estas son las dos alternativas.\\
	Entender esto es la Virtud Primigenia.\\
	La Virtud Primigenia es profunda y llega lejos.\\
	Trae a todas las cosas de vuelta\\
	Hacia la gran Unidad.
	
	\chapter*{Sesenta y seis}
	
	¿Por qué es el mar rey de las cien corrientes?\\
	Porque descansa debajo de ellas.\\
	Por lo tanto es el rey de las cien corrientes.
	
	Si el Sabio conduce a la gente, debe servir con humildad.\\
	Si él los guía, debe seguirlos atrás.\\
	De esta manera, cuando el Sabio gobierna, la gente no se siente
	oprimida;\\
	Cuando permanece detrás de ellos no se sienten amenazados.\\
	Todo el mundo lo apoyará y no se aburrirán de él.
	
	Como él no rivaliza,\\
	Tampoco tendrá rivales.
	
	\chapter*{Sesenta y siete}
	
	Todo el mundo dice que mi Tao es grande y más allá de toda
	comparación.\\
	Como es grande, parece diferente.\\
	Si no fuese diferente se habría desvanecido mucho tiempo atrás.
	
	Tengo Tres Tesoros que guardo con cuidado y por ellos vivo:\\
	El primero es la misericordia sin preferencia.\\
	El segundo es nunca en exceso.\\
	Y el tercero es atreverse a no competir con los demás.
	
	De la misericordia viene el coraje.\\
	De la frugalidad viene la generosidad.\\
	De la humildad viene el liderazgo.
	
	En estos días, los hombres rechazan la misericordia, y tratan de ser
	valientes.\\
	Abandonan la frugalidad, y tratan de ser generosos.\\
	No creen en la humildad, sino que siempre tratan de ser primeros.\\
	Esta es una muerte cierta.
	
	La misericordia trae la victoria en las batallas y la fortaleza en la
	defensa.\\
	De esta manera, el cielo salva y protege.
	
	\chapter*{Sesenta y ocho}
	
	Un buen soldado no es violento;\\
	Un buen luchador no siente furia;\\
	Un buen vencedor no es vengativo;\\
	Un buen jefe es humilde.\\
	Esto se conoce como la virtud de no forzar las cosas.\\
	Esto se conoce como el talento de tratar con la gente.\\
	Esto se ha conocido desde tiempos antiguos\\
	como la máxima unidad con el cielo.
	
	\chapter*{Sesenta y nueve}
	
	Existe un dicho entre los soldados:\\
	``Antes de hacer el primer movimiento prefiero esperar que lo haga el
	adversario;\\
	Antes de avanzar una sola pulgada, mejor retrocedo un pie''.\\
	Esto se llama marchar sin que parezca que te estás moviendo,\\
	Arremangar las mangas sin mostrar los músculos,\\
	Capturar al enemigo sin atacar,\\
	Estar armado sin tener armas.
	
	No hay mayor catástrofe que subestimar al enemigo.\\
	Subestimando al enemigo, prácticamente pierdo todo lo que valoro.
	
	Por lo tanto, cuando la batalla final se está librando,\\
	El que ha aceptado la aflicción más triste es quien triunfará.
	
	\chapter*{Setenta}
	
	Mis palabras son fáciles de entender y fáciles de llevar a la
	práctica,\\
	Pero ningún hombre bajo este cielo verdaderamente\\
	las comprende ni las pone en práctica.
	
	Mis palabras tienen raíces antiguas.\\
	Mis acciones están controladas por el Todo\\
	Como los hombres no comprenden, ellos no tienen conocimiento de mí.
	
	Son pocos aquellos que me conocen.\\
	Los que me insultan son ensalzados.\\
	Por lo tanto, el Sabio viste ropas gastadas y oculta la Joya en su
	corazón.
	
	\chapter*{Setenta y uno}
	
	Reconocer la ignorancia es fortaleza;\\
	Ignorar la sabiduría es una locura.
	
	Cuando uno se vuelve loco reconociendo esta locura, no es loco.\\
	El Sabio no está loco, porque él está loco y lo reconoce.\\
	Por lo tanto, él no es loco.
	
	\chapter*{Setenta y dos}
	
	Cuando un hombre no siente respeto, entonces viene el desastre.\\
	No irrumpir en sus hogares.\\
	No los molestar en sus trabajos.\\
	Si uno no interfiere, lo acogerán con gusto.
	
	Por lo tanto, el Sabio se conoce a sí mismo, pero no lo demuestra;\\
	Se respeta al Ser, pero no es arrogante.\\
	Él deja ir las apariencias y escoge lo esencial.
	
	\chapter*{Setenta y tres}
	
	Un hombre valiente y apasionado mata o es muerto;\\
	Un hombre valiente y calmado siempre preserva la vida.\\
	De estos dos ¿Cuál es bueno y cual es dañino?\\
	Algunas cosas no son favorecidas por el cielo ¿Quién sabe por qué?\\
	Incluso el Sabio esta inseguro de esto.
	
	El Tao del cielo no lucha, aun así vence;\\
	No habla, aun así recibe respuesta;\\
	No pide, aun así todas sus necesidades son satisfechas.\\
	Parece que no tuviese meta sin embargo sus propósitos son logrados.
	
	Las redes del cielo se estiran por todas partes.\\
	Sus nudos están separados, pero nada se le escapa.
	
	\chapter*{Setenta y cuatro}
	
	Si un hombre no le teme a la muerte\\
	No se gana nada amenazando su vida.\\
	Si un hombre vive en constante miedo a morir,\\
	Y romper la ley implica que el hombre será ejecutado,\\
	¿Quién se atreve a romper la ley?
	
	Siempre existe un verdugo oficial.\\
	Si tratas de tomar su lugar\\
	Es como tratar de ser un maestro ebanista y cortar madera.\\
	Si tratas de cortar madera como un maestro ebanista,\\
	sólo arruinarás tus manos.
	
	\chapter*{Setenta y cinco}
	
	¿Por qué hay gente hambrienta?\\
	Porque los gobernantes se comen el dinero en impuestos.\\
	Por lo tanto, hay gente hambrienta.
	
	¿Por qué se rebela la gente?\\
	Porque los gobernantes interfieren demasiado.\\
	Por lo tanto, hay rebelión.
	
	¿Por qué la gente se preocupa tan poco de la muerte?\\
	Porque los gobernantes les exigen demasiado en la vida.\\
	Por lo tanto, la gente toma la muerte a la ligera.
	
	Teniendo pocas razones para vivir, no valoras mucho la vida.
	
	\chapter*{Setenta y seis}
	
	Un hombre nace gentil y débil;\\
	A su muerte queda duro y rígido.\\
	Las plantas verdes son tiernas y llenas de savia;\\
	Cuando mueren son marchitas y secas.\\
	Por lo tanto, la dureza y rigidez son discípulos de la muerte;\\
	La gentileza y la flexibilidad son discípulos de la vida.
	
	Por eso un ejército sin flexibilidad nunca ganará una batalla;\\
	El árbol que no se dobla se romperá con facilidad.
	
	El duro y fuerte fallará;\\
	El suave y débil vencerá.
	
	\chapter*{Setenta y siete}
	
	El Tao del cielo es como un arco que se curva:\\
	Lo alto es bajado y lo bajo es levantado.\\
	Si la cuerda es demasiado larga, será acortada;\\
	Si no es suficiente será alargada.\\
	El Tao del cielo consiste en tomar de aquellos que tienen demasiado\\
	y darlo a aquellos que no tienen lo suficiente.\\
	Los caminos del hombre difieren:\\
	Ellos toman de aquellos que no tienen lo suficiente\\
	y se lo dan a aquellos que ya tienen demasiado.
	
	¿Qué hombre es ése que tiene más que suficiente y se lo entrega al
	mundo?\\
	Solo el hombre del Tao.
	
	Por lo tanto, el Sabio trabaja sin buscar el reconocimiento;\\
	El hace lo que se debe hacer sin jactancia.\\
	El no trata de ostentar su conocimiento.
	
	\chapter*{Setenta y ocho}
	
	Bajo este cielo, nada es más suave y fluido que el agua,\\
	Pero nada es mejor que el agua para atacar lo sólido y fuerte.\\
	No existe nada igual.
	
	El débil puede vencer al fuerte;\\
	El dócil puede vencer al rígido.
	
	Bajo este cielo todo el mundo sabe esto,\\
	Sin embargo nadie lo pone en práctica.
	
	Por lo tanto, el Sabio dice:\\
	El que acepta la humillación de la gente es apto para gobernar;\\
	El que acepta los desastres del país merece ser el rey del universo.\\
	La verdad a menudo parece paradójica.
	
	\chapter*{Setenta y nueve}
	
	Después de una disputa amarga, siempre queda algo de resentimiento.\\
	¿Qué podemos hacer?\\
	Por lo tanto, el Sabio respeta su parte del trato\\
	Pero no exige lo que le adeudan.\\
	Un hombre virtuoso hace su parte,\\
	Pero un hombre sin virtud requiere que otros cumplan sus obligaciones
	por él.
	
	El Tao del cielo es imparcial:\\
	¡Acompaña a los hombres buenos todo el tiempo!
	
	\chapter*{Ochenta}
	
	Un país pequeño tiene menos habitantes.\\
	Como hay máquinas que pueden trabajar diez o cien veces más rápido\\
	que un hombre, ellas no son necesarias.\\
	La gente se toma la muerte en serio y no viaja demasiado lejos.\\
	Aunque tienen botes y carruajes, nadie los usa.\\
	Aunque tienen corazas y armas, nadie las muestra.\\
	Los hombres retoman a los nudos en una cuerda y nadie vuelve a
	escribir.\\
	Su alimento es simple y bueno, sus ropas finas y simples,\\
	sus hogares seguros.\\
	Ellos son felices a su manera.\\
	Aunque viven a la vista de sus vecinos,\\
	Y los gallos bulliciosos y los perros que ladran se escuchan en las
	calles,\\
	Así viven en paz entre ellos mientras crecen, se hacen viejos y mueren.
	
	\chapter*{Ochenta y uno}
	
	Las palabras verdaderas no son agradables;\\
	Las palabras agradables no son verdaderas\footnote{Gia-fu y Jane dicen: ``Las palabras veraces no son hermosas; las
		palabras hermosas no son veraces''. Pero hay un tipo de ``hermosura''
		que se refiere a una emoción sentimental, y también hay otro tipo de
		``hermosura'' que es más profunda. Tratamos evitar esta confusión.}.\\
	Un hombre bueno no discute;\\
	Aquellos que discuten no son buenos.\\
	Aquellos que conocen, no son eruditos;\\
	El erudito no conoce.
	
	El Sabio nunca trata de almacenar las cosas.\\
	Mientras más vive para los demás más plena es su vida;\\
	Mientras más da a los otros, mayor es su abundancia.
	
	El Tao del cielo apunta, pero no causa daño.\\
	El Tao del Sabio es trabajo sin esfuerzo.
	
	\chapter*{}

	El Néctar
	
	de
	
	Chuang Tse
	
	\chapter*{}
	
	Los escritos que siguen son adaptados de Merton, T. The Way of Chuang
	Tzu (El Camino de Chuang Tse), New Directions, 1965, Abbey of
	Gethsemani. La traducción del inglés es de John Thomas Wilke y Tomàs
	Bradanovic Pozo, quienes agradecemos especialmente la traducción de
	Colodron en la versión original en español.
	
	El Padre Merton fue `un hombre de este mundo', que profesaba ``Yo no
	creo nada'', sentía una antipatía específica hacía la iglesia católica.
	Según declara en su autobiografía \textit{The Seven Storey Mountain} que
	se dio cuenta de vivir una vida sin sentido ni propósito. Es posible que
	haya tenido un hijo, con que no tuvo nunca una relación. Después mucha
	duda, debido a los excesos de su pasado, a la edad de 27 tomó los votos
	permanentes de pobreza, castidad, obediencia, y el silencio, haciéndose
	miembro de la estricta comunidad católica trapista en Gethsemani,
	Kentucky. Encontró el estímulo constante en su elección de una vida
	monástica, en la contemplación de Lao Tse y Chuang Tse, y en el budismo
	zen, así continuó desafiando los dogmas y a los líderes de esas
	religiones hasta su muerte.
	
	Merton escribió en un momento cuando la iglesia católica romana estaba
	mucho menos abierta para condonar las ideas religiosas o teológicas que
	estaban extrañas de su propia doctrina. Aunque esta obra sobre Chuang
	Tse y algunos de sus otros escritos sobre las religiones orientales
	recibieron la \textit{Imprimi Potest} y \textit{Nihil Obstat} de la
	jerarquía de la iglesia católica romana, Thomas Merton a veces desafió
	la estrechez de su propia religión. Él buscó mostrarnos la universalidad
	de la verdad, y especialmente la universalidad de una vida de soledad y
	silencio -- en contraste con los rituales sociales -- como una vía
	poderosa para distinguir la veraz desde la ilusión. Él murió de una
	manera extraña y sincrónica: fue en la Tailandia en el año 1968 para
	participarse en un congreso zen, y mientras se bañaba un ventilador
	eléctrico aparentemente se cayó al agua y lo electrocutó. Digamos
	``aparentemente'' porque alguna gente que estaba íntimamente ligados con
	él han sugerido que fue asesinado.
	
	Vivió una vida en la que atraía los opuestos fuertes, lo que decía mucho
	sobre su ser y su naturaleza humana. Uno de los eventos más poderosos de
	su historia fue una relación sexual, dos años antes de su muerte, con
	una enfermera joven la que atendía le durante su convalecencia. ¿Hasta
	qué punto en una relación sexual rompe el voto de la castidad de un
	sacerdote? Thomas desafió eso también.
	
	\chapter*{El hombre de Tao}
	
	El hombre en el cual el Tao\\
	actúa sin impedimento\\
	no daña a ningún otro ser con sus actos,\\
	y aun así no se considera sí mismo ``bondadoso'' ni ``manso''.
	
	El hombre en que el Tao\\
	actúa sin impedimento\\
	no se preocupa por sus propios intereses\\
	y no desprecia a aquellos que sí lo hacen.\\
	No lucha por ganar dinero\\
	y no convierte en virtud la pobreza.\\
	Sigue su camino sin apoyarse en los demás\\
	y no se enorgullece de andar solo.\\
	Si bien no sigue a la muchedumbre,\\
	no se queja de aquellos que lo hacen.\\
	El rango y la recompensa no lo atraen;\\
	la desgracia y la vergüenza no lo desaniman.\\
	No está buscando constantemente\\
	`el bien' y `el mal',\\
	ni decidiendo continuamente `sí' o `no.'
	
	Los antiguos decían, por tanto:\\
	``El hombre del Tao\\
	permanece en el anonimato.\\
	La virtud perfecta no produce nada.\\
	`No-ser' es el `Ser verdadero,'\\
	y el más grande entre los hombres es el señor nadie''.
	
	\chapter*{La alegría perfecta}
	
	¿Existe sobre la Tierra una plenitud de alegría, o acaso no existe tal
	cosa?\\
	¿Existe alguna manera de hacer que la vida sea realmente digna de
	vivirse, o es imposible?\\
	Si existe esa manera, ¿Cómo es posible encontrarla?\\
	¿Qué debemos intentar hacer?\\
	¿Qué debemos intentar evitar?\\
	¿Cuál debería ser la meta en la que nuestra actividad llega a su fin?\\
	¿Qué debemos aceptar?\\
	¿Qué debemos negarnos a aceptar?\\
	¿Qué debemos amar?\\
	¿Qué debemos odiar?
	
	Lo que el mundo valora es el dinero, la reputación, la larga vida, los
	logros. Lo que considera alegría es la salud y el bienestar del cuerpo,
	la buena comida, la buena ropa, las cosas bellas de ver, la música
	agradable que escuchar.
	
	Lo que condena es la falta de dinero, un rango social bajo, la
	reputación de no valer para nada y la muerte temprana.
	
	Lo que considera desgracia es la incomodidad corporal y el trabajo. La
	falta de oportunidad de hartarse de buenas comidas, no tener ropas
	elegantes, no tener medios para entretener o deleitar los sentidos. Si
	la gente se encuentra privada de estas cosas, le entra el pánico o la
	depresión y desesperación. ¡Está tan preocupada por su vida, que su
	ansiedad se la hace insoportable, incluso cuando tiene todo lo que cree
	desear! ¡Su propia preocupación por divertirse la hace infeliz!
	
	Los ricos hacen intolerable la vida, esforzándose por conseguir cada vez
	más dinero que, en realidad, no pueden usar. Al hacer esto, quedan
	alienados de sí mismos y se agotan a su propio servicio, como si fueran
	esclavos de alguna otra persona.\\
	Los ambiciosos corren día y noche en persecución de honores,
	constantemente angustiados por el éxito de sus planes, temiendo el error
	de cálculo que lo puede echar todo a perder. Así, están alienados de sí
	mismos, agotando su vida real al servicio de una sombra creada por su
	insaciable esperanza.
	
	El nacimiento de un hombre es el nacimiento de su dolor.
	
	Cuando más tiempo vive, más estúpido se vuelve, porque su ansiedad por
	evitar la inevitable muerte se hace cada vez más aguda. ¡Qué amargura!
	¡Vive para algo que está siempre fuera de su alcance! Su sed de
	supervivencia en el futuro lo hace incapaz de vivir en el presente.
	
	¿Y qué hay de los líderes y los eruditos que tanto se sacrifican? Son
	honrados por el mundo, porque son hombres buenos, rectos y sacrificados.
	
	Y aun así su buen carácter no los preserva de la infelicidad, ni
	siquiera de la ruina, la desgracia y la muerte.
	
	¡Me pregunto, en este caso, si su ``bondad'' es realmente tan buena
	después de todo! ¿No será tal vez una fuente de infelicidad?\\
	Supongamos que admitimos que son felices, ¿Pero es acaso algo alegre
	tener un carácter y una carrera que llevan finalmente a la propia
	destrucción? Por otra parte, ¿puede llamárselos ``infelices'', si al
	sacrificarse salvan las vidas y fortunas de otros?\\
	Tomemos el caso del ministro que, consciente y rectamente, se opone a
	una decisión injusta de su rey. Algunos dicen: ``Di la verdad y, si el
	rey se niega a escuchar, déjalo que haga lo que quiera. Ya no tienes
	mayor compromiso''. Por otra parte, Shu Tse siguió oponiéndose a la
	injusta política de su soberano. Fue, por consiguiente, destruido. Pero
	si no se hubiera alzado por lo que consideraba correcto, su nombre no
	será honrado como lo es. De forma que ésta es la cuestión: ¿Habrá de
	considerarse ``bueno'' el camino que siguió si, al mismo tiempo, le fue
	fatal?
	
	No puedo decir si lo que las personas consideran ``felicidad'' es
	felicidad o no. Lo único que sé es que, cuando considero la manera en
	que buscan conseguirla, los veo bajos de cabeza, adustos y obsesionados
	por la marea general del rebaño humano, incapaces de detenerse o de
	cambiar de dirección. Continuamente afirman estar a punto de alcanzar
	`la felicidad'.
	
	Por lo que a mí respecta, no puede aceptar sus parámetros, ya sean de
	felicidad o de infelicidad. Me pregunto si, después de todo, su concepto
	de la felicidad tiene realmente algún significado.
	
	Mi opinión es que nunca se encuentra la felicidad hasta que se deja de
	buscarla. Mi mayor felicidad consiste precisamente en no hacer
	absolutamente nada pensado para obtener la felicidad; ¡y éste, según el
	criterio de la mayor parte de la gente, es el peor de todos los caminos
	posibles!
	
	Me remito al dicho de que: ``La alegría perfecta es carecer de ella. La
	alabanza perfecta es carecer de alabanzas''.
	
	Si preguntáis ``qué hacer'' y ``qué no debe hacerse'' sobre la tierra
	para obtener la felicidad, yo contesto que estas preguntas no tienen
	respuesta. No hay forma de determinar tales cosas.
	
	Y, aun así, al mismo tiempo, si dejo de buscar la felicidad, el ``buen
	camino'' y el ``mal camino'' resultan inmediatamente evidentes por sí
	mismos.
	
	El contento y el bienestar se hacen posibles al instante en que se deja
	de actuar con ellos en la mente; y, si se practica el no-hacer (\textit{wu
		wei}), se consigue tanto la felicidad como el bienestar,
	automáticamente.
	
	He aquí cómo resumo todo esto:
	
	El Cielo no hace nada: su no-hacer es su serenidad.\\
	La Tierra no hace nada: su no-hacer es su reposo.\\
	De la unión de estos dos no-haceres,\\
	proceden todos los actos,\\
	se componen todas las cosas.\\
	¡Cuán vasto, qué invisible\\
	este llegar-a-ser!\\
	¡Todas las cosas van a ninguna parte!\\
	¡Cuán vasto, qué invisible...\\
	no hay forma de explicarlo!\\
	Todos los seres en su perfección\\
	nacen del no-hacer.\\
	Es por esto por lo que se dice:\\
	``El Cielo y la Tierra no hacen nada,\\
	y aun así no hay nada que no hagan''.
	
	¿Dónde estará el hombre capaz de alcanzar\\
	este no-hacer?
	
	¡Él es ese con que quiero hablar!
	
	Cuando la vida era plena, no había historia
	
	En la era en que la vida sobre la Tierra era plena, nadie prestaba
	particular atención a los hombres valiosos, ni señalaba al hombre de
	habilidad. Los gobernantes eran simplemente las ramas más altas del
	árbol, y el pueblo era como los ciervos en los bosques. Eran honestos y
	justos, sin darse cuenta de que estaban ``cumpliendo con su deber''. Se
	amaban los unos a los otros, y no sabían que esto significaba ``amar al
	prójimo''. No engañaban a nadie y aun así no sabían que eran hombres de
	``fiar''. Eran íntegros y no sabían que aquello era ``buena fe''. Vivían
	juntos libremente, dando y tomando, y no sabían que eran ``generosos''.
	Por esta razón, sus hechos no han sido narrados. No hicieron historia.
	
	\chapter*{Los cincos enemigos}
	
	Con madera de un árbol de cien años de edad,\\
	construyen vasos para el sacrificio,\\
	cubiertos de diseños verdes y amarillos.\\
	Las astillas cortadas\\
	yacen si ser utilizables en la cuneta.\\
	Si comparamos los vasos de sacrificio con la madera de la cuneta,\\
	vemos que difieren en apariencia:\\
	uno es más bello que la otra;\\
	pero aun así son iguales en esto: ambos han perdido su naturaleza
	original.\\
	De modo que, si comparamos al ladrón con el ciudadano respetable,\\
	vemos que uno es, desde luego, más respetable que el otro;\\
	y aun así coinciden en esto: ambos han perdido\\
	la simplicidad original del hombre.
	
	¿Cómo la perdieron? He aquí las cinco maneras:\\
	El amor a los colores atonta el ojo\\
	y ya no consigue ver correctamente.\\
	El amor a las armonías hechiza el oído\\
	y se pierde el verdadero oído.\\
	El amor a los perfumes\\
	llena la cabeza de vahídos.\\
	El amor a los sabores\\
	arruina el gusto.\\
	Los deseos desazonan el corazón\\
	hasta que la naturaleza original enloquece.
	
	Estos cinco son los enemigos de la verdadera vida.\\
	Y aun así son aquello para lo que ``hombres de gran discernimiento''
	afirman que viven.\\
	No es aquello para lo que yo vivo:\\
	¡si esto es la vida, entonces, los palomos enjaulados\\
	han encontrado la felicidad!
	
	\chapter*{Ayuno del corazón}
	
	Yen Hui, el discípulo favorito de Confucio, apareció para despedirse de
	su Maestro.
	
	``¿Dónde vas?'', preguntó Confucio.
	
	``Voy a Wei''.
	
	``¿Y para qué?''
	
	``He oído que el príncipe de Wei es un individuo autoritario, sensual y
	totalmente egoísta. No se preocupa en absoluto de su gente y se niega a
	admitir cualquier defecto en su persona. No presta la más mínima
	atención al hecho de que sus súbditos mueren por doquier. Todo el campo
	está lleno de cadáveres como heno en un prado. El pueblo está
	desesperado. Pero yo le he oído decir, Maestro, que se debe abandonar el
	estado bien gobernado e ir al que esté sumido en el desorden. A las
	puertas del médico hay abundantes enfermos. Deseo aprovechar esta
	oportunidad para poner en práctica lo que he aprendido de usted y ver si
	puedo lograr alguna mejora de las condiciones de aquel lugar''.
	
	``¡Ay!'', dijo Confucio, ``no te das cuenta de lo que haces. Atraerás el
	desastre sobre tu cabeza. El Tao no necesita de tus anhelos y sólo
	lograrás desperdiciar tus energías con tus mal encaminados esfuerzos. Al
	desperdiciar tus energías, te encontrarás confuso y después ansioso. Una
	vez que te invada la ansiedad, ya no serás capaz de ayudarte a ti mismo.
	Los antiguos sabios empezaban por buscar el Tao en ellos mismos, después
	miraban a ver si encontraban en los demás algo que se correspondiera al
	Tao, tal como ellos lo conocían. Pero si tú mismo no tienes el Tao, ¿qué
	ganas tú desperdiciando el tiempo en vanos esfuerzos por llevar al
	camino correcto a unos políticos corruptos? No obstante, supongo que has
	de tener alguna base para tus esperanzas de éxito. ¿Cómo te propones
	conseguirlo?''
	
	Yen Hui respondió: ``Pretendo presentarme como un hombre humilde y
	desinteresado, que sólo busca hacer lo que está bien y nada más: un
	planteamiento sencillo y honesto. ¿Ganaré con esto su confianza''\\
	``Por supuesto que no'', replicó Confucio. ``Ese hombre está convencido
	de que sólo él está en lo cierto. Podrá fingir ante el público que se
	toma interés en un patrón objetivo de justicia, pero no te dejes engañar
	por ello. Él no está acostumbrado a que nadie se le oponga. Su método es
	confirmarse a sí mismo que está en lo cierto pisoteando al resto de la
	gente. Si esto lo hace con hombres mediocres, con más seguridad aún lo
	hará con alguien que representa una amenaza para él al afirmar que es un
	hombre de grandes cualidades. Él se aferrará tozudamente a su método.\\
	Podrá fingir que está interesado en tus palabras acerca de lo que es
	objetivamente bueno, pero en su interior no te oirá y no lograrás cambio
	alguno. No llegarás a ninguna parte de esta manera''.
	
	Yen Hui dijo entonces: ``Muy bien. En lugar de oponerme a él
	directamente, mantendré mis propios valores interiormente, pero
	exteriormente fingiré ceder. Apelaré a la autoridad de la tradición y a
	los ejemplos del pasado. Aquel que interiormente se niega a aceptar
	compromisos es tan hijo del Cielo como cualquier gobernante. No me
	apoyaré en ninguna enseñanza propia y, por tanto, no tendré preocupación
	alguna sobre si se aprueba mi conducta o no. Finalmente seré aceptado
	como una persona desinteresada y sincera. Todos llegarán a apreciar mi
	candor y así seré un instrumento del Cielo en medio de ellos.
	
	De esta manera, cediendo obedientemente ante el príncipe como hacen
	otros hombres, inclinándome, arrodillándome, postrándome como cualquier
	sirviente debe hacer, seré aceptado como limpio de culpa. Así, otros
	tendrán confianza en mí y gradualmente empezarán a usarme, viendo que
	tan sólo deseo hacerme útil y trabajar para el bien de todos. Seré así
	un instrumento de los hombres.\\
	Mientras tanto, todo lo que tenga que decir será expresado en términos
	de la antigua tradición. Trabajaré con la sagrada tradición de los
	sabios de la antigüedad. Aunque lo que diga pueda ser objetivamente una
	condena de la conducta del príncipe, no seré yo el que la pronuncie,
	sino la propia tradición. De esta forma, seré perfectamente honesto sin
	ser ofensivo. Así, seré un instrumento de la tradición. ¿Cree usted que
	es ésta la forma correcta de abordar la cuestión?''
	
	``Desde luego que no'', dijo Confucio. ``¡Tienes demasiados planes de
	acción, mientras que ni siquiera has conocido al príncipe u observado su
	carácter! En el mejor de los casos, tal vez puedas librarte y salvar tu
	pellejo, pero no conseguirás cambiar absolutamente nada. Tal vez él se
	adapte superficialmente a tus palabras, pero no existirá un cambio real
	en su actitud''.
	
	Yen Hui dijo entonces: ``Está bien, esto es todo lo que se me ocurre.
	¿Querría usted, Maestro, ¿decirme qué sugiere?''
	
	``¡Debes ayunar!'', dijo Confucio. ``¿Sabes a qué me refiero cuando
	hablo de ayunar? No es fácil. Pero los caminos fáciles no provienen de
	Dios''.
	
	``¡Oh!'', dijo Yen Hui. ``¡Estoy acostumbrado al ayuno! En casa éramos
	pobres. Pasábamos meses sin ver carne o vino. Eso es ayuno, ¿no es
	así?''
	
	``Bueno, puedes llamarlo `observar un ayuno', si quieres'', dijo
	Confucio, ``pero no es el ayuno del corazón''.
	
	``Dígame'', dijo Yen Hui. ``¿Qué es el ayuno de corazón?'' Confucio
	respondió: ``El objetivo del ayuno es la unidad interior. Esto significa
	oír, pero no con los oídos; oír, pero no con el entendimiento; oír con
	el espíritu, con todo tu ser. Oír sólo con los oídos es una cosa. Oír
	con el entendimiento es otra. Pero oír con el espíritu no se ve limitado
	a una facultad u otra, al oído o a la mente. Por tanto, exige el vacío
	de todas las facultades. Y cuando las facultades quedan vacías, la
	totalidad del ser escucha. Se da entonces una captación directa de
	aquello que está frente a ti y que no puede ser escuchado con el oído o
	comprendido por la mente. El ayuno del corazón vacía las facultades, te
	libera de las limitaciones y de las preocupaciones. El ayuno del corazón
	da a luz la unidad y la libertad''.
	
	``Ya veo'', dijo Yen Hui. ``Lo que obstruía mi camino era mi propia
	conciencia de mí mismo. Si consigo empezar el ayuno del corazón, esta
	conciencia de mí mismo desaparecerá. ¡Entonces me veré libre de
	limitaciones y preocupaciones! ¿Es eso lo que quiere decir?''
	
	``Sí'', dijo Confucio, ``¡eso es! Si eres capaz de hacerlo, quedarás
	capacitado para ir al mundo de los hombres sin afectarlos. No entrarás
	en conflicto con su propia imagen ideal de sí mismos. Si están
	dispuestos a escuchar, cántales una canción. Si no, mantente en
	silencio. No intentes echar abajo sus puertas. No pruebes nuevas
	medicinas con ellos. Limítate a estar entre ellos, porque no tienes otra
	misión que ser uno de ellos. ¡Entonces podrás tener éxito!
	
	Es fácil mantenerse quieto y no dejar rastro, pero es difícil andar sin
	tocar la tierra. Si sigues los métodos humanos, podrás engañar y aun
	salir bien librado. En el camino del Tao, el engaño es imposible.
	
	Sabes que se puede volar con alas; aún no has aprendido a volar sin
	ellas. Estás familiarizado con la sabiduría de aquellos que saben, pero
	aún no conoces la sabiduría de aquellos que no saben.
	
	Observa esta ventana: no es más que un agujero en la pared, pero gracias
	a ella todo el cuarto está lleno de luz. Así, cuando las facultades
	están vacías, el corazón se llena de luz. Al estar lleno de luz, se
	convierte en una influencia por medio de la cual los demás se ven
	secretamente transformados''.
	
	\chapter*{La Huida de Lin Hui}
	
	Lin Hui de Kia emprendió la huida.\\
	Perseguido por enemigos,\\
	tiró todos los preciosos símbolos\\
	de jade de su rango\\
	y se echó a la espalda a su hijo pequeño.\\
	¿Por qué cogió al niño\\
	abandonando el jade\\
	que valía una pequeña fortuna,\\
	mientras que el niño, de venderlo,\\
	sólo le proporcionaría una suma miserable?
	
	Lin Hui dijo:\\
	``Mi atadura al jade,\\
	El símbolo de mi cargo,\\
	era la atadura del egoísmo.\\
	Mi atadura al niño\\
	era la atadura del Tao.
	
	Allí donde el egoísmo es la atadura,\\
	huye la amistad\\
	cuando la calamidad llega.\\
	Allí donde el Tao es la atadura,\\
	la amistad se hace perfecta\\
	por medio de la calamidad.
	
	La amistad de los hombres sabios\\
	es insípida como el agua.\\
	La amistad de los tontos\\
	es dulce como el vino.\\
	Pero la insipidez de los sabios\\
	trae consigo un afecto verdadero,\\
	y el sabor de la compañía de los tontos\\
	acaba convirtiéndose en odio''.
	
	\chapter*{El bote vacío}
	
	Quien gobierna sobre los otros vive en la confusión;\\
	Quien es gobernado por los otros vive en la tristeza.\\
	Por tanto, Yao deseaba no influir en nadie\\
	ni ser influenciado por nadie.\\
	El camino para apartarse de la confusión y quedar libre de la tristeza\\
	es vivir en el Tao: el país del gran Vacío.
	
	Si un hombre está cruzando un río,\\
	y un bote vacío choca con su esquife,\\
	por muy mal genio que tenga no se enfadará demasiado.\\
	Pero si ve en el bote a un hombre,\\
	le gritará que se aparte.\\
	Si sus gritos no son escuchados, volverá a gritar,\\
	una y otra vez, y empezará a maldecir.\\
	Y todo porque hay alguien en el bote.\\
	No obstante, si el bote estuviera vacío,\\
	no estaría gritando, ni estaría tan irritado.
	
	Si uno puede vaciar el propio bote,\\
	mientras cruzando el río del mundo,\\
	nadie se le opondrá.\\
	Nadie intentará hacerle daño.
	
	El árbol derecho y alto es el primero en ser talado.\\
	El arroyo de aguas claras es el primero en ser agotado y seco.\\
	Si deseas engrandecer tu sabiduría y avergonzar al ignorante,\\
	cultivar tu carácter y ser más brillante que los demás,\\
	¡una luz brillará en torno a ti\\
	como si te hubieras tragado ambos el Sol y la Luna!\\
	No podrás evitar las calamidades.
	
	Un hombre sabio ha dicho:\\
	``Aquel que está orgulloso consigo mismo\\
	ha realizado un trabajo carente de valor.\\
	El éxito es el principio del fracaso.\\
	La fama es el origen de la desgracia''.
	
	¿Quién puede liberarse del éxito y de la fama,\\
	descender y perderse entre las masas de los hombres?\\
	Fluirá como el Tao, libre, inadvertido.\\
	Se moverá con la propia vida\\
	sin nombre, sin hogar.\\
	Él es simple, sin distinciones.\\
	¡Según todas las apariencias, es un tonto!\\
	Sus pasos no dejan huella.\\
	No tiene poder alguno.\\
	No logra nada, carece de reputación.\\
	Ya que no juzga a nadie, nadie lo juzga.\\
	Así es el hombre perfecto:\\
	su bote está vacío.
	
	\chapter*{Plenitud}
	
	``¿Cómo puede el verdadero hombre de Tao\\
	atravesar paredes sin obstáculos,\\
	mantenerse en medio del fuego sin quemarse?''
	
	No a causa de su astucia\\
	o su audacia;\\
	no porque haya aprendido\\
	sino porque ha desaprendido.
	
	Todo aquello que está limitado por medio de la forma, aspecto, color,
	sonido,\\
	es llamado objeto.\\
	De entre todos ellos, tan sólo el hombre\\
	es más que un objeto.\\
	Aunque, como los objetos, tiene forma y aspecto,\\
	no se ve limitado a la forma. Es más.\\
	Puede lograr ser sin-forma.
	
	Cuando está más allá de la forma y el aspecto,\\
	más allá de ``esto'' y de ``aquello'',\\
	¿Dónde está el punto de vista de un objeto?\\
	¿Dónde está el contraste?\\
	¿Qué puede obstruir su camino?
	
	Reposará en su lugar eterno,\\
	que es el no-lugar.\\
	Estará escondido\\
	en su propio e insondable secreto.\\
	Su naturaleza profundiza hasta la raíz\\
	en el Uno.\\
	Su vitalidad, su poder\\
	se esconden en el Tao secreto.\\
	Cuando es todo uno,\\
	no hay falla en él por la cual pueda entrar una cuña.
	
	Igualmente, un hombre borracho, al caer de un carro,\\
	queda contusionado, pero no destruido.\\
	Sus huesos son como los huesos de los demás hombres,\\
	pero su caída es diferente.\\
	Su espíritu está completo.\\
	No es consciente\\
	de su cuerpo subiendo a un carro,\\
	ni del carro del que se ha caído.
	
	La vida y la muerte no significan nada para él.\\
	Desconoce la alarma, se encuentra con los obstáculos\\
	sin pensar, sin preocupaciones,\\
	los enfrenta sin saber que están ahí.
	
	Si existe tal seguridad en el vino,\\
	cuánta más habrá en el Tao.\\
	El hombre sabio está escondido en el Tao,\\
	nada puede tocarlo.
	
	\chapter*{Grande y pequeño}
	
	Cuando observamos las cosas a la luz del Tao,\\
	nada es lo mejor, nada es lo peor.\\
	Cada cosa, vista bajo su propia luz,\\
	destaca a su manera.\\
	Puede parecer ``mejor''\\
	de lo que se compara con ella\\
	en sus propios términos.\\
	Pero en términos de la totalidad,\\
	nada destaca como ``lo mejor''.\\
	Si medimos las diferencias,\\
	lo que es más grande que otra cosa es ``grande''.\\
	Por lo tanto, no hay nada que no sea ``lo grande''.\\
	Lo que es más pequeño que otra cosa es ``pequeño''.\\
	Por lo tanto, no hay nada que no sea ``lo pequeño''.
	
	Así que todo el cosmos es un grano de arroz,\\
	y la punta de un cabello\\
	es grande como una montaña...\\
	Éste es el punto de vista relativo.\\
	Se pueden derribar muros con arietes,\\
	pero no se pueden tapar agujeros con ellos.\\
	Todas las cosas tienen diferentes usos.\\
	Los buenos caballos pueden hacer cien millas al día,\\
	pero no pueden cazar ratones.\\
	Como los perritos o las comadrejas:\\
	todas las criaturas tienen dones que les son propios.\\
	El búho de cuerno blanco puede cazar pulgas a medianoche\\
	y distinguir la punta de un cabello,\\
	pero en pleno día se queda pasmado, impotente,\\
	y no puede ver ni siquiera una montaña.\\
	Todas las cosas tienen distintas capacidades.
	
	En consecuencia: aquel que desea el bien sin el mal,\\
	el orden sin el desorden,\\
	no comprende los principios\\
	del Cielo y la Tierra.\\
	No sabe cómo\\
	están vinculadas las cosas.\\
	¿Puede un hombre aferrarse nada más que al Cielo\\
	y olvidarse de la Tierra?\\
	Son correlativos: el conocer el uno\\
	es conocer la otra.\\
	El renegar de uno\\
	es renegar de ambos.
	
	¿Puede un hombre aferrarse a lo positivo\\
	sin nada negativo\\
	en contraste con lo cual se ve\\
	que es positivo?\\
	Si afirma poder hacerlo,\\
	es un bellaco o un loco.
	
	Los tronos pasan\\
	de dinastía a dinastía,\\
	ora hacia acá, ora hacia allá.\\
	Aquel que llega al poder por la fuerza,\\
	en contra de la corriente,\\
	es llamado tirano y usurpador.\\
	Aquel que se mueve con la corriente de los acontecimientos,\\
	es llamado sabio estadista.
	
	Kui, el dragón de una sola pata,\\
	tiene envidia del ciempiés.\\
	El ciempiés tiene envidia de la serpiente.\\
	La serpiente tiene envidia del ojo.\\
	El ojo tiene envidia de la mente.\\
	Kui le dijo al ciempiés:\\
	``Controlo mi única pata con dificultad.\\
	¿cómo puedes controlar tú un centenar?''\\
	El ciempiés replicó:\\
	``Yo no las controlo.\\
	Caen por todas partes\\
	como gotas de un escupitajo''.\\
	El ciempiés dijo a la serpiente:\\
	``A pesar de todos los pies que tengo, no consigo moverme tan
	rápidamente\\
	como tú lo haces sin tenerlos.\\
	¿Cómo puede ser?''\\
	La serpiente replicó:\\
	``Tengo un deslizamiento natural\\
	que no puede ser cambiado. ¿Para qué quiero yo pies?''\\
	La serpiente habló con el viento:\\
	``Yo ondeo mi columna dorsal y me muevo\\
	de una manera física. Tú, sin huesos,\\
	sin músculos, sin método,\\
	soplas desde el Mar del Norte hasta el Océano del Sur.\\
	¿Cómo consigues llegar hasta allí sin tener nada?''\\
	El viento respondió:\\
	``Cierto, surjo del Mar del Norte\\
	y llego sin obstáculos hasta el Océano del Sur.
	
	Pero cada ojo que me observa,\\
	cada ala que me utiliza,\\
	es superior a mí, a pesar de que\\
	yo puedo arrancar los más grandes árboles, o derribar\\
	grandes edificios.
	
	El verdadero conquistador es aquel\\
	que no es conquistado\\
	por la multitud de lo pequeño.\\
	Este conquistador es la mente.\\
	Pero sólo la mente\\
	del hombre sabio''.
	
	\chapter*{Hombre nace en el Tao}
	
	Los peces nacen en el agua,\\
	el hombre nace en el Tao.\\
	Si los peces, nacidos en el agua,\\
	buscan la sombra profunda\\
	del estanque o la alberca,\\
	todas sus necesidades\\
	son satisfechas.\\
	Si el hombre, nacido en el Tao,\\
	se hunde en la profunda sombra\\
	de la no-acción,\\
	para olvidar la agresión y las preocupaciones,\\
	no le falta nada,\\
	su vida es segura.
	
	Moraleja: ``Todo lo que necesita el pez\\
	es perderse en el agua.\\
	Todo lo que necesita el hombre es perderse\\
	en el Tao''.
	
	\chapter*{La acción y la no-acción}
	
	La no-acción del hombre sabio no es inacción.\\
	No es nada estudiado. No se ve alterada por nada.\\
	El sabio está quieto porque no se ve que se mueva,\\
	no porque quiere estar tranquilo.\\
	El agua quieta es como el cristal.\\
	Puedes mirarte en ella y ver la barba de tu mentón.\\
	Es un nivel perfecto;\\
	podría usarlo el carpintero.\\
	Si el agua es tan clara, tan nivelada,\\
	¿cuánto más lo será el espíritu del hombre?\\
	El corazón del hombre sabio es sereno.\\
	Es el espejo del Cielo y la Tierra,\\
	el cristal de todo.\\
	Vaciedad, quietud, tranquilidad, insipidez.\\
	Silencio, no-acción: éste es el nivel del Cielo y la Tierra.\\
	Esto es el Tao perfecto. Los hombres sabios encuentran aquí\\
	su lugar de reposo.\\
	En reposo, están vacíos.
	
	Del vacío viene lo no condicionado.\\
	De esto, lo condicionado, las cosas individuales.\\
	De modo que, del vacío del sabio, surge la quietud;\\
	De la quietud, la acción.\\
	De la acción, el logro.\\
	De su quietud viene su no-acción, que es también acción.\\
	Y es, por tanto, su logro.\\
	Porque la quietud es el goce.\\
	El goce está libre de preocupación,\\
	fructífero durante largos años.\\
	El gozo vuelve despreocupadas todas las cosas\\
	porque el vacío, la quietud, la tranquilidad, la insipidez,\\
	el silencio y la no-acción\\
	son la raíz de todas las cosas.
	
	\chapter*{Vida activa}
	
	¡Si un experto no tiene algún problema que lo preocupe, no está feliz!\\
	¡Si las enseñanzas de un filósofo nunca son atacadas, languidece!\\
	¡Si los críticos no tienen en quién verter su veneno, se sienten
	infelices!\\
	Toda esta gente es prisionera del mundo de los objetos.
	
	El que busca seguidores persigue el poder político.\\
	El que busca reputación tiene un cargo.\\
	El hombre fuerte busca pesos que levantar.\\
	El hombre valiente busca alguna emergencia en la que poder mostrar su
	bravura.\\
	El espadachín desea una batalla en la que pueda blandir su espada.\\
	Los hombres maduros prefieren un retiro digno\\
	en el cual puedan aparentar ser profundos.\\
	Los hombres experimentados en las leyes\\
	buscan casos difíciles en los que extender la aplicación de las leyes.\\
	Los litúrgicos y los músicos gustan de festivales\\
	en los que exhiben sus ceremoniosos talentosos.\\
	Los benevolentes, los dedicados, siempre andan a la búsqueda de
	oportunidades\\
	para manifestar su virtud.\\
	¿Dónde estaría el jardinero si ya no hubiera hierbajos?\\
	¿Qué sería de los negocios si no hubiera un mercado de tontos?\\
	¿Dónde estarían las multitudes si no hubiera pretexto\\
	para apelotonarse y hacer ruido?\\
	¿Qué sería del trabajo si no hubiera objetos superfluos que hacer?\\
	¡Producid! ¡Obtened resultados! ¡Ganad dinero!\\
	¡Haced amigos! ¡Haced cambios!\\
	¡O moriréis de desesperación!
	
	Aquellos que se ven atrapados por la maquinaria del poder no disfrutan
	más que la actividad y el cambio, ¡el zumbido de la máquina! Siempre que
	se presenta una ocasión de actuar, se ven compelidos a hacerlo; no
	pueden remediarlo. Se ven movidos inexorablemente, como la máquina de la
	que forman parte. ¡Prisioneros en el mundo de los objetos, no tienen más
	elección que someterse a las exigencias de la materia! Se ven
	presionados y aplastados por fuerzas externas, la moda, el mercado, los
	sucesos, la opinión pública. ¡Jamás, en el transcurso de su vida,
	consiguen recuperar el sano juicio! ¡La vida activa! ¡Qué lástima!
	
	\chapter*{Dejar las cosas como están}
	
	Sé lo que es dejar el mundo tranquilo, no interferir. No sé nada acerca
	de cómo dirigir las cosas. Dejar las cosas como están ¡de manera que los
	hombres no hagan hinchar su naturaleza hasta que pierde su forma! ¡No
	interferir, para que los hombres no se vean transformados en algo que no
	son! Cuando los hombres no se vean retorcidos y mutilados más allá de
	toda posibilidad de ser reconocidos, cuando se les permita vivir, habrá
	sido logrado el propósito del gobierno.
	
	¿Demasiado placer? El Yang tiene demasiada influencia. ¿Demasiado
	sufrimiento? El Yin tiene demasiada influencia. Cuando uno de éstos se
	impone al otro, es como si las estaciones llegaran cuando no deben. El
	equilibrio entre el frío y el calor queda destruido, el cuerpo del
	hombre sufre.\\
	Demasiada alegría, demasiada tristeza, fuera del momento preciso, y los
	hombres pierden el equilibrio. ¿Qué harán después? El pensamiento divaga
	sin control. Empiezan a hacer de todo, no terminan nada. Aquí comienza
	la competencia, aquí nace la idea de la excelencia, y los ladrones
	surgen sobre la faz de la Tierra.
	
	Ahora, ni el mundo entero es recompensa suficiente para los ``buenos''
	ni hay castigo suficiente para los ``malvados''. Desde ahora, el mundo
	entero no es suficientemente grande ni como premio ni como castigo.
	Desde los tiempos de las Tres Dinastías, los hombres han estado
	corriendo en todas las direcciones imaginables. ¿Cómo van a encontrar
	tiempo para ser humanos?
	
	Entrenas tus ojos y tu visión anhela colores. Educas tus oídos y deseas
	sonidos deliciosos. Te deleitas en hacer el bien y tu bondad natural
	queda deformada. Te regocijas en ser justo y te vuelves más allá de toda
	razón. Te excedes en la liturgia y te conviertes en un comicastro.
	Excede te en tu amor por la música y sólo interpretarás basura. El amor
	a la sabiduría lleva a una sabiduría prefabricada. El amor al
	conocimiento lleva a la búsqueda de fallas. Si los hombres se
	mantuvieran como realmente son, tener o prescindir de estas ocho
	delicias no significaría nada para ellos. Pero si se niegan a permanecer
	en su estado correcto, las ocho delicias se desarrollan como tumores
	malignos. El mundo cae en la confusión. Ya que los hombres alaban estas
	delicias, y las anhelan, el mundo ha quedado ciego como una piedra.
	
	Cuando el deleite haya pasado, aún se aferrarán a él: rodean su memoria
	de adoraciones rituales, caen de hinojos para hablar de él, tocan música
	y cantan, ayunan y se auto disciplinan en honor de las ocho delicias.
	Cuando las delicias se convierten en una religión, ¿cómo puede uno
	controlarlas?
	
	El hombre sabio, entonces, cuando ha de gobernar, sabe cómo no hacer
	nada. Al dejar las cosas estar, descansa en su naturaleza original.
	Aquel que gobierne respetará al gobernado ni más ni menos que en la
	medida en que se respete a sí mismo. Si ama su propia persona lo
	suficiente como para dejarla descansar en su verdad original, gobernará
	a los demás sin hacerles daño. Dejadlo que evite que los profundos
	impulsos de sus entrañas entren en acción. Dejadlo estar tranquilo, sin
	mirar, sin oír. Dejadlo estar sentado como un cadáver, con el poder del
	dragón vivo en torno de sí. En completo silencio, su voz será como el
	trueno. Sus movimientos serán invisibles, como los de un espíritu, pero
	los poderes del Cielo irán con ellos. Inalterado, sin hacer nada, verá
	todas las cosas madurar a su alrededor. ¿De dónde sacará tiempo para
	gobernar?
	
	\chapter*{Huida de la sombra}
	
	Había un hombre que se alteraba tanto al ver a su propia sombra y se
	disgustaba tanto con sus propios pasos, que tomó la determinación de
	librarse de ambos. El método que se le ocurrió fue huir de ellos.
	
	Así que se levantó y echó a correr. ¡Pero cada vez que bajaba el pie
	había otro paso, mientras que su sombra se mantenía a su altura sin
	dificultad alguna!
	
	Atribuyó su fracaso al hecho de que no estaba corriendo con la
	suficiente rapidez. De modo que empezó a correr más y más rápido, sin
	detenerse, hasta que finalmente cató muerto.
	
	\chapter*{Violentando cajas fuertes}
	
	Como garantía contra los ladrones que roban bolsas, desvalijan equipajes
	y revientan cajas fuertes, uno debe asegurar todas las propiedades
	concuerdas, cerrarlas con candados, acerrojar las con cerrojos. Esto
	(para los propietarios) es del más elemental sentido común.
	
	Pero cuando aparece un ladrón fuerte, se lleva todo, se lo echa a la
	espalda y sigue su camino, con un solo temor: que cedan las cuerdas,
	candados y cerrojos. Así, lo que el mundo llama buen negocio no es más
	que una forma de amasar un botín, empaquetarlo y asegurarlo, formando
	una carga cómoda para los ladrones más audaces.
	
	¿Quién hay, entre los llamados inteligentes, que no desperdicie su
	tiempo amasando un botín para un ladrón mayor que él?
	
	En la tierra de Khi, de pueblo a pueblo, se podía oír el canto de los
	gallos, el ladrido de los perros. Los pescadores lanzaban sus redes, los
	campesinos araban los anchos campos, todo estaba pulcramente señalado
	con líneas de demarcación. En quinientas millas cuadradas había templos
	para los antepasados, altares para los dioses de los campos y espíritus
	del grano. Cada cantón, condado y distrito era gobernado con arregla las
	leyes y estatutos... Hasta que una mañana el fiscal general, Tien Khang
	Tse, liquidó al rey y se apoderó de todo el Estado.
	
	¿Quedó acaso conforme con robar la tierra? No, se apoderó también de las
	leyes y de los estatutos, y con ellos de todos los abogados, por no
	mencionar a la policía. Todos formaban parte del mismo paquete.
	
	Por supuesto, la gente llamaba ladrón a Khang Tse, pero lo dejaban
	tranquilo viviendo tan feliz como los Patriarcas. Ningún pequeño Estado
	levantaba la voz contra él, ningún gran Estado hizo el más mínimo
	movimiento en su contra. Así que durante doce generaciones el estado de
	Khi perteneció a su familia. Nadie interfirió sus derechos inalienables.
	
	El invento de los pesos y medidas hace más fácil el robo. La firma de
	contratos, la implantación de sellos, hacen más seguro el robo. Enseñar
	amor y obligaciones suministra un lenguaje adecuado con el cual
	demostrar que el robo es en realidad para el bien de todos. Un hombre
	pobre ha de ser ahorcado, por robar una hebilla de cinturón, pero si un
	hombre rico roba todo un Estado se aclamado como el estadista del año.
	
	De modo que, si queréis escuchar los mejores discursos sobre el amor, el
	deber, la justicia, etc., escuchad a los hombres de Estado. Pero cuando
	el arroyo se seca, nada crece en el valle. Cuando el montículo se
	aplana, el hueco junto a él se llena. Y cuando los hombres de Estado y
	los abogados y los predicadores del deber desaparecen, no hay tampoco
	más robos y el mundo queda en paz.
	
	Moraleja: cuanto más acumules principios éticos y deberes y
	obligaciones, para meter en cintura a todo el mundo, más botín acumulas
	para los ladrones como Khang. Por medio de argumentos éticos y
	principios morales, se demuestra finalmente que los mayores crímenes
	eran necesarios, y que, de hecho, fueron un señalado beneficio para la
	humanidad.
	
	\chapter*{Huida de la benevolencia}
	
	Hsu yu se encontró con un amigo al abandonar la capital, en la carretera
	principal, en dirección a la frontera más cercana.
	
	``¿Dónde vas?'', le preguntó el amigo.
	
	``Dejo al Rey Yao. Está tan obsesionado con las ideas de la justicia y
	la benevolencia, que temo que al final ocurra algo ridículo. En
	cualquier caso, sea divertido o no, este tipo de cosas terminan con las
	personas devorándose crudas las unas a las otras.
	
	De momento, hay una gran oleada de solidaridad. El pueblo cree que es
	amado y responde con entusiasmo. Están todos apoyando al Rey, porque
	piensan que los está haciendo ricos. Las alabanzas no cuestan dinero, y
	están todos compitiendo a ver qué obtiene más favores. Pero pronto
	habrán de aceptar algo que no les guste y todo se vendrá abajo.
	
	Cuando la justicia y la benevolencia flotan en el aire, unas cuantas
	personas están realmente preocupadas por el bienestar de los demás, pero
	la mayoría son conscientes de que es un buen momento, maduro para ser
	explotado. Sacan partido de la situación. Para ellos, la benevolencia y
	la justicia son trampas para cazar pájaros. Así, la benevolencia y la
	justicia quedan rápidamente asociadas al fraude y la hipocresía.
	Entonces todo el mundo empieza a dudar. Y es entonces cuando realmente
	empiezan los problemas.
	
	El rey Yao sabe hasta qué punto benefician a la nación los funcionarios
	probos y rectos, pero no sabe el daño que proviene de su rectitud: son
	un frente tras el cual los sinvergüenzas operan con más seguridad. Pero
	hay que ver esta situación con objetividad para darse cuenta.
	
	Hay tres clases de personas para considerar: hombres del sí, las
	sanguijuelas y los que manipulan.
	
	Los hombres de sí adoptan la línea de algún líder político y repiten sus
	afirmaciones de memoria, imaginándose que saben algo, confiados en que
	van a alguna parte y completamente satisfechos del sonido de sus propias
	voces. Son unos completos estúpidos. Y, dado que son estúpidos, se
	someten de esta manera a la manera de hablar de otro hombre.
	
	Las sanguijuelas son como parásitos sobre una cerda. Se apelotonan allá
	donde las cerdas son escasas, y este lugar se convierte en parque y
	palacio. Se deleitan con las grietas, entre los dedos de las cerdas, en
	torno a las articulaciones y las ubres, o debajo del rabo. Allí se hacen
	fuertes y se imaginan que no podrán ser expulsados por ningún poder del
	mundo. Pero no se dan cuenta de que un día llegará el carnicero con
	cuchillo y oscilante hoz. Recogerá paja seca y le pegará fuego para
	quemar las cerdas y abrazar a los parásitos. Tales parásitos mueren
	cuando la cerda es sacrificada.
	
	Los que manipulan son hombres como Shun.
	
	La carne de carnero no se siente atraída por las hormigas, pero las
	hormigas se sienten atraídas por la carne del carnero porque es
	maloliente y rancia. Así, Shun era un operador vigoroso y con éxito, y a
	la gente le gustaba por eso. Tres veces se desplazó de ciudad en ciudad,
	y cada vez su nueva casa se convertía en capital. Finalmente se mudó a
	la selva, y hubo cien mil familias que se mudaron con él para colonizar
	el lugar.
	
	Finalmente, Yao propuso la idea de que Shun debería irse al desierto a
	ver qué partido podía sacar de aquello. Aunque por aquel entonces Shun
	era ya un hombre viejo y su mente se iba debilitando, no podía negarse.
	No fue capaz de retirarse. Había olvidado cómo detener su carro. Era un
	operador, ¡y nada más!
	
	El hombre de espíritu, por otra parte, detesta ver que la gente se reúne
	a su alrededor. Evita a la multitud. Porque allí donde hay muchos
	hombres, existen también muchas opiniones y pocos acuerdos. No se puede
	ganar nada de un montón de medio idiotas que están condenados a acabar
	peleando el uno contra el otro.
	
	El hombre de espíritu no es muy íntimo de nadie ni demasiado distante.
	Se mantiene interiormente consciente, y conserva su equilibrio de tal
	forma que no está en conflicto con nadie. ¡Éste es tu hombre verdadero!
	Él deja que las hormigas sean listas. Él deja que el carnero apeste de
	actividad. Por su parte, imita al pez que nada indiferente, rodeado de
	un elemento amigo y ocupándose de sus asuntos.
	
	Los hombres de verdad ven lo que ve el ojo y no le añaden nada que no
	esté ahí. Oyen lo que oyen sus oídos y no detecta sobretonos
	imaginarios. Comprende las cosas en su interpretación obvia y no se
	ocupa de ocultos significados y misterios. Su curse, por tanto, sigue el
	camino principal. Y, no obstante, está capacitado para cambiar de
	dirección en cuanto las circunstancias así lo aconsejen.
	
	\chapter*{Keng San Chu}
	
	El Maestro Keng San Chu, discípulo de Lao Tse, se hizo famoso por su
	sabiduría, y la gente de Wei-Lei comenzó a venerarlo como a un sabio. El
	esquivó sus homenajes y rechazó sus regalos. Se mantuvo escondido y no
	les permitía ir a verlo. Sus discípulos discutieron con él y dijeron
	que, desde los tiempos de Yao y Shun, era tradicional que los hombres
	sabios aceptaran la veneración, ejerciendo así una buena influencia.
	
	El Maestro Keng replicó:
	
	Venid aquí, hijos míos, escuchad esto.\\
	Si una bestia lo suficientemente grande para tragarse un carro\\
	abandonará su bosque de la montaña,\\
	jamás escaparía a la trampa del cazador.\\
	Si un pez lo suficientemente grande como\\
	para tragarse un bote\\
	deja que la marea baja lo deje varado en la arena,\\
	entonces hasta las hormigas podrán destruirlo.\\
	Así que las aves vuelan por las alturas, las bestias permanecen\\
	en soledades sin caminos,\\
	se mantienen ocultas de la vista; y los peces\\
	y las tortugas se sumergen\\
	hasta el mismo fondo.\\
	El hombre que tiene algo de respeto por su persona\\
	mantiene su carcasa alejada de la vista,\\
	se esconde tan perfectamente como puede.\\
	En cuanto a Yao y Shun: ¿Por qué alabar a tales reyes?\\
	¿Qué bien hizo su moralidad?\\
	Hicieron un agujero en la pared\\
	y lo dejaron llenarse de zarzas.\\
	Numeraban los pelos de tu cabeza\\
	antes de peinarlos.\\
	Contaban cada grano de arroz\\
	antes de cocinar su cena.\\
	¿Qué bien le hicieron al mundo\\
	con sus escrupulosas distinciones?\\
	Si los virtuosos son honrados,\\
	el mundo se llenará de envidias.\\
	Si el hombre inteligente es premiado,\\
	el mundo se llenará de ladrones.\\
	No puede hacer buenos y honestos a los hombres\\
	alabando la virtud y el conocimiento.\\
	Desde los días del piadoso Yao y el virtuoso Shun,\\
	todo el mundo ha estado intentando hacerse rico:\\
	un hijo es capaz de matar a su padre por dinero;\\
	un ministro, de matar a su soberano\\
	para satisfacer su ambición.\\
	A plena luz del día se roban los unos a los otros,\\
	a medianoche derriban paredes:\\
	la semilla de todo esto fue plantada\\
	en tiempos de Yao y Shun.\\
	Sus ramas crecerán durante un millar de eras\\
	y de aquí a mil eras\\
	¡los hombres se estarán comiendo crudos los unos a los otros!''
	
	\chapter*{El hombre verdadero}
	
	¿Qué se quiere decir con ``el hombre verdadero''?\\
	Los hombres verdaderos de antaño no tenían miedo.\\
	cuando se encontraban solos en sus puntos de vista.\\
	Nada de grandes logros. Nada de planes.\\
	Si fracasaban, nada de dolor.\\
	Nada de autocomplacencia en caso de éxito.\\
	Escalaban farallones, siempre sin vértigo;\\
	se sumergían en las aguas, jamás se mojaban,\\
	caminaban a través del fuego y no se quemaban.\\
	Así su conocimiento llegaba hasta el Tao.
	
	Los hombres verdaderos de antaño\\
	dormían sin sueños,\\
	despertaban sin preocupaciones.\\
	Su comida era sencilla.\\
	Respiraban profundamente.\\
	Los hombres verdaderos respiran desde sus talones.\\
	Otros respiran con sus gargantas,\\
	medio estrangulados.\\
	En las disputas\\
	arrojan argumentos\\
	como si vomitaran.
	
	Donde las fuentes de las pasiones\\
	yacen profundas,\\
	los arroyos celestiales pronto se secan.\\
	Los hombres verdaderos de antaño\\
	no conocían la pasión por la vida,\\
	ni el miedo a la muerte.\\
	Su aparición carecía de alegría,\\
	su salida, más allá,\\
	se producía sin resistencia.\\
	Fácil viene, fácil se va.\\
	No olvidaban de dónde,\\
	ni preguntaban a dónde,\\
	ni caminaban inflexiblemente hacia adelante\\
	luchando a todo lo largo de su vida.\\
	Tomaban la vida como venía, sin preocupación;\\
	y se iban, allá. ¡Allá!
	
	No tenían intención de combatir el Tao.\\
	No intentaban, por sus cuentas, ayudar al Tao.\\
	Ésos son los que llamamos hombres verdaderos.
	
	Mentes libres, pensamientos desaparecidos.\\
	Frentes despejadas, rostros serenos.\\
	¿Eran frescos? No más frescos que el otoño.\\
	¿Eran cálidos? No más que la primavera.\\
	Todo lo que salía de ellos\\
	salía tranquilamente, como las cuatro estaciones.
	
	\chapter*{El hombre soberano}
	
	Mi Maestro dijo:
	
	``Aquello que actúa sobre todo y no interfiere con nada, es el cielo. El
	hombre soberano se da cuenta de esto, lo oculta en su corazón, crece sin
	límite, con amplia mentalidad, lo atrae todo a sí.
	
	Y así deja al oro yacer oculto en la montaña, deja la perla descansando
	en las profundidades. Los bienes y las propiedades no suponen ganancia
	alguna ante sus ojos, se mantiene alejado de la riqueza y los honores.
	
	Una larga vida no es motivo de regocijo, ni una muerte temprana de pena.
	
	El éxito es algo de lo que no tiene por qué enorgullecerse, el fracaso
	no es una vergüenza.
	
	Si tuviera todo el poder del mundo, no lo consideraría como propio; si
	lo conquistara todo, no se lo apropiaría.
	
	Su gloria está en saber que todas las cosas se funden en Una, y que la
	vida y la muerte son iguales''.
	
	\chapter*{El duque Hwan y el carretero}
	
	El mundo valora los libros, y piensa que haciendo esto está valorando el
	Tao. Pero los libros no contienen más que palabras. Aun así, hay algo
	más que da valor a los libros. No sólo las palabras ni el pensamiento
	contenido en las palabras, sino algo más contenido en el pensamiento,
	inclinando lo en cierta dirección que las palabras no pueden aprehender.
	Pero son las palabras en sí mismas lo que valora el mundo cuando las
	introduce en los libros; y aunque el mundo las valore, estas palabras
	carecen de valor mientras aquello que se lo da no sea honrado.
	
	Aquello que un hombre aprehende por medio de la observación no es más
	que la forma y los colores exteriores, el nombre y el sonido, y cree que
	esto lo pondrá en posesión del Tao. Forma y color, nombre y sonido, no
	alcanzan a reflejar la realidad. Por eso: ``Aquel que sabe no dice,
	aquel que dice no sabe''.
	
	¿Cómo va el mundo a conocer, entonces, el Tao por medio de las palabras?
	
	\begin{quote}
		El duque Hwan de Khi, el primero de su dinastía,\\
		estaba sentado bajo su toldilla leyendo filosofía.\\
		Phien el carretero estaba en el patio haciendo una rueda.\\
		Phien dejó a un lado el martillo y el cincel,\\
		ascendió los escalones, y dijo al duque Hwan:\\
		``¿Puedo preguntarle, Señor, ¿qué es eso que usted está leyendo?''\\
		El Duque dijo: ``A los expertos. Las autoridades''.\\
		Y Phien preguntó: ``¿Vivos o muertos?''\\
		``Muertos hace mucho tiempo''.\\
		``Entonces'', dijo el carretero,\\
		``no está leyendo más que la basura que dejaron atrás''.\\
		Entonces el Duque replicó: ``¿Qué sabes tú de esto?\\
		No eres más que un carretero.\\
		Mas te vale darme una buena explicación o morirás''.\\
		El carretero dijo: ``Veamos el asunto desde mi punto de vista.\\
		Cuando yo hago ruedas, si me lo tomo con calma, se deshacen;\\
		si soy demasiado violento, no encajan;\\
		si no soy ni demasiado calmoso ni demasiado violento, sale bien.\\
		El trabajo resulta como yo deseo.\\
		Esto no puede ser traducido a palabras:\\
		simplemente hay que saber cómo es.\\
		Ni siquiera puedo explicar a mi hijo cómo hacerlo,\\
		y mi propio hijo no puede aprenderlo de mí.\\
		¡Así que aquí estoy, con mis setenta años, haciendo ruedas todavía!\\
		Los hombres de antaño\\
		se llevaron todo lo que realmente sabían con ellos a la tumba.\\
		Y así, mi Señor, lo que está leyendo ahí\\
		no es más que la basura que dejaron tras de ellos''.
	\end{quote}
	
	\chapter*{Destazando un buey}
	
	El cocinero del príncipe Wen Hui\\
	estaba destazando un buey.\\
	Extendió una mano,\\
	bajó un hombro,\\
	apoyó un pie,\\
	presionó con una rodilla.\\
	El buey quedó deshecho.\\
	Con un susurro,\\
	el brillante cuchillo de carnicero murmuraba\\
	como un viento suave.\\
	¡Ritmo! ¡Cronometraje!\\
	¡Como una danza sagrada,\\
	como las antiguas armonías!
	
	``¡Buen trabajo!'', exclamó el príncipe.\\
	``¡Su método es impecable!''\\
	``¿Método?'', dijo el cocinero\\
	dejando a un lado su cuchilla.\\
	``¡Lo que hago es seguir el Tao\\
	más allá de todo método!
	
	Cuando empecé a\\
	destazar bueyes,\\
	veía ante mí\\
	al buey entero,\\
	toda una masa única.\\
	Después de tres años,\\
	ya no veía aquella masa.\\
	Veía sus distinciones.\\
	Pero ahora ya no veo nada\\
	con los ojos. Todo mi ser aprehende.\\
	Mis sentidos están ociosos. El espíritu,\\
	libre para trabajar sin un plan concreto,\\
	sigue su propio instinto\\
	guiado por una línea natural.\\
	Por la abertura secreta, el espacio oculto,\\
	mi cuchilla no encuentra su propio camino.\\
	No atravieso ninguna articulación, no corto hueso alguno.
	
	Un buen cocinero necesita cortador nuevo,\\
	una vez al año. Corta.\\
	Un mal cocinero necesita uno nuevo\\
	todos los meses. ¡Él mutila!
	
	Llevo utilizando esta misma hoja\\
	diecinueve años.\\
	Ha destazado\\
	un millar de bueyes.\\
	Su hoja sigue cortando\\
	como si estuviera recién afilada.
	
	Hay espacios entre las articulaciones;\\
	la hoja es delgada y cortante:\\
	cuando esta delgadez\\
	encuentra aquel espacio,\\
	¡hay todo el sitio que se pudiera desear!\\
	¡Pasa como una brisa!\\
	¡Por eso mantengo esta hoja desde hace diecinueve años\\
	como si estuviera recién afilada!
	
	Cierto es, en ocasiones hay\\
	articulaciones duras. Las siento venir,\\
	entonces me detengo, observo con atención,\\
	me contengo, casi no muevo la hoja,\\
	y ¡whump! la parte se desprende\\
	cayendo como un trozo de tierra.
	
	Entonces retiro la hoja,\\
	me quedo quieto,\\
	y dejo que la alegría del trabajo\\
	penetre en mí.\\
	Limpio la hoja\\
	y la guardo''.
	
	El príncipe Wan Hui dijo:\\
	``¡Eso es! ¡Mi cocinero me ha mostrado\\
	como debiera vivir\\
	mi propia vida!
	
	\chapter*{El tallador de madera}
	
	Khing, el maestro tallador, hizo un soporte de campana\\
	con maderas preciosas. Cuando lo hubo terminado,\\
	todos aquellos que lo veían quedaban asombrados.\\
	Decían que tenía que ser trabajo de los espíritus.\\
	El Príncipe de Lu preguntó al maestro tallador:\\
	``¿Cuál es tu secreto?''
	
	Khing replicó: ``Yo no soy más que un trabajador:\\
	carezco de secretos. Sólo hay esto:\\
	cuando empecé a pensar en el trabajo que usted ordenó,\\
	conservé mi espíritu, no lo malgasté\\
	en minucias que no tuvieran nada que ver con él.\\
	Ayuné para dejar sereno mi corazón.\\
	Después de tres días de ayuno,\\
	me había olvidado de las ganancias y el éxito.\\
	A los cinco días,\\
	había olvidado los halagos y las críticas.\\
	Al cabo de siete días,\\
	había olvidado mi cuerpo\\
	con todas sus extremidades.
	
	A estas alturas, todo pensamiento acerca de su Alteza\\
	y la corte se habían desvanecido.\\
	Todo aquello que pudiera distraerme de mi trabajo\\
	había desaparecido.\\
	Estaba concentrado en el único pensamiento\\
	del soporte para la campana.
	
	Después fui al bosque\\
	para ver los árboles en su propio estado natural.\\
	Cuando ante mis ojos apareció el árbol adecuado,\\
	también apareció sobre él el soporte, claramente, más allá de toda
	duda.\\
	Todo lo que tuve que hacer fue alarga la mano\\
	y empezar.
	
	Si no me hubiera encontrado con este árbol en particular,\\
	no hubiera habido soporte para la campana.
	
	¿Qué pasó?\\
	Mi ser concentrado\\
	se encontró con el potencial oculto en la madera.\\
	De este encuentro vital surgió el trabajo,\\
	que usted atribuye a los espíritus''.
	
	\chapter*{La necesidad de vencer}
	
	Cuando un arquero dispara porque sí,\\
	está en posesión de toda su habilidad.\\
	Si está disparando por ganar una hebilla de bronce,\\
	ya está nervioso.\\
	Si el premio es de oro,\\
	se ciega\\
	o ve dos blancos...\\
	¡Ha perdido la cabeza!
	
	Su habilidad no ha variado. Pero el premio\\
	lo divide. Está preocupado.\\
	Piensa más en vencer\\
	que en disparar...\\
	Y la necesidad de ganar\\
	le quita poder.
	
	\chapter*{El gallo de pelea}
	
	Chi Hsing Tse era un entrenador de gallos de pelea\\
	empleado por el rey Hsuan.\\
	Estaba entrenando un ave magnífica.\\
	El rey no hacía más que preguntar si el ave estaba\\
	preparada para combatir.\\
	``Aún no'', dijo el entrenador.\\
	``Está llena de fuego,\\
	dispuesta a pelear\\
	con cualquier otra ave. Es vanidosa y confía\\
	en su propia fuerza''.\\
	Diez días más tarde, contestó de nuevo:\\
	``Aún no. Explota\\
	en cuanto oye cantar a otra ave''.\\
	Diez días más tarde:\\
	``Aún no. Todavía se le pone\\
	ese gesto iracundo\\
	e hincha las plumas''.\\
	De nuevo, diez días,\\
	el entrenador dijo: ``Ahora ya está casi listo.\\
	Cuando canta otro gallo, sus ojos\\
	ni siquiera parpadean,\\
	Se mantiene inmóvil\\
	como un gallo de madera.\\
	Es un luchador maduro.\\
	Las demás aves\\
	lo mirarán una sola vez\\
	y echarán correr''.
	
	\chapter*{La tortuga}
	
	Chuang Tse, con su caña de bambú,\\
	pescaba en el río Pu.
	
	El príncipe de Chu\\
	mandó a dos vicecancilleres\\
	con un documento oficial:\\
	``Por la presente queda usted nombrado\\
	primer ministro''.
	
	Chuang Tse cogió su caña de bambú.\\
	Observando aún el río Pu,\\
	dijo:\\
	``Tengo entendido que hay una tortuga sagrada,\\
	ofrecida y canonizada\\
	hace tres mil años.\\
	que es venerada por el príncipe,\\
	envuelta en sedas,\\
	en un precioso relicario\\
	sobre un altar,\\
	en el Templo.
	
	¿Qué creen ustedes:\\
	es acaso mejor otorgar la propia vida\\
	y dejar atrás una concha sagrada\\
	como objeto de culto\\
	en una nube de incienso\\
	durante tres mil años,\\
	o será mejor vivir\\
	como una tortuga vulgar\\
	arrastrando su rabo por el cieno?''
	
	``Para la tortuga'', dijo el vicecanciller,\\
	``será mejor vivir\\
	y arrastrar la cola por el cieno''.
	
	``¡Váyanse a casa!'', dijo Chuang Tse.\\
	``¡Déjenme aquí\\
	para arrastrar mi cola por el cieno!''
	
	\chapter*{El búho y el fénix}
	
	Hui Tse era el primer ministro de Liang. Estaba en posesión de
	información que creía de buena fuente, de que Chuang Tse aspiraba a su
	puesto y estaba intrigando para suplantarlo. De hecho, cuando Chuang Tse
	fue a visitar a Liang, el primer ministro mandó a la policía para
	prenderlo. La policía lo anduvo buscando tres días y tres noches, pero
	mientras tanto Chuang se presentó ante Hui Tse por su propia cuenta y
	dijo:
	
	¿Has oído hablar del ave\\
	que vive en el sur,\\
	el fénix que jamás envejece?
	
	Este fénix inmortal\\
	surge del Mar del Sur\\
	y vuela hasta el Mar del Norte\\
	sin posarse jamás,\\
	excepto en ciertos árboles sagrados.\\
	Jamás prueba bocado\\
	salvo la más exquisita\\
	fruta exótica.\\
	Tan sólo bebe\\
	de los más límpidos arroyos.
	
	Una vez, un búho\\
	que roía una rata muerta,\\
	ya medio podrida,\\
	vio al fénix volar sobre él,\\
	miró hacia lo alto,\\
	y chilló alarmado,\\
	aferrándose a la rata\\
	aterrado y con angustia.
	
	¿Por qué te aferras tan frenéticamente\\
	a tu ministerio\\
	y me chillas\\
	con tanta angustia?''
	
	\chapter*{Un vendedor de sombreros y un gobernante capaz}
	
	Un hombre de Sung negociaba\\
	con sombreros ceremoniales de seda.\\
	Viajó con una carga de sombreros\\
	hacia donde vivían los salvajes hombres del Sur.\\
	Los hombres salvajes tenían las cabezas afeitadas,\\
	cuerpos cubiertos de tatuajes.\\
	¿Para qué podían querer sombreros ceremoniales de seda?
	
	Yao había gobernado sabiamente toda China.\\
	Había llevado al mundo entero\\
	a un estado de sosiego.\\
	Después de esto, fue a visitar\\
	a los cuatro Hombres Perfectos\\
	a las distintas montañas\\
	de Ku Shih.\\
	Cuando volvió,\\
	al cruzar la frontera\\
	y entrar en su propia ciudad,\\
	su mirada perdida\\
	no vio trono alguno.
	
	\chapter*{Aconsejando al Príncipe}
	
	El ermitaño Hsu Su Kwei había ido a ver al Príncipe Wu.\\
	El Príncipe se alegró. ``He estado deseando verte'', dijo, ``durante
	mucho tiempo.\\
	Dime si estoy en lo correcto.\\
	Quiero amar a mi pueblo y, a través del ejercicio de la justicia,\\
	poner fin a la guerra. ¿Es esto suficiente?''
	
	``Ni mucho menos'', dijo el ermitaño.\\
	``Su `amor' hacia su pueblo lo pone en un peligro mortal.\\
	¡Su ejercicio de la justicia es la raíz de una guerra tras otra!\\
	¡Sus grandes intenciones acabarán en el desastre!
	
	Si se propone `lograr algo grande',\\
	sólo se está engañando a sí mismo.\\
	Su amor y su justicia son fraudulentos.\\
	Son meros pretextos\\
	para su autoafirmación, para la agresión.\\
	Una acción traerá consigo otra\\
	y, en la cadena de los acontecimientos,\\
	sus ocultas intenciones\\
	quedarán al descubierto.
	
	Usted afirma practicar la justicia. En el caso de que aparentemente
	tenga éxito,\\
	ese éxito será portador de nuevos conflictos.\\
	¿Por qué todos estos guardias\\
	Vigilan las puertas del palacio,\\
	alrededor del altar del templo,\\
	por todas partes?
	
	¡Está usted en guerra consigo mismo!\\
	Usted no cree en la justicia,\\
	sólo en el poder y el éxito.\\
	Si derrota a un enemigo, si se anexiona su país,\\
	quedará aún menos en paz\\
	con usted mismo de lo que está ahora.\\
	Tampoco le permitirán sus pasiones\\
	quedarse quieto. ¡Luchará\\
	continuamente por\\
	un más perfecto ejercicio de la `justicia'!
	
	Abandone su plan\\
	de ser un `amante y equitativo gobernante'.\\
	Intente responder\\
	a las exigencias de la verdad interior.\\
	¡Deje de humillarse a sí mismo y a su pueblo\\
	con estas obsesiones!\\
	Su pueblo respirará al fin tranquilo.\\
	¡Vivirá,\\
	y la guerra terminará por sí misma!''
	
	\chapter*{La importancia de no tener dientes}
	
	Nieh Ch'ueh, que no tenía dientes,\\
	fue a ver a P'i y le pidió una lección sobre el Tao.\\
	(¡Tal vez eso sí pudiera masticarlo!)
	
	De modo que P'i comenzó:\\
	``En primer lugar, logra el control del cuerpo\\
	controla la mente. Alcanza\\
	la idea fija. Entonces\\
	la armonía de los Cielos\\
	descenderá y morará en ti.\\
	Reposarás en el Tao.\\
	Tendrás el aspecto simple\\
	de un ternero recién nacido.\\
	Oh, afortunado de ti,\\
	ni siquiera sabrás la causa\\
	de tu estado...
	
	Pero mucho antes de que P'i llegara a este punto de su sermón, el
	desdentado se había quedado dormido. Su mente era simplemente incapaz de
	masticar la sustancia de las enseñanzas. Pero P'i quedó satisfecho. Echó
	a andar cantando:
	
	``Su cuerpo está enjuto y seco\\
	como un hueso viejo;\\
	su mente está muerta\\
	como las cenizas consumidas.\\
	¡Su sabiendo es sólido,\\
	su sabiduría es cierta!\\
	En una noche profunda y oscura,\\
	vaga con libertad,\\
	sin objetivos\\
	y sin designios;\\
	¿Quién puede compararse\\
	con este hombre sin dientes?''
	
	\chapter*{El hombre con un solo pie y el faisán del pantano}
	
	Kung Wen Hsien vio a un oficial mutilado,\\
	cuyo pie izquierdo le había sido amputado.\\
	¡Una penalización del juego político!
	
	``¿Qué clase de hombre'', exclamó, ``es esa\\
	extraña cosa con un solo pie?\\
	¿Cómo ha llegado a esto? ¿Habremos de\\
	decir que fue el hombre\\
	el que hizo esto, o que fue el Cielo?''.
	
	``El Cielo'', dijo, ``esto viene del\\
	Cielo, no del hombre.\\
	Cuando el Cielo le dio vida a este hombre,\\
	quiso que se distinguiera de los demás\\
	y lo introdujo en la política,\\
	para que así se hiciera famoso.\\
	¡Observen! ¡Un solo pie! ¡Este hombre es diferente!
	
	El pequeño faisán del pantano necesita dar diez saltos\\
	para conseguir un bocado de grano.\\
	Ha de correr cien pasos\\
	antes de poder tomar un sorbo de agua.\\
	Y a pesar de todo no pide\\
	que se lo mantenga en un corral,\\
	aunque así podría tener todo lo que pudiera desear,\\
	ante sus pies.
	
	Antes prefiere correr\\
	y buscarse su propia y pequeña subsistencia,\\
	libre de jaulas''.
	
	\chapter*{Las tres de la madrugada}
	
	Cuando desgastamos nuestras mentes, aferrándonos tozudamente a una
	visión parcial de las cosas, negándonos a ver un más profundo acuerdo
	entre éste y su opuesto complementario, sufrimos lo que se llama ``las
	tres de la madrugada''.
	
	¿Qué es esto de ``las tres de la madrugada''?
	
	Un domador de monos fue a ver a sus monos y les dijo:
	
	``Con respecto a lo de vuestras castañas: vais a recibir tres medidas
	por la mañana y cuatro por la tarde''.
	
	Ante esto, todos se enfadaron. De modo que dijo: ``Está bien, en este
	caso os daré cuatro por la mañana y tres por la tarde''. En esta ocasión
	quedaron satisfechos.
	
	Ambas soluciones eran lo mismo, en tanto en que el número de castañas no
	variaba. Pero, en un caso, los animales quedaban descontentos y en el
	otro satisfechos. El guarda estuvo dispuesto a cambiar sus planes para
	hacer frente a las condiciones objetivas. ¡No perdió nada al hacerlo!
	
	El hombre verdaderamente sabio, considerando ambos lados de una cuestión
	sin parcialidad, ve ambos a la luz del Tao.
	
	Esto se llama seguir dos cursos a la vez.
	
	\chapter*{El pivote}
	
	El Tao se ve oscurecido cuando los hombres comprenden tan sólo uno de un
	par de opuestos, o se concentran tan sólo en un aspecto parcial del ser.
	Entonces, la expresión clara se ve también enturbiada por meros juegos
	de palabras, al afirmar un aspecto cualquiera y negar todo el resto.
	
	De aquí las disputas entre los confucianos y los mohístas; cada uno
	niega lo que el otro afirma, y afirma lo que el otro niega. ¿Qué
	utilidad tiene esta lucha por oponer el ``¿No'' al ``¿Sí'', y el ``Sí''
	al ``No''? Es mejor abandonar tan desesperado esfuerzo y buscar la
	verdadera luz.
	
	No hay nada que no pueda observarse desde el punto de vista del
	``No-Yo''. Y no hay nada que no pueda ser visto desde el punto de vista
	del ``Yo''. Si comienzo observando cualquier cosa desde el punto de
	vista del ``No-Yo'', entonces no la veo realmente, dado que es ``No-Yo''
	el que la ve. Si empiezo a partir de donde estoy y la veo como yo la
	veo, entonces también puede ser posible que pueda llegar a verla como la
	ve otro.
	
	De aquí la teoría de la inversión, de que los opuestos se producen el
	uno al otro, dependen en el uno del otro y se complementan el uno al
	otro.
	
	Sea como sea, la vida viene seguida de la muerte; la muerte viene
	seguida por la vida. Lo posible se convierte en imposible; lo imposible
	se convierte en posible. El bien se convierte en mal y el mal en bien;
	el flujo de la vida altera las circunstancias y, así, las propias cosas
	se ven alteradas a su vez. Pero los disputantes continúan afirmando y
	negando las mismas cosas que siempre han afirmado y negado, ignorando
	los nuevos aspectos de la realidad presentados por el cambio de las
	condiciones.
	
	El hombre sabio, por tanto, en lugar de tratar de demostrar esto o
	aquello por medio de disputas lógicas, ve todas las cosas a la luz de la
	intuición. No se ve apresado por las limitaciones del ``Yo'', dado que
	el punto de vista de la intuición directa es, a la vez, el del ``Yo'' y
	el del ``No-Yo''. Por tanto, ve que a ambos lados de cada argumento
	existen tanto la verdad como el error. Ve también que al final son
	reducibles a la misma cosa, una vez que han sido relacionados entre sí
	por medio del pivote del Tao.
	
	Cuando el hombre sabio se sustenta en este pivote, está en el centro del
	círculo y ahí se mantiene mientras el ``Sí'' y el ``No'' se persiguen en
	torno a la circunferencia.
	
	El pivote del Tao pasa a través del centro, donde convergen todas las
	afirmaciones y negaciones. Aquel que abraza el pivote está en el punto
	fijo desde el cual todos los movimientos y oposiciones pueden ser vistos
	a la luz de su correcta relación. Por tanto, ve las ilimitadas
	posibilidades tanto del ``Sí'' como del ``No''. Abandonando toda idea de
	imponer límites o de tomar partido, descansa en la intuición directa.
	Por esto dije: ``¡Mejor será abandonar la disputa y buscar la verdadera
	luz!''
	
	\chapter*{La alegría de peces}
	
	Chuang Tse y Hui Tse estaban cruzando\\
	el río Hao junto a la presa.
	
	Chuang dijo:\\
	``Fíjate qué libremente saltan y corren los peces;\\
	eso es su felicidad''.
	
	Hui replicó:\\
	``Ya que tú no eres un pez,\\
	¿cómo sabes\\
	qué es lo que hace felices a los peces?''
	
	Chuang dijo:\\
	``Dado que tú no eres yo,\\
	¿cómo es posible que puedas saber\\
	que yo no sé\\
	qué es lo que hace felices a los peces?''
	
	Hui argumentó:\\
	``Si yo, no siendo tú,\\
	no puedo saber lo que tú sabes,\\
	es evidente que tú, no siendo pez,\\
	no puedes saber lo que ellos saben''.
	
	Pues Chuang dijo:\\
	``¡Espera un momento!\\
	¡Volvamos a tu pregunta original!\\
	Lo que tú me preguntaste fue:\\
	`¿Como puedes tú saber\\
	lo que hace felices a los peces? `\\
	Por la forma en que planteaste la cuestión,\\
	evidentemente sabes que sé\\
	lo que hace felices a los peces.
	
	Yo conozco la alegría de los peces en este río\\
	porque esa es mi propia alegría\\
	mientras camino a lo largo del mismo río''.
	
	\chapter*{La torre del Espíritu}
	
	El Espíritu tiene una torre inexpugnable\\
	a la cual no puede alterar peligro alguno,\\
	siempre y cuando la torre esté guardada\\
	por el invisible Protector\\
	que actúa inconscientemente, y cuyos actos\\
	se desvían cuando se hacen deliberados,\\
	impulsivos o tercos.
	
	La inconsciencia\\
	y total sinceridad del Tao\\
	se ven alteradas por cualquier esfuerzo\\
	de demostración de autoconsciencia.\\
	Todas esas demostraciones\\
	son mentiras.
	
	Cuando uno se exhibe\\
	de tan dividida manera,\\
	el mundo exterior entra en tromba\\
	y lo aprisiona.
	
	Ya no está protegido\\
	por la sinceridad del Tao.
	
	Cada nuevo acto\\
	es un nuevo fracaso.
	
	Si sus actos son realizados en público,\\
	a plena luz del día,\\
	será castigados\\
	por humanos.\\
	Si ellos son realizados en privado,\\
	con secretos,\\
	será castigados por espíritus.
	
	¡Qué cada cual comprenda\\
	el significado de la sinceridad\\
	y se guarde de exhibirse!
	
	Ése estará en paz\\
	con ambos los hombres y los espíritus,\\
	y actuará correctamente, sin ser visto,\\
	en su propia soledad,\\
	en la torre de su espíritu.
	
	\chapter*{La ley interior}
	
	Aquel cuya ley está dentro de sí mismo\\
	camina en el Escondido.\\
	Sus actos no se ven influenciados\\
	por aprobaciones y desaprobaciones.\\
	Aquél cuya ley está fuera de sí mismo\\
	dirige su voluntad hacia lo que está\\
	más allá de su control\\
	y busca\\
	extender su poder\\
	sobre los objetos.
	
	Aquel que camina en el Escondido\\
	tiene luz para guiarlo\\
	en todos sus actos.\\
	Aquel que busca extender su control\\
	no es más que un operador.\\
	Mientras cree que está\\
	superando a los otros,\\
	los otros lo ven tan sólo\\
	esforzarse, estirarse,\\
	para ponerse de puntillas.
	
	Cuando intenta extender su poder\\
	sobre los objetos,\\
	esos objetos ganan control\\
	sobre él.\\
	Aquel que se ve controlado por objetos\\
	pierde la posesión de su ser interior.\\
	Si ya no se valora a sí mismo,\\
	¿cómo puede valorar a otros?\\
	Si ya no valora a otros.\\
	queda abandonado.\\
	¡No le queda nada!
	
	¡No hay arma más mortífera que la voluntad!\\
	¡Ni la más afiladas de las espadas\\
	puede compararse le!\\
	No hay ladrón más peligroso\\
	que la Naturaleza (Yang y Yin).\\
	Y aun así no es la Naturaleza\\
	la causante del daño:\\
	¡es la propia voluntad del hombre!
	
	\chapter*{Disculpas}
	
	Si un hombre pisa a un desconocido\\
	en el mercado,\\
	ofrece cortésmente disculpas\\
	y una explicación:\\
	``¡Este lugar está tan enormemente\\
	lleno!''.
	
	Si un hermano mayor\\
	pisa a su hermano menor,\\
	dice: ``Lo siento''\\
	y ahí queda eso.
	
	Si un padre\\
	pisa a un hijo suyo,\\
	no se dice absolutamente nada.
	
	La mayor educación\\
	está libre de toda formalidad.\\
	La conducta perfecta\\
	está libre de preocupaciones.\\
	La sabiduría perfecta\\
	no está planificada.\\
	El amor perfecto\\
	no necesita demostraciones.\\
	La sinceridad perfecta\\
	no ofrece garantías.
	
	\chapter*{La buena suerte}
	
	El Maestro Ki tenía ocho hijos.\\
	Un día llamó a un adivino, puso en fila a los muchachos y dijo:\\
	``Estudie sus rostros. Dígame cuál es el afortunado''.
	
	Después de su examen, el viejo dijo:\\
	``Kwan es el afortunado''.
	
	Ki quedó contento y sorprendido.\\
	``¿De qué forma?'', inquirió.\\
	El adivino replicó:\\
	``Kwan comerá carne y beberá vino\\
	por el resto de sus días\\
	a cargo del gobierno''.
	
	Ki se derrumbó y sollozó:\\
	``¡Mi pobre hijo! ¡Mi pobre hijo!\\
	¿Qué ha hecho para merecer tanta desgracia?''
	
	``¡Cómo!'', exclamó el adivino.\\
	``¡Cuando uno comparte\\
	las comidas de un príncipe,\\
	las bendiciones alcanzan\\
	a toda la familia,\\
	especialmente al padre y a la madre!\\
	¿Rechazaría usted\\
	la buena suerte?''
	
	Ki dijo: ``¿Qué es lo que hace que esta suerte sea `buena'?\\
	La carne y el vino son para la boca y el estómago.\\
	¿Acaso la buena suerte está tan sólo en la boca\\
	y en el estómago?\\
	Estas `comidas del príncipe',\\
	¿cómo ha de compartirlas él?
	
	Yo no soy ningún pastor\\
	y de repente nace en mi casa una oveja.\\
	Yo no soy ningún guardián de caza\\
	y nacen codornices en mi patio.\\
	¡Son éstos terribles portentos!
	
	No tengo más deseo\\
	para mis hijos y para mí,\\
	que vagar libremente\\
	por la Tierra y los Cielos.
	
	No busco gozo alguno\\
	para ellos y para mí,\\
	más que el goce del Cielo,\\
	sencillos frutos de la Tierra.
	
	No busco ventaja alguna, no hago planes,\\
	no me introduzco en negocios.\\
	Con mis muchachos, busco tan sólo el Tao.
	
	¡Yo no he luchado contra la vida!\\
	y ahora esta espeluznante promesa\\
	de lo que nunca busqué:\\
	¡Buena suerte!
	
	Todo efecto extraño tiene alguna causa extraña.\\
	Mis hijos y yo no hemos hecho nada para merecer esto.\\
	Es un castigo inescrutable.\\
	¡Por tanto, sollozo!''
	
	Y así ocurrió, algún tiempo más tarde, que Ki mandó de viaje a su hijo
	Kwan. El joven fue capturado por bandoleras que decidieron venderlo como
	esclavo. Creyendo que no podrían venderlo tal como estaba, le cortaron
	los pies. Así, al no poder huir, resultaba un mejor negocio. Lo
	vendieron al gobierno de Chi, y fue puesto a cargo de una puerta de
	peaje en la carretera. Dispuso de vino y carne, durante el resto de sus
	días, a cargo del gobierno.
	
	¡De este modo, Kwan resultó ser el afortunado!
	
	\chapter*{Metamorfosis}
	
	Cuatro hombres entablaron una discusión. Cada uno decía:\\
	``¿Quién sabe cómo\\
	tener el Vacío por cabeza,\\
	la Vida por espina dorsal\\
	y la Muerte por rabo?\\
	¡Quien sepa cómo será mi amigo!''
	
	Con esto se miraron entre sí,\\
	vieron que estaban de acuerdo,\\
	se echaron a reír\\
	y se hicieron amigos.
	
	Entonces uno de ellos cayó enfermo,\\
	y otro fue a verlo.\\
	``¡Grande es el Creador'', dijo el enfermo,\\
	``que me ha hecho como soy!
	
	Estoy tan doblado\\
	que mis tripas están por encima de mi cabeza;\\
	reposo la mejilla\\
	sobre mi ombligo;\\
	mis hombros sobresalen\\
	por encima de mi cuello,\\
	mi coronilla es una úlcera\\
	que inspecciona el cielo;\\
	mi cuerpo es un caos\\
	pero mi mente está en orden''.
	
	Se arrastró hasta el pozo,\\
	vio su reflejo y declaró:\\
	``¡Menuda porquería\\
	ha hecho de mí!''
	
	Su amigo le preguntó:\\
	``¿Estás descorazonado?''
	
	``¡En absoluto! ¿Por qué habría de estarlo?\\
	Si Él me hace pedacitos,\\
	y con mi hombro izquierdo\\
	hace un gallo,\\
	yo anunciaré el alba.\\
	Si Él hace una ballesta\\
	de mi hombro derecho,\\
	suministraré pato asado.\\
	Si mis nalgas se convierten en ruedas\\
	y si mi espíritu es un caballo.\\
	¡me pondré yo mismo los aparejos y cabalgaré\\
	en mi propio carro!
	
	Hay un tiempo para unir\\
	y otro para deshacer.\\
	Aquel que entiende\\
	esta sucesión de hechos\\
	acepta cada nuevo estado\\
	en su momento preciso\\
	sin dolor ni regocijo.\\
	Los antiguos dijeron: `El ahorcado\\
	no puede descolgarse solo.'\\
	Pero a la larga la Naturaleza es más fuerte\\
	que todas sus cuerdas y ataduras.\\
	Siempre fue así.\\
	¿Qué razón hay\\
	para descorazonarse?
	
	\chapter*{Confucio y el loco}
	
	Cuando Confucio estaba visitando el estado de Chu\\
	apareció Kien Yu,\\
	el loco de Chu,\\
	y cantó a la puerta del Maestro:\\
	``Oh, Fénix, Fénix,\\
	¿dónde ha ido a parar tu virtud?\\
	¡No puede alcanzar el futuro\\
	ni traer de vuelta el pasado!\\
	Cuando el mundo tiene sentido,\\
	los sabios tienen trabajo.\\
	Sólo pueden esconderse\\
	cuando el mundo está patas arriba.\\
	Hoy en día, si consigues mantenerte con vida,\\
	afortunado eres:\\
	¡Intenta sobrevivir!
	
	La alegría es ligera como una pluma,\\
	Pero ¿quién puede llevarla?\\
	El dolor cae como un corrimiento de tierras,\\
	¿quién puede detenerlo?\\
	Nunca, nunca\\
	vuelvas a enseñar la virtud.\\
	Caminas en peligro.\\
	¡Cuidado! ¡Cuidado!\\
	Hasta los helechos pueden cortar tus pies.\\
	Cuando yo camino, loco,\\
	camino bien;\\
	pero ¿soy yo un hombre\\
	para imitar?''
	
	El árbol en lo alto de la montaña es su propio enemigo.\\
	La grasa que alimenta la luz se devora a sí misma.\\
	El árbol de la canela es comestible: ¡así que se lo derriba!\\
	El árbol de la laca es rentable: lo mutilan.\\
	Todo hombre sabe lo útil que es ser útil.
	
	Nadie parece saber\\
	lo útil que es ser inútil.
	
	\chapter*{El árbol inútil}
	
	Hui tse le dijo a Chuang:\\
	``Tengo un árbol grande,\\
	de los que llaman árboles apestosos.\\
	El tronco está tan retorcido,\\
	tan lleno de nudos,\\
	que nadie podría obtener una tabla derecha\\
	de su madera. Las ramas están tan retorcidas\\
	que no se pueden cortar en forma alguna\\
	que tenga sentido.
	
	Ahí está junto al camino.\\
	Ni un solo carpintero se dignaría siquiera mirarlo.
	
	Iguales son tus enseñanzas,\\
	grandes e inútiles''.
	
	Chuang Tse replicó:\\
	``¿Has observado alguna vez al gato salvaje?\\
	Agazapado, vigilando a su presa,\\
	salta en ésta y aquella dirección,\\
	arriba y abajo, y finalmente\\
	aterriza en la trampa.
	
	Pero ¿has visto al yak?\\
	Enorme como una nube de tormenta,\\
	firme en su poderío.\\
	¿Qué es grande? Desde luego.\\
	¡No puede cazar ratones!
	
	Igual ocurre con tu gran árbol. ¿Inútil?\\
	Entonces plántalo en las tierras áridas.\\
	En solitario.\\
	Pasea apaciblemente por debajo,\\
	descansa bajo su sombra;\\
	ningún hacha ni decreto preparan su fin.\\
	Nadie lo cortará jamás.
	
	¿Inútil? ¡Eres tú el que debería preocuparse!''
	
	\chapter*{Cuando el zapato se adapta}
	
	Ch'ui, el diseñador.\\
	era capaz de trazar círculos más perfectos a mano alzada\\
	que con un compás.
	
	Sus dedos hacían brotar\\
	formas espontáneas de la nada. Su mente\\
	estaba, mientras tanto, libre y sin preocupaciones\\
	acerca de lo que estaba haciendo.
	
	No le era necesario aplicarse, pero\\
	su mente era perfectamente simple\\
	y desconocía obstáculo alguno.
	
	Al igual que, cuando el zapato se adapta,\\
	se olvida el pie;\\
	cuando el cinturón se adapta,\\
	se olvida el estómago;\\
	cuando el corazón está bien,\\
	el pro y el contra se olvidan.
	
	Sin inclinaciones, sin compulsiones,\\
	sin necesidades, sin inclinaciones:\\
	entonces los asuntos de uno\\
	están ocupados se.
	
	Uno se convierte en un hombre libre.
	
	Tomárselo todo con calma es correcto,\\
	Empieza correctamente,\\
	y estarás en calma.\\
	Continúa con calma,\\
	y estarás en lo correcto.\\
	¡La manera correcta de tomárselo todo con calma\\
	es olvidarse del camino correcto\\
	y olvidarse de que seguirlo es fácil!
	
	\chapter*{La perla perdida}
	
	El Emperador Amarillo fue paseando\\
	al norte de Agua Roja,\\
	a la montaña de Kwan Lun. Miró a su alrededor\\
	desde el borde del mundo. Camino a casa,\\
	perdió su perla del color de la noche.\\
	Mandó a la Ciencia a buscar su perla, y no consiguió nada.\\
	Mandó al Análisis a buscar su perla, no consiguió nada.\\
	Mandó a la Lógica a buscar su perla, y no consiguió nada.\\
	Entonces preguntó a la Nada, ¡y la Nada la tenía!
	
	El emperador Amarillo dijo:\\
	``¡Es en verdad extraño: la Nada,\\
	que no fue mandada,\\
	que no trabajó para encontrarla,\\
	tenía la perla del color de la noche!''
	
	\chapter*{¿Dónde está el Tao?}
	
	El Maestro Tung le preguntó a Chuang:\\
	``Muéstrame dónde se encuentra el `Tao'''.\\
	Chuang Tse replicó:\\
	``No hay lugar alguno donde no se encuentre.\\
	El primero insistió:\\
	``Muéstrame al menos algún lugar concreto\\
	donde se encuentre el Tao''.\\
	``Está en la hormiga'', dijo Chuang.\\
	``Está en algún ser inferior?''\\
	``Está en los hierbajos''.\\
	``¿Puede seguir descendiendo en la escala de las cosas?''\\
	``Está en este trozo de baldosa''.\\
	``Y aún más?''\\
	``Está en este excremento''.
	
	Ante esto, Tung Kuo no tuvo nada más que decir.\\
	Pero Chuang continuó: ``Ninguna de tus preguntas\\
	es relevante. Son como las preguntas\\
	de los inspectores del mercado\\
	que comprueban el valor de los cerdos\\
	palpándoles las partes más delgadas.\\
	¿Por qué buscar el Tao bajando la ``escala del ser'\\
	como si aquello que llamamos `ínfimo'\\
	tuviera menos Tao?
	
	El Tao es grande en todas las cosas,\\
	Completo en todas, Universal en todas,\\
	Estas tres palabras son distintas,\\
	pero la Realidad es una.
	
	Por tanto, ven conmigo\\
	al palacio de Ninguna Parte\\
	donde toda la multitud de cosas son Una;\\
	donde por fin podamos hablar\\
	de lo que no tiene limitación ni final.\\
	Ven conmigo a la tierra del No-Hacer.\\
	¿Qué debemos decir allí? ¿Qué el Tao\\
	es simplicidad, quietud,\\
	indiferencia, pureza,\\
	armonía y serenidad?\\
	Todas estas descripciones no dicen nada,\\
	porque sus distinciones han desaparecido.
	
	Mi voluntad carece de utilidad allí.\\
	Si está en Ninguna Parte, ¡cómo iba a enfocarse la?\\
	Si se va y vuelve, no sé\\
	dónde ha estado descansado. Si vaga\\
	primero por aquí y luego por allá,\\
	no sé dónde irá a parar al final.
	
	La mente queda indecisa en el gran Vacío.\\
	Allí, el más alto conocimiento\\
	queda liberado. Aquello que da a las cosas\\
	su razón de ser no puede ser delimitado por las cosas.\\
	De modo que, cuando hablamos de `esto',\\
	permanecemos confinados a cosas limitadas.\\
	El límite de lo ilimitado se llama `plenitud'.\\
	La carencia de límites de lo limitado se llama vacío'.\\
	El Tao es el origen de ambos. Pero él mismo\\
	no es ni la plenitud ni el vacío.\\
	El Tao produce tanto la renovación como la descomposición,\\
	pero no es ni renovación ni descomposición.\\
	Causa el ser y el no-ser,\\
	pero no es ni ser ni no-ser.\\
	Tao une y destruye,\\
	pero no es ni la Totalidad ni el Vacío''.
	
	\chapter*{El discípulo de Keng}
	
	Un discípulo se quejó a Keng:\\
	``Los ojos de todos los hombres parecen iguales,\\
	yo no detecto en ellos diferencia alguna:\\
	y aun así algunos hombres son ciegos;\\
	sus ojos no ven.\\
	Los oídos de todos los hombres parecen ser iguales,\\
	yo no detecto en ellos diferencia alguna:\\
	y aun así algunos hombres son sordos;\\
	sus oídos no oyen.\\
	Las mentes de los hombres tienen la misma naturaleza.\\
	No detecto diferencia alguna entre ellas;\\
	pero los locos no pueden hacer suya\\
	la mente de otro hombre.\\
	Heme aquí, aparentemente como los demás discípulos,\\
	pero hay una diferencia:\\
	ellos captan el significado de lo que usted dice y lo ponen en
	práctica;\\
	yo no puedo.
	
	Usted me dice: `Mantén tu ser seguro y en calma.\\
	Mantén tu vida reunida en su propio centro.\\
	No permitas que tus pensamientos\\
	sean alterados.'\\
	Pero, por mucho que lo intente,\\
	el Tao no es más que una palabra para mis oídos.\\
	No hace resonar ninguna campana en mi interior''.
	
	Keng San replicó: ``No tengo nada más que decir.\\
	Los gallos no empollan huevos de ganso,\\
	aunque las aves de Lu sí pueden.\\
	No es tanto una diferencia de naturaleza\\
	como una diferencia de capacidad.\\
	Mi capacidad es demasiado escasa\\
	como para transformarte.\\
	¿Por qué no vas al sur\\
	a ver a Lao Tse?''
	
	El discípulo tomó algunas provisiones,\\
	viajó durante siete días y siete noches solo,\\
	y llegó ante Lao Tse,\\
	Lao le preguntó: ``¿Vienes de parte de Keng?''\\
	``Sí'', replicó el estudiante.\\
	``¿Quiénes son todas esas personas que has traído contigo?''\\
	El discípulo se volvió rápidamente para mirar.\\
	No había nadie. ¡Pánico!\\
	Lao dijo: ``¿No comprendes?''\\
	El discípulo agachó la cabeza. ¡Confusión!\\
	Después un suspiro. ``Ay de mí, he olvidado mi respuesta''.
	
	(¡Más confusión!) ``También he olvidado mi pregunta''.\\
	Lao dijo: ``¿Qué estás intentado decir?''\\
	El discípulo: ``Cuando no sé, la gente me trata como a un tonto.\\
	Cuando sé, el conocimiento me causa problemas.\\
	Cuando no logro hacer el bien, hago daño a otros.\\
	Cuando lo hago, me daño a mí mismo.\\
	Si esquivo mis deberes, soy un negligente;\\
	pero si los cumplo, me arruino.\\
	¿Cómo puedo escapar de estas contradicciones?\\
	Esto es lo que vine a preguntarle''.
	
	Lao Tse replicó:\\
	``Hace un momento,\\
	observé tus ojos.\\
	Vi que estabas agobiado\\
	por las contradicciones. Tus palabras\\
	confirman esto.\\
	Tienes un miedo mortal,\\
	como un niño que ha perdido\\
	a su padre y a su madre.\\
	Estás intentando sondear\\
	el centro del océano\\
	con una pértiga de dos metros.\\
	Te has perdido, e intentas\\
	encontrar el camino de vuelta\\
	a tu verdadero ser.\\
	No encuentras más\\
	que señales ilegibles\\
	que indican todas las direcciones.\\
	Siento pena por ti''.
	
	El discípulo solicitó ser admitido.\\
	Tomó una celda y en ella\\
	meditó,\\
	intentando cultivar cualidades\\
	que consideraba deseables,\\
	y librarse de otras\\
	que le desagradaban.\\
	¡Diez días así!\\
	¡Desesperación!
	
	``¡Miserable!'', dijo Lao\\
	¡Totalmente bloqueado!\\
	¡Hecho un nudo! ¡Intenta\\
	desatarte!\\
	Si tus obstáculos\\
	están en el exterior,\\
	no intentes agarrarlos de uno en uno\\
	y arrojarlos lejos de ti.\\
	¡Imposible! Aprende\\
	a descubrir el juego.\\
	Si están en ti mismo,\\
	no puedes destruir los,\\
	pero puedes ver\\
	que no tienen efectos en verdad.\\
	Si están tanto dentro como fuera,\\
	no intentes aferrarte al Tao.\\
	¡Limítate a tener esperanza en que el Tao\\
	te mantenga sujeto!''
	
	El discípulo gimió:\\
	``Cuando un granjero se pone enfermo\\
	y los otros granjeros vienen a verlo,\\
	si puede al menos decirles\\
	qué es lo que pasa,\\
	su enfermedad no es tan grave.\\
	Pero yo, en mi búsqueda del Tao,\\
	soy como un hombre enfermo que toma medicinas\\
	que le hacen sentirse diez veces peor.\\
	¡Dígame tan sólo\\
	los primeros elementos,\\
	así será lo suficiente!
	
	Lao Tse replicó:\\
	``¿Puedes abrazarte al Uno\\
	y no perderlo?\\
	¿Puedes intuir cosas buenas y malas\\
	sin la concha de la tortuga\\
	o los palillos?\\
	¿Puedes descansar donde hay descanso?\\
	¿Sabes cuándo detenerte?\\
	¿Eres capaz de ocuparte de tus asuntos\\
	sin preocupaciones, sin desear informes\\
	acerca del progreso de los demás?\\
	¿Eres capaz de mantenerte sólo?\\
	¿Puedes retirarte?\\
	¿Puedes ser como un niño\\
	que llora todo el día\\
	sin quedarse afónico,\\
	o crispa el puño todo el día\\
	sin que le duela la mano,\\
	o que mira todo el día\\
	sin que se canse la vista?\\
	¿Quieres los primeros elementos?\\
	El niño los posee.\\
	Libre de preocupaciones, libre de egoísmo,\\
	actúa sin reflexión.\\
	Se queda donde lo ponen, no sabe por qué,\\
	no se explica las cosas,\\
	se limita a dejarse llevar,\\
	es parte de la corriente.\\
	¡Éstos son los primeros elementos!''
	
	El discípulo preguntó:\\
	``¿Es esto la perfección?''
	
	Lao replicó: ``En absoluto.\\
	No es más que el principio.\\
	Esto es lo que rompe el hielo.
	
	Esto te capacita\\
	para desaprender,\\
	de forma que puedas ser guiado por el Tao,\\
	ser un niño del Tao.
	
	Si persistes en intentar\\
	alcanzar lo que jamás se alcanza,\\
	(es el regalo del Tao);\\
	si insistes en esforzarte\\
	por obtener lo que ningún esfuerzo puede lograr;\\
	si insistes en razonar\\
	acerca de lo que no puede ser comprendido,\\
	serás destruido\\
	por aquello que buscas.
	
	Saber cuándo detenerse,\\
	saber cuándo no puedes llegar más allá\\
	por tus propios medios,\\
	¡ésta es la forma correcta de empezar!
	
	\chapter*{Cuando el Conocimiento fue al norte}
	
	El Conocimiento vagó hacia el norte\\
	buscando al Tao, sobre el Mar Oscuro.\\
	y en lo alto de la Montaña Invisible.\\
	Allí en la montaña se encontró\\
	con el No-Hacer, el Sin-Palabras.
	
	Preguntó:\\
	``Por favor, señor, ¿me podría informar\\
	bajo qué sistema de pensamiento\\
	y qué disciplina de meditación\\
	¿Podría aprehender el Tao?\\
	¿Por medio de qué renuncia\\
	o qué solitario retiro\\
	podría reposar en el Tao?\\
	¿Dónde he de comenzar,\\
	qué camino he de seguir\\
	para alcanzar el Tao?
	
	Tales fueron sus tres preguntas.\\
	No-Hacer, el Sin-Palabras,\\
	no respondió.\\
	No sólo eso,\\
	¡ni siquiera sabía\\
	cómo responder!
	
	El Conocimiento giró hacia el sur,\\
	hacia el Mar Brillante,\\
	y ascendió la Montaña Luminosa\\
	llamada ``Fin de la Duda''.\\
	Allí se encontró con\\
	``Actúa-según-tus-impulsos'', el Inspirado Profeta,\\
	y le hizo las mismas preguntas.
	
	``Ah'', exclamó el Inspirado,\\
	``¡Tengo las respuestas, y te las revelaré!''\\
	Pero justo cuando estaba a punto de decirle todo,\\
	se volvió aturdido por otro seguidor.\\
	El Conocimiento no obtuvo respuesta alguna.
	
	De modo que el Conocimiento fue por fin\\
	al palacio del Emperador Ti,\\
	y le hizo sus preguntas a Ti.\\
	Ti replicó:\\
	``Ejercitar el no-pensamiento\\
	y seguir el no-camino de la meditación\\
	es el primer paso para empezar a comprender el Tao.\\
	No vivir en ninguna parte\\
	y no apoyarse en nada\\
	es el primer paso para descansar en el Tao.\\
	Empezar desde ninguna parte\\
	y no seguir camino alguno\\
	es el primer paso para alcanzar el Tao''.
	
	El Conocimiento respondió: ``Tú sabes esto\\
	y ahora yo también lo sé. Pero los otros dos no lo sabían.\\
	¿Qué te parece eso?\\
	¿Quién está en lo cierto?''
	
	Ti replicó:\\
	``Sólo No-Hacer, el Sin-Palabras,\\
	estaba absolutamente en lo cierto. Él no sabía lo que decir.\\
	Actúa-según-tus-impulsos, el Profeta Inspirado,\\
	sólo parecía estar en lo cierto\\
	a causa de los seguidores.\\
	En cuanto a nosotros,\\
	no estamos ni siquiera cerca de la verdad\\
	porque tenemos las respuestas''.
	
	`Aquel que sabe no habla.\\
	Aquel que habla no sabe. `\\
	Y `El Hombre Sabio instruye\\
	sin utilizar las palabras, `
	
	Esta historia llegó a los oídos de Actúa-según-tus-impulsos,\\
	que se concedió con la forma\\
	de plantearlo de Ti.
	
	Que se sepa.\\
	No-Hacer jamás oyó hablar sobre el asunto,\\
	ni hizo comentario alguno.
	
	\chapter*{La Luz de las Estrellas y el No-Ser}
	
	La Luz de las Estrellas le preguntó al No-Ser:\\
	``Maestro, ¿es usted' ¿O no es usted?
	
	Como no recibió ninguna clase de respuestas, la Luz de las Estrellas se
	dispuso a observar al No-Ser. Esperó a ver si aparecía el No-Ser.
	Mantuvo su mirada fija en el profundo vacío, con la esperanza de echar
	una mirada al No-Ser.
	
	Todo el día estuvo a la expectativa, y no vio nada. Escuchó, pero no oyó
	nada. Se extendió para tocar, y no agarró nada. Después, la Luz de las
	Estrellas exclamó al fin:
	
	``¡ESTO es!\\
	¡Es lo más distante que nunca!\\
	¿Quién podría alcanzarlo?\\
	Puedo comprender la ausencia del Ser,\\
	pero ¿quién puede comprender la ausencia de la Nada?\\
	Si ahora, encima de todo, el No-Ser Es,\\
	¿quién puede comprenderlo?''
	
	\chapter*{Dos reyes y Sin-Forma}
	
	El Rey del Mar del Sur era Actúa-según-tu-juicio.\\
	El Rey del Mar del Norte era Actúa-como-el-rayo.\\
	El Rey del lugar que había en medio era\\
	Sin-Forma.
	
	Ahora bien, el Rey del Mar del Sur\\
	y el Rey del Mar del Norte\\
	solían ir juntos, a menudo,\\
	a las tierras de Sin-Forma:\\
	los trataba muy bien.
	
	De modo que consultaron entre sí\\
	y pensaron en algo bueno,\\
	en una agradable sorpresa para Sin-Forma.\\
	como prueba de aprecio.
	
	``Los hombres'', dijeron, ``tienen siete aberturas\\
	para ver, oír, comer, respirar\\
	y demás. Pero Sin-Forma\\
	no tiene abertura alguna. Hagamos le\\
	unos cuantos agujeros''.\\
	De modo que, sin pensarlo dos veces,\\
	hicieron agujeros a Sin-Forma,\\
	uno por día, durante siete días.\\
	Y cuando terminaron el séptimo agujero,\\
	su amigo yacía muerto.
	
	Lao Tan dijo: ``Organizar es destruir''.
	
	\chapter*{Sinfonía para un ave marina}
	
	No se puede poner una carga grande en una bolsa pequeña,\\
	ni tampoco se puede, con una cuerda corta,\\
	sacar agua de un pozo profundo.\\
	No se puede hablar con un político poderoso\\
	como si fuera un hombre sabio.\\
	Si busca comprenderte,\\
	si mira dentro de sí mismo\\
	para buscar la verdad que le has dado,\\
	no consigue encontrarla.\\
	Al no encontrarla, duda.\\
	Cuando un hombre duda, matará.
	
	¿No habéis oído contar cómo un ave marina\\
	fue arrastrada tierra adentro por el viento y se posó\\
	afuera de la capital de Lu?
	
	El Príncipe ordenó una recepción solemne.\\
	Ofreció al ave marina vino en la camarera sagrada.\\
	Mandó llamar a los músicos para que interpretaran las composiciones de
	Shun.\\
	Sacrificaron vacas para darle de comer.\\
	Aturdida por el tratamiento real, la infeliz ave marina murió de
	desesperación.
	
	¿Cómo se debe tratar a un ave?\\
	¿Cómo a uno mismo o como a un ave?\\
	¿Acaso no debería un ave anidar en los bosques profundos,\\
	o volar sobre los valles y las marismas?\\
	¿Acaso no debe nadar en ríos y estanques?\\
	alimentarse de anguilas y pescado,\\
	volar en formación con otras aves marinas\\
	y descansar en los cañaverales?
	
	¡Bastante malo es para un ave marina\\
	estar rodeada de hombres y asustada por sus voces!\\
	¡Pues no fue suficiente para ellos!\\
	¡La mataron con música!
	
	Tocad todas las sinfonías que queráis\\
	en los pantanos de Thung-Ting.\\
	Las aves escaparán en todas las direcciones;\\
	los animales se esconderán;\\
	los peces bucearán hasta el fondo;\\
	pero los hombres se reunirán en torno para escuchar.\\
	El agua es para los peces\\
	y el aire para los hombres.\\
	Las naturalezas difieren, y con ellas las necesidades.
	
	Por esto los sabios de antaño\\
	no medían todo por el mismo rasero.
	
	\chapter*{El cerdo para el sacrificio}
	
	El Gran Augur, que sacrificaba cerdos y leía presagios en el sacrificio,
	apareció vestido con sus largas túnicas oscuras en la pocilga y se
	dirigió a los cerdos de la siguiente manera: ``He aquí el consejo que os
	doy. No os quejéis por tener que morir. Dejad de lado vuestras
	objeciones, por favor. Tened en cuenta que yo os alimentaré con granos
	selectos durante tres meses. Yo mismo tendré que observar una estricta
	disciplina durante diez días y ayunar tres. Después, con gran ceremonia,
	extenderé alfombras de hierba y ofreceré vuestros jamones y vuestras
	paletillas sobre fuentes, maravillosamente talladas. ¿Qué más queréis?
	
	Después, reflexionando, consideró la cuestión desde el punto de vista de
	los cerdos: ``Por supuesto, supongo que preferiríais alimentaros de
	comida grosera y ordinaria, y que os dejaran en paz para revolcarse en
	vuestras pocilgas''.
	
	Pero de nuevo, viéndolo desde su propio punto de vista, contestó: ``¡No,
	definitivamente no existe un tipo más noble de existencia! Vivir
	honrado, recibir el mejor de los tratos, montar en carroza, llevar con
	magníficos ropajes, a pesar de que en cualquier momento uno pueda caer
	en desgracia y ser ejecutado; ése es el noble, aunque incierto, destino
	que he elegido''.
	
	De modo que optó en contra del punto de vista de los cerdos y adoptó su
	propio punto de vista, tanto para él como para los cerdos.
	
	¡Qué afortunados aquellos cerdos, cuya existencia fue así ennoblecida
	por alguien que era, a la vez, una autoridad del Estado y un ministro de
	la religión!
	
	\chapter*{El aliento de la naturaleza}
	
	Cuando la gran Naturaleza suspira, oímos los vientos\\
	que, silenciosos por sí mismos,\\
	despiertan voces de otros seres,\\
	soplando sobre ellos.\\
	Desde todas las aberturas\\
	suenan fuertes voces. ¿No habéis oído nunca\\
	este ajetreo de tonos?
	
	Ahí está el bosque colgado\\
	sobre la empinada montaña:\\
	viejos árboles con agujeros y grietas\\
	como muescas para vigas, como cuencos,\\
	surcos en la madera, huecos llenos de agua;\\
	se oyen mugidos y rugidos, silbidos,\\
	voces de mando, gruñidos,\\
	profundos zumbidos, tristes flautas.\\
	Una llamada despierta a otra entablando un diálogo.\\
	Los vientos suaves cantan tímidamente,\\
	los fuertes truenan sin restricción.\\
	Entonces el viento se abate. Las aberturas\\
	emiten su último sonido.\\
	¿No habéis observado cómo entonces todo tiembla y se aquieta?
	
	Yu replicó: Comprendo.\\
	La música de la Tierra canta a través de mil orificios.\\
	La música del hombre está interpretada con flautas e instrumentos.\\
	¿Qué es lo que interpreta la música de los cielos?
	
	El maestro Ki dijo:\\
	Algo sopla sobre mil orificios diferentes.\\
	Algún poder está detrás de todo esto y hace que los sonidos se
	apaguen.\\
	¿Qué es este poder?
	
	\chapter*{¡Qué profundo es el Tao!}
	
	Mi Maestro dijo: ``¡Tao, ¡qué profundo, qué quietud la de su escondrijo!
	¡Tao, cuán puro! Sin esa quietud, el metal no reverberaría. El poder del
	sonido está en el metal y el Tao en todas las cosas. Cuando chocan,
	resuenan en el Tao y quedan de nuevo en silencio. ¿Quién podría entonces
	asignar a todas las cosas su lugar? El rey de la vida anda su camino
	libre, inactivo, desconocido. Se sonrojaría de intervenir. Él mantiene
	sus profundas raíces ancladas en el origen, abajo, en el arroyo. Su
	conocimiento está envuelto de espíritu y él se hace grande, grande, abre
	un gran corazón, un refugio para el mundo. Sin pensarlo previamente,
	sale en toda su majestad. Sin planes previos, sigue su camino y todas
	las cosas lo siguen. Éste es el hombre soberano, que cabalga por encima
	de la vida.
	
	Éste ve en la oscuridad, oye donde no hay sonido alguno. En la profunda
	oscuridad, sólo él ve luz. Sumido en el silencio, sólo él percibe
	música. Puede ir a los lugares más profundos de las profundidades y
	encontrar gente. Puede alzarse hasta lo más alto de las alturas y ver
	significado. Él está en contacto con todos los seres. Aquello que no es
	sigue su camino. Aquello que se mueve es sobre lo que él se implanta. La
	grandeza es pequeñez para él, lo largo es corto para él, y todas las
	distancias son cercanas''.
	
	\chapter*{El gran conocimiento}
	
	El gran conocimiento lo ve todo en uno.\\
	El poco conocimiento se deshace en la multiplicidad.
	
	Cuando el cuerpo duerme, el alma está envuelta en Uno.\\
	Cuando el cuerpo despierta, las aberturas empiezan a funcionar.\\
	Resuenan con cada encuentro,\\
	con todas las diversas labores de la vida, los anhelos del corazón;\\
	los hombres quedan bloqueados, perplejos, perdidos en sus dudas.\\
	Pequeños miedos corroen su paz de espíritu.\\
	Los grandes miedos los devoran por completo.\\
	Flechas disparadas contra un blanco: acierto o fallo, bien o mal.\\
	Eso es a lo que los hombres llaman juicio, decisión.\\
	Sus pronunciamientos son tan definitivos\\
	como los tratados entre emperadores.\\
	¡Oh, dejan claro su punto de vista!\\
	Pero sus argumentos caen cada vez más rápida y débilmente\\
	que las hojas muertas en otoño e invierno.\\
	Sus palabras fluyen como la orina,\\
	para jamás ser recuperadas.\\
	Finalmente quedan bloqueados, amarrados y amordazados.\\
	Taponeados como viejas tuberías de desagüe.\\
	La mente falla. Ya no volverá a ver la luz.
	
	El placer y la ira,\\
	la tristeza y la alegría,\\
	las esperanzas y los arrepentimientos,\\
	el cambio y la estabilidad,\\
	la debilidad y la decisión,\\
	la impaciencia y la haraganería:\\
	son todos sonidos de la misma flauta,\\
	todos hongos del miso moho húmedo.\\
	¡El día y la noche persiguen y caen sobre nosotros\\
	sin que veamos cómo brotan!
	
	¡Suficiente! ¡Suficiente!\\
	¡Tarde o temprano nos encontramos con ``aquello''\\
	de lo que todos ``estos'' crecen!\\
	Si no hubiera un ``aquello'',\\
	no habría un ``esto''.\\
	Si no hubiera un ``esto'',\\
	no habría instrumento para que tocaran todos estos vientos.\\
	Hasta aquí podemos llegar.\\
	Pero ¿cómo podemos comprender\\
	qué es lo que lo produce?
	
	Uno podría perfectamente suponer que el Verdadero Gobernante\\
	está detrás de todo esto. Que opere un Poder tal es algo que\\
	puedo creer. No puedo ver su forma.\\
	Él actúa, pero no tiene forma.
	
	\chapter*{Tres amigos}
	
	Había tres amigos\\
	discutiendo sobre la vida.\\
	Uno dijo:\\
	``¿Pueden los hombres vivir juntos\\
	y no ser conscientes de ello?\\
	¿Trabajar juntos\\
	y no producir nada?\\
	¿Pueden volar en el espacio\\
	y olvidarse de que existe\\
	el mundo sin fin?''\\
	Los tres amigos se miraron\\
	y rompieron a reír:\\
	No sabían cómo explicarlo.\\
	Así fueron mejores amigos que antes.
	
	Entonces un amigo murió\\
	Confucio mandó a un discípulo\\
	para ayudar a los otros dos a cantar sus exequias.
	
	El discípulo se encontró con que uno de los amigos\\
	había compuesto una canción.\\
	Mientras el otro tocaba un laúd,\\
	cantaron:\\
	``¡Oye, Sung Hu!\\
	¿Dónde te fuiste?\\
	¡Oye, Sung Hu!\\
	¿Dónde te fuiste?\\
	Te has ido\\
	A donde realmente estabas.\\
	Y nosotros estamos aquí.\\
	¡Maldición! ¡Nosotros estamos aquí!
	
	Entonces el discípulo de Confucio los interrumpió y\\
	exclamó: ``¿Puedo preguntarles dónde\\
	han encontrado ustedes esto en las\\
	rúbricas para las exequias,\\
	este frívolo canturrear en presencia del que se ha ido?''
	
	Los dos amigos se miraron y se echaron a reír:\\
	``Pobre tipo'', dijeron. ``¡No conoce la nueva liturgia!''
	
	\chapter*{El velatorio de Lao Tse}
	
	Lao Tan yacía muerto.\\
	Chin Shih asistió al velatorio.\\
	Lanzó tres alaridos\\
	y se fue a casa.
	
	Uno de los discípulos dijo:\\
	``¿No era usted el amigo del Maestro?''\\
	``Desde luego'', respondió.
	
	``¿Entonces le parece suficiente\\
	afligirse tan poco como usted?''
	
	``Al principio'', dijo Chin Shih, ``pensaba que era el más grande entre
	los hombres.\\
	¡Ya no! Cuando vine a condolerme,\\
	encontré viejos lamentándose por él como si fuera su hijo,\\
	hombres sollozando como si fuera su madre.\\
	¿Cómo los ató tanto a sí, sino\\
	por medio de palabras que jamás debió decir\\
	y de lágrimas que jamás debió derramar?\\
	Debilitó su verdadero ser,\\
	depositó carga sobre\\
	carga de emociones, incrementó ese enorme cómputo;\\
	olvidó el regalo que Dios le había confiado:\\
	a esto los antiguos lo llamaban `el castigo\\
	por descuidar el Verdadero Ser?
	
	El Maestro vino al mundo en su momento oportuno.\\
	Cuando se consumió su tiempo,\\
	lo abandonó de nuevo.\\
	Aquel que espera su hora, que se somete,\\
	cuando su labor queda concluida,\\
	no tiene lugar en sí\\
	para el dolor o el regocijo.\\
	Así es como los antiguos expresaban esto\\
	en cuatro palabras: `Dios corta el hilo.'
	
	Hemos visto consumirse un fuego de ramas.\\
	El fuego arde ahora en algún otro sitio.\\
	¿Dónde?\\
	¿Quién sabe? Estos tizones\\
	están ya consumidos''.
	
	\chapter*{El funeral de Chuang Tse}
	
	Cuando Chuang Tse estaba al borde de la muerte, sus discípulos empezaron
	a planear un espléndido funeral.
	
	Pero él dijo: ``Tendré como ataúd el Cielo y la Tierra; el Sol y la Luna
	serán los símbolos de jade que pendan junto a mí; los planetas y las
	constelaciones brillarán como joyas a mi alrededor, y todos los seres
	estarán presentes como comitiva fúnebre en mi velatorio. ¿Qué más me
	hace falta? ¡Todo está suficientemente dispuesto!''
	
	Pero ellos dijeron: ``Tememos que los cuervos y milanos devoren a
	nuestro Maestro''.
	
	``Bien'', dijo Chuang Tse, ``sobre la tierra tendré que ser devorado por
	los cuervos y los milanos; debajo de ella, por las hormigas y los
	gusanos. En cualquier caso, tendré que ser devorado. ¿Por qué tanta
	parcialidad a las aves?''
	
	\chapter*{En mi fin está mi principio}
	
	En el Principio de los Principios estaba el\\
	Vacío de los Vacíos, lo Innominado.\\
	Y en lo Innominado estaba el Uno, sin cuerpo, sin forma.\\
	Este Uno --- este Ser en el cual todo encuentra el poder de existir---\\
	es lo viviente.\\
	De lo viviente procede lo Sin-Forma, lo Indiviso.\\
	Del acto de este Sin-Forma proceden los Existentes,\\
	todos con arreglo a su principio interior.\\
	Eso es la Forma.\\
	Aquí el cuerpo abraza y abriga al espíritu.\\
	Ambos trabajan juntos como uno, aleándose y manifestando\\
	sus Caracteres. Y esto es la Naturaleza.
	
	Pero aquel que obedece a la Naturaleza\\
	vuelve a través de forma y Sin-Forma a lo Viviente.\\
	Y en lo Viviente,\\
	se une al incomenzado Principio.\\
	La unión es la Igualdad. La igualdad es el Vacío.\\
	El Vacío es infinito.\\
	El ave abre su pico y canta su nota\\
	y entonces el `pico se cierra de nuevo en el Silencio.\\
	Así la Naturaleza y lo Viviente se unen en el Vacío.\\
	Como el cerrarse del pico de un ave\\
	después de su canción.\\
	El cielo y la tierra se juntan en lo No Iniciado.\\
	¡Y todo es tontería, todo es desconocido, todo es como\\
	las luces de un idiota, todo carece de mente!\\
	Armonizar es cerrar el pico y caer en el No Inicio.
	
	\chapter*{Inundaciones de otoño}
	
	Las inundaciones de otoño habían llegado. Miles de torrentes salvajes se
	vertían furiosamente en el río Amarillo. Éste engordó e inundó sus
	riberas hasta hacer imposible distinguir un buey de un caballo desde la
	otra orilla. Entonces el Dios del Río se echó a reír, deleitado con el
	pensamiento de que toda la belleza del mundo había caído bajo su tutela.
	De modo que giró corriente abajo hasta llegar al océano. Allí miró por
	encima de las olas hacia el vacío horizonte del este y quedó
	consternado. Oteando el horizonte, recuperó el sentido y murmuró al Dios
	del Océano: ``Bien, el proverbio está en lo cierto. Aquel que se ha
	hecho con ideas piensa que sabe más que cualquier otra persona. Así he
	sido yo. ¡Sólo que ahora comprendo lo que quiere decir EXTENSIÓN!
	
	El Dios del Océano replicó: ``¿Acaso puedes hablar del mar a una rana en
	un pozo?\\
	¿Puedes hablar del hielo a una libélula?\\
	¿Puedes hablar acerca del camino de la Vida a un doctor en filosofía?\\
	De todas las aguas del mundo, el océano es la mayor.\\
	Todos los ríos van a verterse en él día y noche, jamás se llena;\\
	devuelve sus aguas día y noche, jamás se vacía.\\
	En épocas de sequía, no baja el nivel.\\
	En tiempos de inundaciones, no aumenta.\\
	¡Más grande que todas las demás aguas!\\
	¡No existe medida para decir cuánto más grande!\\
	¿Pero estoy orgulloso de ello?\\
	¿Qué soy yo bajo el Cielo?\\
	¿Qué soy yo sin el Yang o el Yin?\\
	Comparado con el cielo, soy una roca diminuta,\\
	un achaparrado roble en la ladera de una montaña.\\
	¿Debería acaso actuar como si fuera algo?''
	
	De todos los seres que existen (y hay millones), el hombre no es más que
	uno. De entre los millones de hombres que viven en la Tierra, la gente
	civilizada que vive del cultivo es tan sólo una pequeña proporción.
	Menores aún son los números de aquellos que, teniendo cargo o fortuna,
	viajan en carruaje o en barco. Y de todos estos, un hombre en su
	carruaje no es más que la punta de un pelo en el costado de un caballo,
	¿Por qué, entonces, tanto alboroto en torno a los grandes hombres y los
	grandes cargos? ¿Por qué tantas disputas entre eclesiásticos? ¿Por qué
	tanta pugna entre políticos?
	
	No hay límites fijos. El tiempo no se detiene.\\
	Nada perdura. Nada es definitivo.\\
	No se puede agarrar el final o el principio.\\
	El que es sabio ve que cerca o lejos es lo mismo.\\
	No desprecia lo pequeño ni valora lo grande.\\
	Donde difieren todos los parámetros, ¿cómo se puede comparar?\\
	Con una mirada, absorbe el pasado y el presente,\\
	sin lástima por el pasado ni impaciencia con el presente.\\
	Todo está en movimiento.\\
	Él tiene la experiencia de la plenitud y el vacío.\\
	No se regocija con el éxito, ni se lamenta del fracaso.\\
	El juego jamás se acaba.\\
	El nacimiento y la muerte están empatados.\\
	Los términos no son definitivos.
	
	\chapter*{Medios y fines}
	
	El portero de la capital de Sung se convirtió en un plañidero tan
	experto tras la muerte de su padre, y se consumió hasta tal punto con
	ayuno y austeridades, que fue promovido a un alto rango para que
	sirviera de modelo. Como resultado de esto, sus imitadores se
	mortificaron hasta tal punto que la mitad de ellos murió. Los restantes
	no fueron ascendidos.
	
	El propósito de una trampa para peces es cazar peces y, cuando éstos han
	sido capturados, la trampa ha sido olvidada.
	
	El propósito de un cepo para conejos es cazar conejos. Una vez
	capturados éstos, el cepo cae en el olvido.
	
	El propósito de las palabras es transmitir ideas. Una vez captada la
	idea, las palabras quedan olvidadas.
	
	¿Dónde podría yo encontrar a un hombre que haya olvidado las palabras?
	¡Es con él con quien me gustaría hablar!
	
	Después de leer los escritos, hay una nueva vida a vivir. En la época de
	Lao Tse y Chuang Tse, ni leer ni practicar eran necesarios, porque el
	mundo no presentaba distracciones de la conciencia de lo Sagrado. Pero,
	al comenzar la civilización, cerca el siglo 10, fueron necesarios
	ejercicios físicos para relajar el cuerpo, los sentidos, y la mente. Son
	una ayuda para revivir lo vacío.
	
	El ejercicio sagrado más poderoso era el T'ai Chi Ch'uan: ``forma
	suprema última''. Hay varios estilos que se practican en el mundo hasta
	ahora. Para seguir explorando la cosmovisión de la China antigua,
	consulta nuestro libro ``El néctar del T'ai Chi Ch'uan'' y otros
	recursos en español e inglés en el sitio web
	\textit{www.nectarproject.org}
	
\end{document}